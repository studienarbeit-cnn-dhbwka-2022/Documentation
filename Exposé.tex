\documentclass{article}

% Language setting
% Replace `english' with e.g. `spanish' to change the document language
\usepackage[english]{babel}

% Set page size and margins
% Replace `letterpaper' with `a4paper' for UK/EU standard size
\usepackage[letterpaper,top=2cm,bottom=2cm,left=3cm,right=3cm,marginparwidth=1.75cm]{geometry}

% Useful packages
\usepackage{amsmath}
\usepackage{graphicx}
\usepackage[colorlinks=true, allcolors=blue]{hyperref}

\title{Exposé - Studienarbeit Marc Gökce & Lukas Hörnle}
\author{Lukas Hörnle 28.02.2023}

\begin{document}
 \maketitle

 \begin{abstract}
  Ein Abstract lohl
 \end{abstract}

 \section{Einführung}
 Was wollen wir ganz ganz grob machen?
 \subsection{die Problemstellung des Schreibprojekts}
 \subsection{Der aktuelle Forschungsstand}
 \subsection{Fragestellung der Arbeit}
 \subsection{das Erkenntnisinteresse des Verfassers}
 \subsection{das Ziel bzw. die der Arbeit zugrundeliegende Hypothese}
 \subsection{die Theorie(n), auf die Bezug genommen werden soll}
 \subsection{die Methode(n), nach der/denen vorgegangen werden soll}
 \subsection{die Quellen bzw. das Material, die/das verwendet werden soll/en}
 \subsection{die vorläufige Gliederung und den Zeitplan bis zum Abgabetermin}
 \subsection{How to include Figures}
 \subsection{Good luck!}
\end{document}