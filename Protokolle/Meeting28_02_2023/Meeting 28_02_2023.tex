\documentclass{article}
%packages
\usepackage[english]{babel}
\usepackage[letterpaper,top=2cm,bottom=2cm,left=3cm,right=3cm,marginparwidth=1.75cm]{geometry}
\usepackage{amsmath}
\usepackage{graphicx}
\usepackage[colorlinks=true, allcolors=blue]{hyperref}
\usepackage{tikz}
\def\checkmark{\tikz\fill[scale=0.4](0,.35) -- (.25,0) -- (1,.7) -- (.25,.15) -- cycle;}

%Titel TODO
\title{Protokoll zu TODO}
\author{Lukas H�rnle}
%Begin of file
\begin{document}
    \maketitle

%Abstract TODO
    \begin{abstract}

    \end{abstract}

%Anwesenheitsliste TODO
    \section{Anwesenheitsliste<}
    \begin{table}
        \centering
        \begin{tabular}{l|r}
            Mitglied & Anwesend \\\hline
            Lukas H�rnle & \checkmark \\
            Marc G�kce & \checkmark \\
            Ralph Lausen & X \\
        \end{tabular}
        \caption{\label< a Anwesenheitsliste}
    \end{table}

%Ziele des Treffens
    \section{Ziele}
    \begin{description}
        \item[Definition der Projektziele]
            Das Ziel und die wissenschaftliche Frage des Projektes sollen definiert werden.  
        \item[Methoden zu Projektzielen]
            Methoden zur Umsetzung der Projektziele sollen festgelegt werden.
            Ein Ablaufplan hilft bei der Visualisierung
        \item[Zeitplan zu Meilensteinen] 
            Es soll ein Zeitplan mit Zielen f�r einzelne Meilensteine des Projektes festgelegt werden.  
    \end{description}

%Umgebung/Durchf�hrung des Treffens TODO 
    %TODO Quelle f�r ChatGPT und einf�gen von Bildern und einf�gen/verweisen zu Zeitplan 
    \section{Umsetzung des Treffens}
    Das Meeting findet am 28.02.2023 von 15 bis 17 Uhr im Raum a264 an der DHBW Karlsruhe statt.
    Mithilfe der online verf�gbaren k�nstlichen Intelligenz \"ChatGPT\" werden Beispielthemen und Ans�tze im Bereich des zuvor definierten Tech Stacks unter Beachtung der Teamgr��e und bereits entschiedenen Themen generiert. 
    Die generierten Themenans�tze werden nach Interesse und Komplexit�t bewertet. 
    Ein Themenansatz mit angemessener Komplexit�t sowie bestehendem Interesse bei allen Beteiligten Entwicklern wird abgestimmt und ausgew�hlt. 
    Eine beispielhafte Umsetzung wird oberfl�chlich mithilfe des Whiteboards erarbeitet. 
    Die Arbeit wird in Arbeitspakete und Meilensteine aufgeteilt. 
    Diesen wird ein Zeitplan zugeordnet.

%Einbindung der Quellen
    \bibliographystyle{alpha}
    \bibliography{28022023}
%Dokumentenende
\end{document}