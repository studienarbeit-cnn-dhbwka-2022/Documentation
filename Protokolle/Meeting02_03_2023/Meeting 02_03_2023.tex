\documentclass{article}
%packages
\usepackage[english]{babel}
\usepackage[letterpaper,top=2cm,bottom=2cm,left=3cm,right=3cm,marginparwidth=1.75cm]{geometry}
\usepackage{amsmath}
\usepackage{graphicx}
\usepackage[colorlinks=true, allcolors=blue]{hyperref}
\usepackage{tikz}
\def\checkmark{\tikz\fill[scale=0.4](0,.35) -- (.25,0) -- (1,.7) -- (.25,.15) -- cycle;}

%Titel TODO
\title{Protokoll zu TODO}
\author{Lukas H�rnle}
%Begin of file
\begin{document}
    \maketitle

%Abstract TODO
    \begin{abstract}

    \end{abstract}

%Anwesenheitsliste TODO
    \section{Anwesenheitsliste<}
    \begin{table}
        \centering
        \begin{tabular}{l|r}
            Mitglied & Anwesend \\\hline
            Lukas H�rnle & \checkmark \\
            Marc G�kce & \checkmark \\
            Ralph Lausen & X \\
        \end{tabular}
        \caption{\label< a Anwesenheitsliste}
    \end{table}

%Ziele des Treffens TODO
    \section{Ziele}
    \begin{description}
        \item[Ein Stichpunkt]
        Hier muss etwas stehen um den Effekt sehen zu k{\"o}nnen
        \item[Noch ein Stichpunkt] und Text dahinter
    \end{description}
%Umgebung/Durchf�hrung des Treffens TODO
    \section{Umsetzung des Treffens}

%Fortschritte des Treffens TODO
    \section{Fortschritte}



%Einbindung der Quellen
    \bibliographystyle{alpha}
    \bibliography{02032023}
%Dokumentenende
\end{document}