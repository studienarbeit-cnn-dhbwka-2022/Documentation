%!TEX root = ../main.tex

\chapter{Anleitung - Copy Paste Snippets}

Copy Paste Beispiele, die in Arbeiten oft benötigt werden!

\section{Best Practices}
\begin{description}[style=nextline]
	\item[1 Satz pro Zeile]  
	Latex ignoriert einzelne Zeilenumbrüche, erst doppelte werden angezeigt.
	Das erlaubt eine deutlich leserlichere Korrektur durch Git.
\end{description}

\section{Absatz im Text}
Lorem ipsum dolor sit amet, consetetur sadipscing elitr, sed diam nonumy eirmod tempor invidunt ut labore et dolore magna aliquyam erat, sed diam voluptua. At vero eos et accusam et justo duo dolores et ea rebum. Stet clita kasd gubergren, no sea takimata sanctus est Lorem ipsum dolor sit amet. Lorem ipsum dolor sit amet, consetetur sadipscing elitr, sed diam nonumy eirmod tempor invidunt ut labore et dolore magna aliquyam erat, sed diam voluptua.
\newline

At vero eos et accusam et justo duo dolores et ea rebum. Stet clita kasd gubergren, no sea takimata sanctus est Lorem ipsum dolor sit amet. Lorem ipsum dolor sit amet, consetetur sadipscing elitr, sed diam nonumy eirmod tempor invidunt ut labore et dolore magna aliquyam erat, sed diam voluptua. At vero eos et accusam et justo duo dolores et ea rebum. Stet clita kasd gubergren, no sea takimata sanctus est Lorem ipsum dolor sit amet.

\section{Zitat}
Zitat \cite{beispiel}

\section{Glossar}
\subsection{Singular}
\Gls{glossary:glossar}
\subsection{Plural}
\Glspl{glossary:glossar}

\section{Akronym}
Dokumentation: \url{https://mirror.physik.tu-berlin.de/pub/CTAN/macros/latex/contrib/acro/acro-manual.pdf} 
\subsection{Automatisch}
Bei der ersten Verwendung Langform, dann nur noch Kurzform: \Ac{acr:ufo}.
\subsection{Immer Kurzform}
\acs{acr:ufo}
\subsection{Immer Langform}
\acf{acr:ufo}
\subsection{Plural}
\acsp{acr:ufo} oder \acfp{acr:ufo}

\section{Tabelle}
\begin{table}[H]
	\caption{Eine Beispiel Tabelle}
	\centering
	\begin{tabular}{ |c|c|c| } 
		\hline
		cell1 & cell2 & cell3 \\ 
		cell4 & cell5 & cell6 \\ 
		cell7 & cell8 & cell9 \\ 
		\hline
	\end{tabular}
	\label{Table:Beispiel}
	\caption[Meine Tabelle]{in TexStudio kann man unter 'Latex/Manipulate Tables/Align Columns' den Tabellencode formatieren :D}
\end{table}

\section{Bilder}
\subsection{Ein Bild}
\begin{figure}[H]
	\centering
	\includegraphics[width=2cm]{kapitel/offizielles/img/cas}
	\caption{Ein Beispiel Bild}
	\label{Image:Beispiel1}
\end{figure}

\subsection{Zwei Bilder nebeneinander}
\begin{figure}[H]
	\centering
	\subfigure[Bezeichnung der linken Grafik]{
		\includegraphics[width=2cm]{kapitel/offizielles/img/cas}
		\label{Image:Beispiel2a}
	}
	\hspace{1em} % space between images
	\subfigure[Bezeichnung der rechten Grafik]{
		\includegraphics[width=2cm]{kapitel/offizielles/img/cas}
		\label{Image:Beispiel2b}
	}
	
	\caption{Hier steht ein wundervoller Titel}
	\label{Image:Beispiel2}
\end{figure}

\subsection{Bild neben Text}
\begin{wrapfigure}{r}{5cm}
	\centering
	\includegraphics[width=4cm]{kapitel/offizielles/img/cas}
	\caption{Ein Beispiel Bild}
	\label{Image:Beispielbild-Rechts}
\end{wrapfigure}
\blindtext

\section{To Do}
\todo[inline,shadow]{Immer schön weiter ergänzen!}

\section{Formel}
\subsection{Mit Name}
\begin{equation}
	\tag{Satz des Pythagoras}
	a^2 + b^2 = c^2
	\label{equation:SatzDesPythagoras}
\end{equation}

\subsection{Mit Nummer}
\begin{equation}
	a^2 + b^2 = c^2
	\label{equation:SatzDesPythagoras}
\end{equation}

\subsection{Im Formelverzeichnis}
\begin{equation}
	a^2+b^2=c^2
\end{equation}
\myequations{Satz des Pythagoras}

\section{Kommentar}
Das Folgende wird nicht dargestellt: 
\begin{comment}
	Lorem ipsum dolor sit amet, consetetur sadipscing elitr, sed diam nonumy eirmod tempor invidunt ut labore et dolore magna aliquyam erat, sed diam voluptua. At vero eos et accusam et justo duo dolores et ea rebum. Stet clita kasd gubergren, no sea takimata sanctus est Lorem ipsum dolor sit amet. Lorem ipsum dolor sit amet, consetetur sadipscing elitr, sed diam nonumy eirmod tempor invidunt ut labore et dolore magna aliquyam erat, sed diam voluptua.
\end{comment}

\section{Abbildungen, Sections oä. referenzieren}
\label{Section:cref}

Schau dir doch mal \cref{Image:Beispiel1} an!
Es ist auch möglich auf die Subfigures \cref{Image:Beispiel2a} und \cref{Image:Beispiel2b} zu verweisen, oder auf die Gesamtgrafik \cref{Image:Beispiel2}.

Ist \cref{Section:cref} nicht ein tolles Kapitel?
