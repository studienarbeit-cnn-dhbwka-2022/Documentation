\documentclass{article}

\usepackage[ngerman]{babel}
\usepackage[top=12mm,bottom=12mm,left=25mm,right=25mm,]{geometry}
\usepackage{setspace}

\title{Titel}
\author{Studienarbeit von Hörnle Lukas und Gökce Marc - Studiengang Informatik DHBW}
\date{}

\begin{document}
\begin{spacing}{1.25}
	\maketitle
	\thispagestyle{empty} %suppress pagenumber
	
	\subsection*{Umschreibung des Themas}
		Verschiedene Methoden der Bildbearbeitung werden eingesetzt um Trainings- und Testdaten für neuronale Netzwerke zu schaffen. Eine Fehlerabschätzung nach dem Training zeigt den Einfluss der bearbeiteten Eingangsdaten auf die Erfolgsquote des Netzwerks. 
		\newline
		
		Zum Einsatz kommen die Netzwerke...//TODO
		Einfluss auf die Wahl der Netzwerke hatten der Schierigkeitsgrad der Implementierung, die Möglichkeit das Netzwerk auf leichter Hardware laufen zu lassen und die Vorabschätzung der Erfolgsquote der genannten Netzwerke
		\newline
		
		Mögliche Probleme, wie ein vanishing beziehungsweise ein exploding gradient werden im Verlauf der Arbeit angesprochen und behandelt.
		
	\subsection*{Projektplan}
		Nach dem Verlust eines Teammitglieds umfasst der scope nun nur noch das implementieren, deployen und trainieren der Netzwerke basierend auf den Frameworks PyTorch und TensorFlow. Die Klassifizierung von Obst in Bildern soll durch eine Erweiterung der Datenbasis durch Filter unterstützt werden. Entsprechende Entwicklungen werden nach gesetzten Trainingspunkten dokumentiert und abgeglichen. 
\end{spacing}
\end{document}
