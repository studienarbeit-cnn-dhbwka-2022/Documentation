\chapter{Transfer Learning}

\section{Einführung in Transfer Learning}

    Transfer Learning ist eine leistungsstarke Technik im Bereich des Deep Learnings, die es uns ermöglicht, das Wissen von vorgefertigten Modellen zu nutzen und es auf neue Aufgaben anzuwenden.
    Dabei werden die erlernten Repräsentationen oder Parameter eines Modells wiederverwendet und auf eine andere verwandte Aufgabe übertragen. 
    Diese Herangehensweise hat in den letzten Jahren aufgrund ihrer Fähigkeit, Zeit und Rechenressourcen zu sparen und dennoch hohe Leistung bei neuen Aufgaben zu erzielen, erhebliche Aufmerksamkeit und Beliebtheit erlangt.
    Die grundlegende Idee hinter Transfer Learning ist, dass Modelle, die auf großen und vielfältigen Datensätzen wie ImageNet trainiert wurden, allgemeine Merkmale gelernt haben, die für eine Vielzahl von visuellen Erkennungsaufgaben nützlich sind. 
    Anstatt den Lernprozess für eine neue Aufgabe mit begrenzten Daten von Grund auf zu beginnen, können wir unser Modell mit den vorab trainierten Gewichten aus einer verwandten Aufgabe initialisieren. 
    Dadurch besitzt das Modell bereits Wissen über niedrig eingestufte Merkmale, Formen und Muster, was in der neuen Aufgabe von Vorteil sein kann.
    Einer der Hauptvorteile von Transfer Learning besteht darin, das Problem des Datenmangels zu umgehen. 
    Das Sammeln und Markieren großer Datenmengen für jede spezifische Aufgabe kann zeitaufwändig und kostspielig sein. 
    Durch die Nutzung von Transfer Learning können wir jedoch auf die große Menge an markierten Daten zurückgreifen, die für das Vortraining verfügbar sind, und somit den Bedarf an einem großen markierten Datensatz für die Ziel-Aufgabe verringern. 
    Dies ist besonders vorteilhaft in Bereichen, in denen das Erlangen von markierten Daten herausfordernd oder nicht praktikabel ist.
    Ein weiterer Vorteil von Transfer Learning liegt in seiner Fähigkeit zur Generalisierung auf neuen Aufgaben. 
    Indem wir Wissen von vortrainierten Modellen übertragen, nutzen wir effektiv die erlernten Repräsentationen, die aussagekräftige Merkmale erfassen. 
    Diese Fähigkeit zur Generalisierung ermöglicht es dem Modell, auch mit begrenzten Trainingsdaten für die Ziel-Aufgabe gute Leistungen zu erbringen. 
    Es trägt auch zur Reduzierung von Overfitting bei, da das Modell bereits aus einem vielfältigen Datensatz gelernt hat und robuste und diskriminierende Merkmale entwickelt hat.
    Darüber hinaus ermöglicht Transfer Learning, von der Expertise und den Forschungsanstrengungen zu profitieren, die in die Entwicklung vortrainierter Modelle investiert wurden. 
    Viele fortschrittliche Modelle und Architekturen wurden auf groß angelegten Datensätzen vortrainiert und erreichen hohe Genauigkeit bei verschiedenen Benchmark-Aufgaben. 
    Indem man diese vortrainierten Modelle als Ausgangspunkt nutzt, kann man ihre Architekturen und Merkmalsextraktoren verwenden und sie für eine spezifische Aufgabe abstimmen. Dadurch kann der Lernprozess beschleunigt werden.
    \footfullcite{pan2010transfer}$^{,}$\footfullcite{yosinski2014transfer}

\section{Pre-Trained Models}

    Ein wichtiger Bestandteil des Transfer Learnings sind vortrainierte Modelle. 
    Diese vortrainierten Modelle sind neuronale Netzwerkmodelle, die auf großen Datensätzen trainiert wurden, in der Regel für eine andere Aufgabe als die aktuelle. 
    Durch das Training auf umfangreichen Datensätzen haben diese Modelle allgemeine Merkmale und Muster erlernt, die für eine Vielzahl von verwandten Aufgaben nützlich sein können.
    Vortrainierte Modelle sind das Ergebnis umfangreicher Trainingsverfahren auf großen Rechenressourcen, um komplexe Muster in den Daten zu erkennen. 
    Typischerweise werden vortrainierte Modelle auf großen Bilderkennungsdatensätzen wie ImageNet trainiert, die Millionen von Bildern mit verschiedenen Klassen umfassen. 
    Während des Trainings lernen diese Modelle, verschiedene visuelle Merkmale wie Kanten, Formen, Texturen und Objekte zu erkennen und zu extrahieren. 
    Durch diese umfangreiche Vorarbeit können vortrainierte Modelle als Ausgangspunkt für andere Aufgaben dienen, indem sie bereits gelernte Merkmale zur Verfügung stellen.
    Die Idee hinter der Verwendung vortrainierter Modelle im Transfer Learning besteht darin, das Wissen und die Merkmale, die in den vortrainierten Modellen enthalten sind, auf neue Aufgaben zu übertragen. 
    Anstatt ein Modell von Grund auf neu zu trainieren, können wir ein vortrainiertes Modell verwenden und es an die spezifischen Anforderungen unserer Aufgabe anpassen. 
    Da vortrainierte Modelle bereits eine gewisse Vorstellung von visuellen Merkmalen haben, können sie uns dabei unterstützen, auch mit begrenzten Daten gute Ergebnisse zu erzielen.
    Bei der Verwendung vortrainierter Modelle müssen wir jedoch beachten, dass die vortrainierte Aufgabe und die aktuelle Aufgabe zusammenpassen sollten.
    Wenn die vortrainierte Aufgabe ähnliche Merkmale oder Konzepte wie die aktuelle Aufgabe beinhaltet, ist die Wahrscheinlichkeit höher, dass das Transfer Learning erfolgreich ist. 
    Zum Beispiel könnte ein vortrainiertes Modell, das auf Bilderkennung trainiert wurde, als Ausgangspunkt für eine Aufgabe der Objekterkennung dienen, da beide Aufgaben visuelle Merkmale nutzen.

\section{Fine-tuning von vortrainierten Modellen}

    Die Feinabstimmung\footfullcite{alshalali2018fine} (Fine-tuning) ist ein wichtiger Schritt im Transfer Learning, der es ermöglicht, ein vortrainiertes Modell für eine spezifische Aufgabe anzupassen. 
    Bei der Feinabstimmung wird das vortrainierte Modell zunächst als Ausgangspunkt verwendet und anschließend werden die Gewichte des Modells mithilfe eines kleineren, auf die spezifische Aufgabe zugeschnittenen Datensatzes aktualisiert. 
    Dieser Prozess erlaubt es dem Modell, sich an die spezifischen Merkmale und Nuancen der neuen Aufgabe anzupassen\footfullcite{han2021pre}.
    Der erste Schritt bei der Feinabstimmung besteht darin, das vortrainierte Modell zu laden, das auf einer großen, allgemeinen Aufgabe trainiert wurde. 
    Dieses Modell verfügt bereits über ein gewisses Verständnis von visuellen Merkmalen, das durch das Training auf umfangreichen Datensätzen erworben wurde. 
    Anschließend werden die oberen Schichten des Modells, die für die spezifische Aufgabe weniger relevant sind, eingefroren, um zu verhindern, dass sie während des Trainings aktualisiert werden. 
    Dies ermöglicht es uns, die vortrainierten Merkmale beizubehalten, während wir die Gewichte der unteren Schichten des Modells anpassen.
    Der nächste Schritt besteht darin, ein kleineres, auf die spezifische Aufgabe zugeschnittenes Datenset zu verwenden, um das Modell zu trainieren. 
    Da die Anzahl der verfügbaren Daten möglicherweise begrenzt ist, ist es wichtig, Overfitting zu vermeiden und gleichzeitig das Modell an die spezifischen Merkmale der neuen Aufgabe anzupassen. 
    Durch die Verwendung eines kleineren Datensatzes können wir die Rechenressourcen effizienter nutzen und den Trainingsprozess beschleunigen.
    Während des Trainingsprozesses werden die Gewichte der unteren Schichten des Modells aktualisiert, um die Merkmale der neuen Aufgabe besser zu erfassen. 
    Da die oberen Schichten des Modells eingefroren sind, bleiben die bereits erlernten Merkmale erhalten, während die Gewichte der unteren Schichten an die neuen Daten angepasst werden. 
    Dadurch kann das Modell spezifische Muster und Merkmale der neuen Aufgabe erlernen, während es gleichzeitig von den allgemeinen Merkmalen des vortrainierten Modells profitiert.
    Die Feinabstimmung bietet mehrere Vorteile\footfullcite{jeddi2020simple}. 
    Erstens ermöglicht sie eine schnellere Konvergenz, da das Modell bereits über eine gute initiale Gewichtung verfügt und somit weniger Trainingsiterationen benötigt werden. 
    Zweitens kann die Leistung des Modells durch die Anpassung an die spezifischen Merkmale der neuen Aufgabe verbessert werden.
    Indem das Modell auf die Besonderheiten der neuen Aufgabe abgestimmt wird, kann es genauere Vorhersagen treffen und bessere Ergebnisse erzielen.

\section{Verwendung von vortrainierten Modellen} % als Merkmalsextraktoren}

    Eine alternative Methode des Transfer Learning besteht darin, das vortrainierte Modell als festen Merkmalsextraktor zu verwenden. 
    Dabei werden die Faltungsschichten des vortrainierten Modells genutzt, um Merkmale aus den Eingabedaten zu extrahieren, die anschließend einem neuen Klassifikator oder Regressor zugeführt werden. 
    Dieser Ansatz bietet verschiedene Vorteile, wie zum Beispiel die Möglichkeit, vortrainierte Modelle auf kleineren Datensätzen einzusetzen.
    Bei dieser Methode werden die Gewichte des vortrainierten Modells beibehalten und die oberen Schichten eingefroren, um sicherzustellen, dass die bereits erlernten Merkmale nicht verändert werden. 
    
    Die Faltungsschichten des Modells dienen dann als Merkmalsextraktoren, die relevante Informationen aus den Eingabedaten extrahieren. 
    Diese Merkmale werden anschließend als Eingabe für einen neuen Klassifikator oder Regressor verwendet, der auf die spezifische Aufgabe abgestimmt ist.
    Der Vorteil dieser Vorgehensweise liegt darin, dass das vortrainierte Modell bereits über ein umfangreiches Wissen verfügt, das auf großen Datensätzen trainiert wurde. 
    Indem wir diese vortrainierten Merkmalsextraktoren verwenden, können wir von dem bereits erlernten Wissen profitieren, ohne das gesamte Modell neu trainieren zu müssen. 
    Dies ist insbesondere dann vorteilhaft, wenn wir nur über einen begrenzten Datensatz verfügen, da das Training eines Modells von Grund auf mit wenigen Daten schwierig sein kann.
    
    Ein weiterer Vorteil besteht darin, dass die Merkmalsextraktionsschichten des vortrainierten Modells bereits gute Ergebnisse bei der Erkennung allgemeiner Merkmale erzielen. 
    Dies ermöglicht es uns, auch auf kleineren Datensätzen aussagekräftige Merkmale zu extrahieren und sie als Eingabe für den Klassifikator oder Regressor zu verwenden. 
    Somit können wir die Vorteile des vortrainierten Modells nutzen, um bessere Ergebnisse auf unseren spezifischen Aufgaben zu erzielen.
    Es ist wichtig zu beachten, dass die Verwendung vortrainierter Modelle als Merkmalsextraktoren bestimmte Einschränkungen hat. 
    Da die oberen Schichten des Modells eingefroren sind, können sie nicht auf die spezifischen Merkmale der neuen Aufgabe angepasst werden. 
    Daher ist diese Methode besonders geeignet, wenn die Merkmale der neuen Aufgabe bereits in den vortrainierten Schichten erfasst werden können. 
    In solchen Fällen bietet die Nutzung der vortrainierten Merkmalsextraktoren eine effiziente Möglichkeit, um genaue Vorhersagen zu treffen.

\section{Anwendung von DeepLabV3 ResNet50}

    In diesem Abschnitt wird die konkrete Umsetzung des Transfer Learning in unserem Projekt durch die Verwendung des DeepLabV3\footfullcite{chen2017rethinking} ResNet50-Modells\footfullcite{he2016deep} erläutert.
    DeepLabV3 ist eine hochmoderne Architektur, die für semantische Segmentierungsaufgaben entwickelt wurde und für ihre herausragende Leistung in der Bildverarbeitung und Szenenanalyse bekannt ist.
    Das ResNet50 dient als grundlegende Netzwerkstruktur innerhalb des DeepLabV3-Frameworks und nutzt Residual-Lernen, um das Training tiefer faltungs-basierter neuronaler Netzwerke zu erleichtern.
    Die Entscheidung, das DeepLabV3 ResNet50-Modell in unserem Projekt zu verwenden, basierte auf einer Vielzahl von Überlegungen\footnote{Basierend auf einer umfangreichen Prüfung verschiedener Faktoren wurde eine fundierte und vielfältig informierte Entscheidung getroffen. Unter Berücksichtigung des Videos \url{https://www.youtube.com/watch?v=aTYVdK3GNxI} wurden geschickt inspirierende Aspekte in die Entscheidungsfindung eingebunden.}.
    Die Architektur des Modells hat sich insbesondere im Bereich der semantischen Segmentierung als äußerst leistungsstark erwiesen, was eine wesentliche Aufgabe in unseren Projektzielen darstellt.
    Durch den Einsatz des DeepLabV3 ResNet50-Modells kann man das umfangreiche Training und die erlernten Repräsentationen nutzen, die das Modell auf großen Datensätzen erfasst hat.
    Dadurch kann das Netzwerk feine semantische Details in unserem spezifischen Anwendungsbereich erkennen.
    Die Verwendung eines vortrainierten Modells wie DeepLabV3 ResNet50 bietet eine Vielzahl von Vorteilen.
    Ein bemerkenswerter Vorteil besteht darin, dass man das Wissen, das durch vorherige Modelle erlangt wurde, in das Projekt übertragen kann, ohne ein neuronales Netzwerk von Grund auf trainieren zu müssen.
    Das vortrainierte Modell enthält eine Vielzahl von allgemeinen Merkmalen und Mustern, die während der umfangreichen Exposition des Modells gegenüber vielfältigen visuellen Daten während seiner Trainingsphase erlernt wurden.
    Durch die Nutzung dieser erlernten Merkmale kann man die Entwicklungsdauer des Projekts verkürzen und die Rechenbelastung, die mit dem Training eines Modells von Grund auf verbunden ist, verringern.
    
    Darüber hinaus bietet das DeepLabV3 ResNet50-Modell eine solide Grundlage für die Feinabstimmung, die es ermöglicht, das Netzwerk an spezifische Aufgaben anzupassen.
    Bei der Feinabstimmung werden die vortrainierten Gewichte des Netzwerks beibehalten und an die Feinheiten des Ziel-Datensatzes angepasst.
    Dieser Prozess ermöglicht eine schnellere Konvergenz des Lernprozesses des Modells, verkürzt die Gesamttrainingszeit und verbessert die endgültige Leistung bei der Segmentierungsaufgabe.
    Die Feinabstimmung ermöglicht es auch, das im vortrainierten Modell vorhandene Wissen an die spezifische Anwendungsdomäne anzupassen.
    Dabei können die allgemeinen Merkmale, die vom Modell extrahiert wurden, verfeinert und an die Feinheiten der Zielaufgabe angepasst werden.
    Zusammenfassend bietet die Einbindung des DeepLabV3 ResNet50-Modells in das Projekt mittels Transfer Learning eine solide Grundlage, um eine präzise semantische Segmentierung zu erreichen. 
    Indem man auf das vorhandene Wissen des Modells zurückgreift und die Vorteile der Feinabstimmung nutzt, kann man die Fachkenntnisse von DeepLabV3 ResNet50 nutzen und sie effektiv auf die einzigartige Anwendungsdomäne anpassen. 
    Diese Vorgehensweise beschleunigt nicht nur den Entwicklungsprozess des Projekts, sondern führt auch zu einer hochwertigen Lösung, indem das Modell von seinem umfangreichen Training mit vielfältigen visuellen Daten profitiert.
    
    Ein entscheidender Aspekt bei der Anwendung des DeepLabV3 ResNet50-Modells besteht darin, dass die vortrainierten Gewichte des Modells eingefroren bleiben, um die extrahierten Merkmale beizubehalten. 
    Die Faltungsschichten des Modells werden verwendet, um Merkmale aus den Eingabedaten zu extrahieren, die dann in einen neuen Klassifizierer oder Regressor eingespeist werden können. 
    Dieser Ansatz ermöglicht es, die Vorteile des vortrainierten Modells als leistungsstarken Feature-Extraktor zu nutzen, während man gleichzeitig einen neuen, auf die spezifische Aufgabe abgestimmten Klassifizierer oder Regressor trainiert.
    Die Verwendung von vortrainierten Modellen als Feature-Extraktoren bietet eine Reihe von Vorteilen. Insbesondere kann man auf kleineren Datensätzen arbeiten, da die vortrainierten Modelle bereits allgemeine Merkmale gelernt haben, die auf verschiedene Aufgaben übertragbar sind. 
    Dadurch wird der Bedarf an großen Trainingsdatensätzen reduziert, was sowohl die Datenbeschaffung als auch die Rechenressourcen erleichtert. 
    Darüber hinaus ermöglicht die Verwendung von vortrainierten Modellen als Feature-Extraktoren eine schnellere Entwicklung von Modellen, da der Schwerpunkt auf der Anpassung des Klassifizierers oder Regressors liegt, anstatt das gesamte Modell von Grund auf neu zu trainieren.
    
    Insgesamt bietet die Anwendung des DeepLabV3 ResNet50-Modells als Feature-Extraktor eine effektive Methode des Transfer Learning, um hochwertige Ergebnisse in der semantischen Segmentierungsaufgabe zu erzielen. 
    Durch die Kombination der Stärken des vortrainierten Modells und der Anpassung an die spezifische Anwendungsdomäne kann man die Effizienz verbessern und gleichzeitig genaue und zuverlässige Ergebnisse erzielen.