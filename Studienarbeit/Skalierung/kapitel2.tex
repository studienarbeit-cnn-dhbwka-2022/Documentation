



\chapter{Grundlagen der Bildverarbeitung und der Skalierung von Bildern}


\section{Einblick in die Bildverarbeitung}
Historie, Entwicklung, aktueller Stand und mögliche Entwicklungen.

\section{Skalierung von Bildern}

\subsection{Arten der Skalierungen: Interpolation und Skalierung}
Die Interpolation und die Skalierung von Bildern oder Bildbereichen sind wichtige Konzepte der Bildverarbeitung. 
Das Verfahren der Interpolation ermöglicht es neue Pixelwerte auf Basis vorgegebener Werte zu berechnen.
Die Skalierung ist eine Anpassung der Bildgröße durch das Ändern der Anzahl von Pixeln oder der Auflösung.

Im Kontext der Bildverarbeitung wird Interpolation häufig verwendet, um die Größe von Bildern zu ändern, ohne dass dabei die Anzahl der Pixel verändert wird. Dazu werden neue Pixelwerte berechnet, indem vorhandene Pixelwerte interpoliert werden. Die Wahl der Interpolationsmethode hat einen großen Einfluss auf die Qualität des interpolierten Bildes. In der Bildverarbeitung gibt es verschiedene Interpolationsmethoden, wie z.B. Nearest-Neighbor-Interpolation, Bilineare Interpolation oder Bicubische Interpolation.

Skalierung hingegen verändert die Größe eines Bildes, indem die Anzahl der Pixel oder die Auflösung verändert wird. Im Gegensatz zur Interpolation wird die Anzahl der Pixel bei der Skalierung verändert, um das Bild kleiner oder größer zu machen. Auch hier hat die Wahl der Skalierungsmethode einen großen Einfluss auf die Qualität des resultierenden Bildes.

\subsection{Bildformate}
Bilder können allgemein als zweidimensionaler Array dargestellt werden.
Historisch gesehen gab es jedoch viele unterschiedliche Formate für Bilder. Zunächst erschufen unterschiedliche Softwareentwickler im Bereich der Bildverarbeitung häufig ihre eigenen Formate.\footfullcite{burger2009digitale}
Einheitliche Standards, wie sie heute im Einsatz sind, etablierten sich erst später.
Ein Vorreiter der modernen Bildformate ist das "Portable Network Graphics Format"\footfullcite{boutellpng}, das 1985 in den USA vorgestellt wurde. 
Moderne Dateiformate zur Speicherung von Bildern werden anhand der Art des Bildes sowie der Kriterien Speicherbedarf und Kompression, Kompatibilität und ihrem Anwendungsbereich bewertet. \footfullcite{burger2009digitale}
\subsubsection{Portable Network Graphics Format}
Portable Network Graphics Format (PNG) setzt einen besonderen Fokus auf eine geringe Komplexität und eine einfache IMplementierung des Standards. 
Der Standard kann frei von jedem genutzt werden.
Des weiteren profitiert das Format von verlustfreier Kompression. \footfullcite{boutell1997png}
PNG unterstützt Vollfarbbilder, Grauwertbilder sowie Indexbilder. \footfullcite{burger2015digitale}
Der PNG-Algorithmus komprimiert Bilder, indem er mehrere Techniken, einschließlich Filterung und Huffman-Codierung anwendet. Zunächst wird das Bild in Blöcke von 16 x 16 Pixeln aufgeteilt und dann wird auf jedem Block ein Filter angewendet, um Redundanzen zu entfernen. Anschließend wird das Ergebnis der Filterung Huffman-codiert, um eine effiziente Darstellung der Daten zu erreichen.
Characteristisch für PNG-Dateien ist auch die Möglichkeit, transparente Flächen einzubauen. 
Der Standard verwendet eine spezielle Methode, um Transparenz darzustellen. 
Diese wird als Alpha-Kanal bezeichnet und ermöglicht es, transparente sowie halbtransparente Bilder zu erstellen.
Die kompekte Komprimmierun des PNG-Formats hat dafür gesorgt, dass der Standard im Internet eine hohe Beliebtheit geniest. \footfullcite{^w3c_png}

\subsubsection{JPG-Format}


\subsubsection{Scalable Vector Graphics}
\begin{quote}
    The main idea motivating SVG was simple: to create a generic document-oriented solution  for graphics that can be adapted to modern media
    \grqq{}~\footnote{\cite{book:729077}}
\end{quote}

Scalable Vector Graphics (SVG) steht für ein Format, das Vektorgafiken basierend auf XML darstellt.
Im Gegensatz zu Rastergrafiken, wie z.B. PNG, die aus Pixeln bestehen und bei Vergrößerung an Schärfe verlieren, sind Vektorgrafiken vektorbasiert und behalten ihre Qualität bei beliebiger Skalierung. 
Der Standard ermöglicht eine besonders effiziente Speicherung von Bildern.
SVG ist außerdem ein offenes Format und unterstützt Interaktivität, Animation und Skripting.\footfullcite{quint2003scalable} 
Da SVG auf XML basiert, kann es auch mit anderen Webtechnologien wie HTML, CSS und JavaScript integriert werden.\footfullcite{mdn_svg} 

\subsubsection{Weitere Standards}
%TODO einführungstext formulieren
\begin{description}
\item[GIF] \\
Ein Format für animierte Rastergrafiken mit einer begrenzten Farbpalette von 256 Farben. Es verwendet eine verlustfreie Kompression, die aber nicht sehr effizient ist. Es ist geeignet für einfache Animationen und Grafiken mit wenigen Farben. \\
\item[TIFF] \\ 
Ein Format für hochauflösende Rastergrafiken ohne Kompression oder mit verlustfreier Kompression. Es wird oft im Druckbereich verwendet, da es viele Optionen für Farbmanagement und Metadaten bietet. Es ist aber nicht sehr kompatibel mit Webbrowsern \\
\item[PSD] \\
Ein Format für Photoshop-Dokumente, das alle Ebenen, Masken, Effekte und andere Informationen speichert. Es ermöglicht eine umfangreiche Bearbeitung von Rastergrafiken, ist aber nur mit Photoshop kompatibel.\\
\item[BMP] \\
Ein Format für unkomprimierte Rastergrafiken mit hoher Qualität. Es wird selten verwendet, da es sehr große Dateien erzeugt und keine Transparenz oder andere Funktionen unterstützt. \\
\item[EPS] \\
Ein Format für vektorbasierte Grafiken, das Kurven, Texte und andere Elemente speichert. Es kann skaliert werden ohne Qualitätsverlust und wird oft im Druckbereich verwendet. Es ist aber nicht sehr kompatibel mit Webbrowsern oder anderen Programmen. \\
\end{description}
Weiterhin gibt es unzählige Standards um Grafiken darzustellen. Diese übersteigen jedoch den Umfang dieser Arbeit.\footfullcite{prepressure_file_formats}\footfullcite{ionos_file_formats}\footfullcite{rabbani2002overview}\footfullcite{marcellin2000overview}\footfullcite{britannica_jpeg}\footfullcite{elsevier_artwork}
\subsection{Wichtige Aspekte der Skalierung}

\section{Anwendungen von Skalierungsmethoden}
% help
