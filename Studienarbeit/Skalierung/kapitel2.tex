



\chapter{Grundlagen der Bildverarbeitung und der Skalierung von Bildern}


\section{Einblick in die Bildverarbeitung}
Historie, Entwicklung, aktueller Stand und mögliche Entwicklungen.

\section{Skalierung von Bildern}

\subsection{Arten von Skalierungen: Interpolation und Skalierung}
Interpolation und Skalierung sind zwei wichtige Konzepte in der Bildverarbeitung. Während Interpolation ein Verfahren ist, um neue Pixelwerte auf Basis von vorhandenen Werten zu berechnen, ermöglicht Skalierung die Anpassung der Größe eines Bildes durch Ändern der Anzahl von Pixeln oder der Auflösung.

Im Kontext der Bildverarbeitung wird Interpolation häufig verwendet, um die Größe von Bildern zu ändern, ohne dass dabei die Anzahl der Pixel verändert wird. Dazu werden neue Pixelwerte berechnet, indem vorhandene Pixelwerte interpoliert werden. Die Wahl der Interpolationsmethode hat einen großen Einfluss auf die Qualität des interpolierten Bildes. In der Bildverarbeitung gibt es verschiedene Interpolationsmethoden, wie z.B. Nearest-Neighbor-Interpolation, Bilineare Interpolation oder Bicubische Interpolation.

Skalierung hingegen verändert die Größe eines Bildes, indem die Anzahl der Pixel oder die Auflösung verändert wird. Im Gegensatz zur Interpolation wird die Anzahl der Pixel bei der Skalierung verändert, um das Bild kleiner oder größer zu machen. Auch hier hat die Wahl der Skalierungsmethode einen großen Einfluss auf die Qualität des resultierenden Bildes.


\subsection{Bildformate}
\subsection{Wichtige Aspekte von Skalierung}


\section{Anwendungen von Skalierungsmethoden}

\subsection{Bildverarbeitung in der Medizin}

\subsection{Videokompression und Streaming}

\subsection{Virtual Reality und Gaming}

\subsection{Andere Anwendungen}
