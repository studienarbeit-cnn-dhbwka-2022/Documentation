%&bericht

%%%%%%%%%%%%%%%%%%%%%%%%%%%%%%%%%%%%%%%%%%%%%%%%%%%%%%%%%%%%%%%%%%%%%%%%%%%%%%%
%% Descr:       Vorlage für Berichte der DHBW-Karlsruhe
%% Author:      Prof. Dr. Jürgen Vollmer, juergen.vollmer@dhbw-karlsruhe.de
%% $Id: bericht.tex,v 1.25 2020/03/13 15:07:45 vollmer Exp $
%%  -*- coding: utf-8 -*-
%%%%%%%%%%%%%%%%%%%%%%%%%%%%%%%%%%%%%%%%%%%%%%%%%%%%%%%%%%%%%%%%%%%%%%%%%%%%%%%

\documentclass[
   ngerman          % neue deutsche Rechtschreibung
  ,a4paper          % Papiergrösse
% ,twoside          % Zweiseitiger Druck (rechts/links)
% ,10pt             % Schriftgrösse
  ,12pt
% ,12pt
  ,pdftex
%  ,disable         % Todo-Markierungen auschalten
]{report}

\renewcommand{\baselinestretch}{1.5}    % Set line spacing to 1.5

% Bitte die Codierung Ihrer Dateien auswählen:
% \usepackage[latin1]{inputenc}    % Für UNIX mit ISO-LATIN-codierten Dateien
% \usepackage[applemac]{inputenc}  % Für Apple Mac
% \usepackage[ansinew]{inputenc}   % Für Microsoft Windows
\usepackage[utf8]{inputenc}        % UTF-8 codierte Dateien
                                   % Dieses Dokument ist unter Unix erstellt, daher
                                   % wird diese Input-Codierung benutzt.

\usepackage{bericht}
\usepackage{amsmath}
\usepackage{amssymb}
%% ACHTUNG, wenn man eine eigene Formatdatei (bericht.fmt) benutzt, werden Änderungen an bericht.sty
%% erst wirksam, wenn die Format-Datei neu erzeugt wurde!!!
%% Genauer alle Änderungen, die textuell vor der nächsten Zeile ".... endofdump...." stehen
%% werden erst wirksam, wenn die Formatdatei neu erzeugt wurde
\csname endofdump\endcsname

%%%%%%%%%%%%%%%%%%%%%%%%%%%%%%%%%%%%%%%%%%%%%%%%%%%%%%%%%%%%%%%%%%%%%%%%%%%%%%%
%% Angaben zur Arbeit
%%%%%%%%%%%%%%%%%%%%%%%%%%%%%%%%%%%%%%%%%%%%%%%%%%%%%%%%%%%%%%%%%%%%%%%%%%%%%%%

\newcommand{\Autor}{Lukas Hörnle \& Marc Gökce}
\newcommand{\MatrikelNummer}{6828354 \& 4587590}
\newcommand{\Kursbezeichnung}{TINF20B4}

\newcommand{\FirmenName}{CAS Software AG \& 1\&1 }
\newcommand{\FirmenStadt}{Karlsruhe}


% Falls es kein Firmenlogo gibt:
  \newcommand{\FirmenLogoDeckblatt}{}

\newcommand{\BetreuerDHBW}{Professor Dr. Ralph Lausen}%TODO Titel von Lausen raussuchen

%%%%%%%%%%%%%%%%%%%%%%%%%%%%%%%%%%%%%%%%%%%%%%%%%%%%%%%%%%%%%%%%%%%%%%%%%%%%%%%%%%%%%

% Wird auf dem Deckblatt und in der Erklärung benutzt:
\newcommand{\Was}{Studienarbeit}
%\newcommand{\Was}{Projektarbeit}
%\newcommand{\Was}{Studienarbeit}
%\newcommand{\Was}{Bachleorarbeit}

%%%%%%%%%%%%%%%%%%%%%%%%%%%%%%%%%%%%%%%%%%%%%%%%%%%%%%%%%%%%%%%%%%%%%%%%%%%%%%%%%%%%%

\newcommand{\Titel}{
Evaluation verschiedener Bildverarbeitungsmethoden und neuronaler Netze sowie Implementierung eines Bildverarbeitungsverfahrens zur Segmentierung und Auswertung von Füllständen in Bildern}
\newcommand{\AbgabeDatum}{22.05.2023}

\newcommand{\Dauer}{300 Stunden x 2}

% \newcommand{\Abschluss}{Bachelor of Engineering}
\newcommand{\Abschluss}{Bachelor of Science}

\newcommand{\Studiengang}{Angewandte Informatik}
% \newcommand{\Studiengang}{Informatik / Angewandte Informatik}

\hypersetup{%%
  pdfauthor={\Autor},
  pdftitle={\Titel},
  pdfsubject={\Was}
}

%%%%%%%%%%%%%%%%%%%%%%%%%%%%%%%%%%%%%%%%%%%%%%%%%%%%%%%%%%%%%%%%%%%%%%%%%%%%%%%

% Wenn \includeonly{..} benutzt wird, werden nur diese Kaptitel ausgegeben.
\includeonly{
  abk
 ,kapitel1
 ,kapitel2
 ,kapitel3
 ,kapitel4
 ,kapitel5
 ,kapitel6
 ,kapitel7
 ,kapitel8
 ,kapitel9
 ,kapitel10
 ,kapitel11
 ,kapitel12
 ,kapitel13
 ,kapitel14
 ,changelog
}

%%%%%%%%%%%%%%%%%%%%%%%%%%%%%%%%%%%%%%%%%%%%%%%%%%%%%%%%%%%%%%%%%%%%%%%%%%%%%%%

% Benutzt man das "biblatex"-Paket, dann muß das hier stehen:
% siehe auch die mit BIBLATEX markierten Zeilen in bericht.sty
\bibliography{bericht}

\begin{document}

%%%%%%%%%%%%%%%%%%%%%%%%%%%%%%%%%%%%%%%%%%%%%%%%%%%%%%%%%%%%%%%%%%%%%%%%%%%%%%%

\begin{titlepage}
\begin{center}
\vspace*{-2cm}
\FirmenLogoDeckblatt\hfill\includegraphics[width=4cm]{img/dhbw-logo.png}\\[2cm]
{\Huge \Titel}\\[1cm]
{\Huge\scshape \Was}\\[1cm]
{\large für die Prüfung zum}\\[0.5cm]
{\Large \Abschluss}\\[0.5cm]
{\large des Studienganges \Studiengang}\\[0.5cm]
{\large an der}\\[0.5cm]
{\large Dualen Hochschule Baden-Württemberg Karlsruhe}\\[0.5cm]
{\large von}\\[0.5cm]
{\large\bfseries \Autor}\\[1cm]
{\large Abgabedatum \AbgabeDatum}
\vfill
\end{center}
\begin{tabular}{l@{\hspace{2cm}}l}
Bearbeitungszeitraum	         & \Dauer 			\\
Matrikelnummer	                 & \MatrikelNummer		\\
Kurs			         & \Kursbezeichnung		\\
Ausbildungsfirma	         & \FirmenName			\\%TODO hier steht aus irgendeinem Grund eine 1 
			      \empty    & \FirmenStadt			\\
Gutachter der Studienakademie	 & \BetreuerDHBW		
\end{tabular}
\end{titlepage}

%%%%%%%%%%%%%%%%%%%%%%%%%%%%%%%%%%%%%%%%%%%%%%%%%%%%%%%%%%%%%%%%%%%%%%%%%%%%%%%

\input{erklaerung.tex}

%%%%%%%%%%%%%%%%%%%%%%%%%%%%%%%%%%%%%%%%%%%%%%%%%%%%%%%%%%%%%%%%%%%%%%%%%%%%%%%

\newpage
\tableofcontents           % Inhaltsverzeichnis hier ausgeben

% Jetzt kommt der "eigentliche" Text




\chapter{Einleitung}
    
    \begin{center}
        \begin{figure}[h]
            \includegraphics[width=\textwidth]{img/PIA23645_PaleBlueDotRevisited_1600.jpg}
            \caption{``The Pale Blue Dot'' Feb. 14, 1990, by NASA\footnotemark.}
            \vspace{-6mm}
            \label{fig:my_label}
        \end{figure}
        \footnotetext{\fullcite{nasa.bluedot}}
    \end{center}
    
    ''Ein Bild sagt mehr aus als tausend Worte.`` Dieses bekannte Sprichwort drückt aus, wie mächtig Bilder als Kommunikationsmittel sind. 
    Bilder beinhalten Informationen, vermitteln Emotionen, erzählen Geschichten und sind ein Fenster in die Vergangenheit. 
    Bilder sind in der modernen Gesellschaft omnipräsent und im Alltag digital als auch analog unentbehrlich.
    Bilder unterscheiden sich je nach Aufnahme in den verschiedenen Eigenschaften ihrer Speicherung und Darstellung. 
    Zwei dieser Eigenschaften sind die Größe und die Auflösung eines Bildes. 
    Die Größe eines digitalen Bildes gibt an, wie viele Pixel es enthält, während die Auflösung eines Bildes angibt, wie viele Pixel pro Flächeneinheit vorhanden sind \ac{PPI}.
    
    Die Größe und die Auflösung eines Bildes haben Einfluss auf seine Qualität und seinen Speicherplatzbedarf. 
    Um ein Bild für einen bestimmten Zweck zu nutzen, muss es häufig in seiner Größe und oder Auflösung verändert werden. 
    Der Vorgang zur Veränderung der Größe und Auflösung wird als Bildskalierung bezeichnet\footfullcite{techlib.scaling}\footfullcite{abcdef.scaling} und ist eine grundlegende Operation in der digitalen Bildverarbeitung.
    Bildskalierung ist eine grundlegende Methode der Bildverarbeitung. Bildskalierung erlaubt die Änderung der Größe eines digitalen Bildes.
    Eine Gute Bildskalierung misst sich an ihren Eigenschaften in den Bereichen Rechenaufwand und Qualitätsverlust. 
    Besonders wichtig ist der Qualitätsverlust, wenn man Bilder größer skaliert. 
    Ein geeignetes Modell um diesen Prozess zu erklären ist das übertragen einer Zeichnung von einem kleinen Papier auf eine große Leinwand. 
    Wird die Zeichnung lediglich unbedacht vergrößert, wird diese unscharf und verliert an Details. 
    Das Ziel einer guten Bildskalierung ist es, diesen Effekt zu verhindern und die Zeichnung größenunabhängig scharf und detailreich darzustellen.
    
    Es gibt viele verschiedene Methoden, um die Größe eines Bildes zu ändern. 
    Klassische Methoden verwenden Interpolationstechniken, die neue Pixel aus den vorhandenen Pixeln berechnen. 
    Diese Methoden sind schnell und stellen einen geringen Rechenaufwand in Kombination mit geringer Komplexität dar. 
    Jedoch kommt es mit diesen Algorithmen oft zu Qualitätsverlusten oder der Erzeugung von Artefakten. 
    Moderne Anwendungen zur Skalierung von Bildern verwenden Deep-Learning-Techniken wie Convolutional Neural Networks \ac{CNN} oder Generative Adversarial Networks \ac{GAN}, die neue Pixel aus einem trainierten Modell erzeugen. 
    Diese neuen Methoden sind komplex und benötigen mehr Rechenaufwand, können allerdings die Qualität des Bildes verbessern oder kreative Effekte erstellen.
    
    \begin{figure}[h!]
        \vspace{8mm}
        \centering
        \includegraphics{img/xaR8r.png}
        \caption{Verschiedene Beispiele von upscaling Algorithmen\cite{whuber.lanczos}.}
        \label{fig:my_label}
        \vspace{4mm}
    \end{figure}
    
    Die Bildskalierung hat heute viele Anwendungen in verschiedenen Bereichen wie beispielweise Webdesign, Fotografie, Druck oder Videotechnik. 
    Es gibt auch in modernen Anwendungen verschiedene Arten von Skalierungsverfahren, die sich in ihrer Funktionsweise und ihrem Ergebnis unterscheiden. 
    Diese Arbeit schafft einen Überblick über klassische und moderne Skalierungsverfahren sowie deren ihre Vor- und Nachteile. 
    Diese werden anhand von Beispielen inszeniert. 
    Zuletzt wird basierend auf der Evaluierung der verschiedenen Verfahren eine Empfehlung für die beste Skalierungsmethode für verschiedene Bildtypen gegeben. 

    In dieser Studienarbeit wird das zentrale Anliegen verfolgt mittels einer akribischen Untersuchung das optimale Gleichgewicht zwischen Komplexität, Rechenaufwand und Resultaten zu ermitteln. 
    Darüber hinaus erfolgt eine umfassende Erläuterung der Bewertungskriterien, die bei der Evaluierung solcher Verfahren Anwendung finden. 
    Das Firschungsvorhaben fokussiert sich auf die systematische Analyse unterschiedlicher Verfahren zur Skalierung von Bildern, um eine eine fundierte Empfehlung hinsichtlich der optimalen Methode abzugeben. 
    Die Berücksichtigung von Aspekten wie Algorithmuskomplexität, Rechenressourcen und erzielten Ergebnissen ist dabei von essenzieller Bedeutung. 
    Ferner werden präzise Kriterien erörtert, die zur objektiven Bewertung und Vergleichbarkeit der verschiedenen Skalierungsmethoden herangezogen werden können.
    Diese Studienarbeit strebt an, durch eine methodische Herangehensweise das Potenzial verschiedener Bildskalierungsmethoden zu evaluieren, um eine optimale Lösung zu finden, die ein ausgewogenes Verhältnis zwischen algorithmischer Komplexität, Ressourcenverbrauch und qualitativen Resultaten bietet. 
    Zusätzlich wird eine umfassende Beschreibung der Kriterien angestrebt, welche zur Beurteilung und Vergleichbarkeit dieser Methoden genutzt werden können.
    Hierzu werden zuerst die wichtigsten Konzepte der digitalen Bildverarbeitung und der Skalierung von Bildern erklärt und einige Beispiele für ihre Anwendung aufgezeigt. 
    Danach werden die traditionellen Skalierungsmethoden vorgestellt und  ihre Stärken sowie Schwächen verglichen. 
    Anschließend evaluiert diese Arbeit die neueren Skalierungsmethoden und vergleicht ihre Stärken sowie Schwächen. 
    Abschließend bewertet diese Studienarbeit die verschiedenen Methoden mit verschiedenen Maßstäben für die Bildqualität und gibt eine Empfehlung für die Auswahl einer passenden Methode. 
    Die Arbeit fasst sämtliche in ihr erarbeiteten Ergebnisse zusammen und bespricht deren Bedeutung bezüglich Möglichkeiten und Einschränkungen.
%----------------------
    \newpage





\chapter{Grundlagen der Bildverarbeitung und der Skalierung von Bildern}


\section{Einblick in die Bildverarbeitung}
Historie, Entwicklung, aktueller Stand und mögliche Entwicklungen.

\section{Skalierung von Bildern}

\subsection{Arten von Skalierungen: Interpolation und Skalierung}
Interpolation und Skalierung sind zwei wichtige Konzepte in der Bildverarbeitung. Während Interpolation ein Verfahren ist, um neue Pixelwerte auf Basis von vorhandenen Werten zu berechnen, ermöglicht Skalierung die Anpassung der Größe eines Bildes durch Ändern der Anzahl von Pixeln oder der Auflösung.

Im Kontext der Bildverarbeitung wird Interpolation häufig verwendet, um die Größe von Bildern zu ändern, ohne dass dabei die Anzahl der Pixel verändert wird. Dazu werden neue Pixelwerte berechnet, indem vorhandene Pixelwerte interpoliert werden. Die Wahl der Interpolationsmethode hat einen großen Einfluss auf die Qualität des interpolierten Bildes. In der Bildverarbeitung gibt es verschiedene Interpolationsmethoden, wie z.B. Nearest-Neighbor-Interpolation, Bilineare Interpolation oder Bicubische Interpolation.

Skalierung hingegen verändert die Größe eines Bildes, indem die Anzahl der Pixel oder die Auflösung verändert wird. Im Gegensatz zur Interpolation wird die Anzahl der Pixel bei der Skalierung verändert, um das Bild kleiner oder größer zu machen. Auch hier hat die Wahl der Skalierungsmethode einen großen Einfluss auf die Qualität des resultierenden Bildes.


\subsection{Bildformate}
\subsection{Wichtige Aspekte von Skalierung}


\section{Anwendungen von Skalierungsmethoden}

\subsection{Bildverarbeitung in der Medizin}

\subsection{Videokompression und Streaming}

\subsection{Virtual Reality und Gaming}

\subsection{Andere Anwendungen}


\chapter{Klassische Skalierungsmethoden}

\section{Pixel-Verdopplung}
Die Pixel-Verdopplung vergrößert das Bild indem jeder Pixel dupliziert wird.
Diese Methode kann schnell und einfach umgesetzt werden, indem jeder Pixelwert einfach auf den Nachbarpixel übertragen wird. 
Wenn Bilder mit dieser Methode stark vergrößert werden, ergeben sich oft pixelige und unscharfe Ausgaben, da die Details nicht wirklich vorhanden sind, sondern nur durch die Duplizierung von Pixeln aufgefüllt werden. 
Aus diesem Grund wird Pixel-Verdopplung oft als eine minderwertige Skalierungsmethode betrachtet und findet in professinellen Anwendungen selten Gebrauch.\footfullcite{WANG1983363}
\newpage
Eine Beispielhafte Implementierung in Python sieht folgendermaßen aus: 
\begin{lstlisting}
import numpy as np
import cv2

def pixel_doubling(image, scale_factor):
    new_size = (int(image.shape[1] * scale_factor), int(image.shape[0] * scale_factor))
    
    scaled_image = np.zeros(new_size + (image.shape[2],), dtype=np.uint8)
    for i in range(new_size[0]):
        for j in range(new_size[1]):
            x = int(i / scale_factor)
            y = int(j / scale_factor)
            scaled_image[j, i] = image[y, x]
    
    return scaled_image

image = cv2.imread('cactus.jpg')
scaled_image = pixel_doubling(image, 2)
cv2.imshow(scaled_image)
\end{lstlisting}
%TODO @Marc kannst du mir ne Schöne Grafik machen, wo du diesen Code kurz über unser Beispielbild laufen lässt und wir so nen rechts/links Vergleich haben? 
\section{Nearest-Neighbor-Interpolation}
Die Nearest-Neighbor-Interpolation ist eine weitere Methode zur Skalierung von Bildern. 
Es wird für jeden Pixel im Ausgabebild der am nächsten liegende Pixel im Eingabebild ausgewählt und der Farbwert des ausgewählten Pixels wird als Farbwert des entsprechenden Pixels im Ausgabebild verwendet.
Die Verwendung von Nearest-Neighbor-Interpolation ist einfach und schnell zu implementieren. 
Aufgrund ihrer geringen Komplexität ist sie daher sehr beliebt. 
Die Methode eignet sich besonders gut für die Vergrößerung von Bildern mit großen, einheitlichen Bereichen oder harten Kanten. 
Bei der Verkleinerung von Bildern erleiden diese jedoch oft einen Qualitätsverlust.
Hier kommt es zu Unschärfe und Blockbildung. 
Diese Effekt verstärkt sich, wenn das Verhältniss zwischen Quellbild und Audgabebild kein Vielfaches ist. 
\begin{acronym}
  \acro{NNI}{Nearest Neighbor Interpolation}
\end{acronym}
\begin{lstlisting}
import numpy as np
import cv2

def nearest_neighbor_interpolation(image, scale_factor):
    new_size = (int(image.shape[1] * scale_factor), int(image.shape[0] * scale_factor))
    
    scaled_image = np.zeros(new_size + (image.shape[2],), dtype=np.uint8)
    for i in range(new_size[0]):
        for j in range(new_size[1]):
            x = int(i / scale_factor)
            y = int(j / scale_factor)
            scaled_image[j, i] = image[y, x]
    
    return scaled_image


image = cv2.imread('example_image.jpg')
scaled_image = nearest_neighbor_interpolation(image, 2)
cv2.imshow(image)
cv2.imshow(scaled_image)
\end{lstlisting}\footfullcite{jiang2015quantum}
%TODO @Marc kannst du mir ne Schöne Grafik machen, wo du diesen Code kurz über unser Beispielbild laufen lässt und wir so nen rechts/links Vergleich haben? 
\section{Bilineare Interpolation}
\begin{lstlisting}
import numpy as np
from scipy import interpolate
from PIL import Image

def bilinear_interpolation(img, scale):
    """
    Performs bilinear interpolation on the input image with a given scale factor.
    """
    width, height = img.shape
    new_width = int(width * scale)
    new_height = int(height * scale)
    
    x_scale = np.arange(0, new_width, 1) / scale
    y_scale = np.arange(0, new_height, 1) / scale
    
    interpolator = interpolate.interp2d(np.arange(height), np.arange(width), img, kind='linear')
    new_img = interpolator(y_scale, x_scale)
    
    return new_img.astype(np.uint8)

# Beispielaufruf
img = np.array(Image.open('bild.jpg').convert('L'))
scaled_img = bilinear_interpolation(img, 2)
\end{lstlisting}\footfullcite{1409828}

\section{Bicubische Interpolation}

\section{Lanczos-Interpolation}
\section{Vor- und Nachteile der klassischen Methoden}




\chapter{Fortgeschrittene Skalierungsmethoden}
%Kollege Hase Otter ich muss Abbreviations richtig in LaTeX deklarieren 
\section{Convolutional Neural Networks / Deep learning}
    \subsection{Grundlagen von Convolutional Neural Networks (CNNs)}
%TODO Abkürzung in latex 
    Convolutional Neural Networks (CNNs) sind eine Art von Deep-Learning-Modell, welches besonders im Hinblick auf die Verarbeitung von Daten mit räumlicher Struktur den aktuellen Stand der Technik darstellt. 
Räum.iche Daten, wie z.B. Bilder können bearbeitet, verarbeitet, erstellt und analysiert werden. 
CNNs bestehen aus mehreren Schichten von Neuronen, die so angeordnet sind, dass sie räumliche Beziehungen in den Daten erfassen können.
    
    Die grundlegende Idee hinter CNNs ist die Verwendung von Faltung (engl. convolution) anstelle der vollständigen Verbindung (engl. fully connected) zwischen den Schichten. 
Dies bedeutet, dass jedes Neuron in einer Schicht nur mit einem Teil des Eingangs verbunden ist, anstatt mit jedem Eingangsneuron. 
Diese Art der Verbindung spart Rechenleistung und ermöglicht eine effektivere sowie schnellere Verarbeitung von großen Datensätzen.
    \footfullcite(kolbentwicklung,oshea2015introduction)
    \subsection{Architekturen von CNNs}
    
    Es gibt mehrere bekannte Architekturen von CNNs, darunter AlexNet, ResNet und Inception. 
AlexNet war das erste CNN, das auf einem großen Datensatz erfolgreich angewendet wurde. 
ResNet zeichnet sich durch seine Fähigkeit aus, sehr tiefe Netzwerke zu trainieren, ohne dass das Problem des Verschwindens des Gradienten auftritt. 
Inception wiederum ist für seine Fähigkeit bekannt, die Effizienz von CNNs durch die Verwendung von sogenannten Inception-Modulen zu erhöhen.
//TODO Grafiken für die Netzwerke raussuchen (BV Vorlesung?)

    
    \subsection{Anwendungen von CNNs}
    
    CNNs haben zahlreiche Anwendungen, darunter Bildklassifizierung, Objekterkennung und Gesichtserkennung. 
Bei der Bildklassifizierung werden Bilder automatisch in verschiedene Kategorien eingeteilt. 
Beispielsweise können Bilder in Klassen wie Hunde, Katzen oder Autos eingeteilt werden. 
Bei der Objekterkennung wird das Modell darauf trainiert, bestimmte Objekte in einem Bild zu erkennen, wie z.B. Personen oder Straßenschilder. 
Die Gesichtserkennung wird oft zur Identifikation von Personen in Sicherheitsanwendungen eingesetzt.
%TODO Grafik für Ergebnisse von Klassifizierung, Objekterkennung etc einfgn
    \footfullcite(oshea2015introduction)
    \subsection{Transfer Learning mit CNNs}
    
    Transfer Learning ist eine Technik, bei der ein bereits trainiertes CNN auf eine neue Aufgabe angewendet wird, ohne es von Grund auf neu zu trainieren. 
    Dies ist nützlich, wenn man nur über begrenzte Trainingsdaten verfügt oder wenn das Trainieren eines neuen Modells zu viel Zeit oder Ressourcen in Anspruch nimmt. 
    Ein Beispiel hierfür ist ????????, bei denen das Modell auf einem bereits trainierten CNN basieren kann, das auf ?????? trainiert wurde.
    %TODO mir is kein Beispiel eingefallen
\subsection{Limitationen von CNNs und aktuelle Forschungsrichtungen}

    Obwohl CNNs sehr erfolgreich bei der Verarbeitung von Bildern sind, haben sie auch einige Limitationen. 
    Zum Beispiel sind sie nicht gut geeignet, um komplexe Abhängigkeiten zwischen verschiedenen Eingabemerkmalen zu erfassen, wie z.B. das Verhalten von Objekten in einem Video.
    Zudem benötigen CNNs weiterhin viel Rechenlleistung und Ressourcen

\section{Super Resolution}
    \subsection{Grundlagen von Super Resolution (SR)}
    
    Super Resolution (SR) ist eine Technik, um aus einer niedrig aufgelösten Eingabe ein hochauflösendes Bild zu generieren. 
    Dies wird oft als Upscaling bezeichnet und findet in vielen Anwendungen wie der Bildrekonstruktion und Videoanalyse Anwendung.
    Die Grundidee hinter SR ist, dass hochauflösende Informationen in einem niedrig aufgelösten Bild versteckt sein können. 
    Die Herausforderung besteht darin, diese Informationen zu extrahieren und in ein hochauflösendes Bild zu integrieren. 
    SR ist somit ein Problem der inversen Bildgebung, bei dem eine hohe Auflösung aus einer niedrigen Auflösung abgeleitet werden muss.
    \footfullcite(7115171)
    \subsection{Super Resolution-Methoden auf Basis von Deep Learning}
    
    Super Resolution-Methoden auf Basis von Deep Learning haben in den letzten Jahren viel Aufmerksamkeit erhalten und sind derzeit der Stand der Technik für SR. 
    Diese Methoden verwenden Convolutional Neural Networks (CNNs) zur Verarbeitung von Bildern und zur Generierung von hochauflösenden Bildern.
    Es gibt verschiedene Arten von SR-Methoden auf Basis von Deep Learning, darunter Single-Image Super Resolution (SISR) und Multi-Image Super Resolution (MISR). 
    SISR-Methoden verwenden nur ein niedrig aufgelöstes Bild als Eingabe, während MISR-Methoden mehrere Bilder verwenden, um ein hochauflösendes Bild zu generieren.
    %TODO mehr auf sisr und misr eingehen hab ich aber nur semi verstanden so far 
    
    \subsubsection{Anwendungen von SR}
    
    SR hat viele Anwendungen in der Bild- und Videoanalyse, einschließlich der Rekonstruktion von Bildern aus medizinischen Scans, der Verbesserung von Bildern für die forensische Analyse und der Verbesserung von Bildern für die Erkennung von Gesichtern und Objekten.
    In der Videoanalyse kann SR verwendet werden, um Videos zu stabilisieren, indem Bewegungsunschärfe reduziert und die Schärfe der Bilder verbessert wird. 
    SR kann auch bei der Entschlüsselung von unscharfen und verschwommenen Bildern in Überwachungsaufnahmen helfen.
    
    \subsubsection{Evaluierung von SR-Methoden}
    
    Die Evaluierung von SR-Methoden ist eine wichtige Aufgabe, um die Qualität und Effektivität der generierten Bilder zu bestimmen. 
    Die gängigen Evaluierungsmethoden umfassen die Verwendung von visuellen Qualitätsmetriken wie Peak Signal-to-Noise Ratio (PSNR) und Structural Similarity Index Measure (SSIM).
    Es gibt auch speziellere Evaluierungsmethoden wie die Verwendung von Perceptual Quality Assessment (PQA)-Maßnahmen, die menschliche Wahrnehmungseigenschaften berücksichtigen, um die Qualität der generierten Bilder zu bestimmen.
    
    \subsubsection{Herausforderungen und zukünftige Forschungsziele von Super Resolution}
    
    Obwohl SR-Methoden auf Basis von Deep Learning vielversprechende Ergebnisse erzielt haben, gibt es immer noch Herausforderungen und zukünftige Forschungsziele, die erforscht werden müssen.
    Eine der Herausforderungen besteht darin, dass SR-Methoden häufig dazu neigen, Artefakte in den generierten Bildern zu erzeugen, insbesondere bei der Verwendung von sehr hohen Upscaling-Faktoren. %TODO Beispielbild Artefakte erklären
    Dies kann die visuelle Qualität der generierten Bilder beeinträchtigen und die Anwendbarkeit von SR-Methoden in bestimmten Szenarien einschränken.
    Eine weitere Herausforderung besteht darin, dass SR-Methoden häufig sehr rechenaufwändig sind, insbesondere wenn sie auf großen Datensätzen oder in Echtzeit angewendet werden müssen. 
    Die benötigten Ressourcen sind teuer.
    Dies kann die praktische Anqwendbarkeit von SR-Methoden in einigen Anwendungen einschränken.   
    Zukünftige Forschungsziele könnten sich darauf konzentrieren, diese Herausforderungen zu überwinden, indem sie neue SR-Methoden entwickeln, die sowohl effektiv als auch effgizient sind. 
    Eine mögliche Lösung wäre die Verwendung von Generative Adversarial Networks (GANs) zur Verbesserung der visuellen Qualität der generierten Bilder und zur Reduzierung von Artefakten. %TODO näher auf GAN eingehen 
    Eine weitere mögliche Lösung wäre die Entwicklung von neuartigen Architekturen von Deep Learning-Netzwerken, die weniger rechenaufwändig sind und schneller ausgeführt werden können.
    % I want a tea party with Y'ha-nthlei
    Insgesamt bleibt SR ein aktives Forschungsfeld mit großem Potenzial für Anwendungen in der Bild- und Videoanalyse. 
    Mit weiteren Fortschritten in der Forschung können SR-Methoden immer leistungsfähiger und praktischer werden, um die Bedürfnisse der Industrie und der Geselllschaft zu erfüllen.

\section{Generative Adversarial Networks (GANs)}

    \subsection{Grundlagen von Generative Adversarial Networks (GANs)}
    
    Generative Adversarial Networks (GANs) sind ein leistungsstarkes Framework für das Training von Deep Learning-Modellen zur Generierung von Daten. 
    GANs bestehen aus zwei miteinander konkurrierenden neuronalen Netzwerken, einem Generator und einem Diskriminator. Der Generator erzeugt neue Daten, während der Diskriminator versucht, zwischen den vom Generator erzeugten Daten und den echten Daten zu unterscheiden. Im Laufe des Trainings passt sich der Generator kontinuierlich an und verbessert seine Fähigkeit, realistische Daten zu generieren, während der Diskriminator gleichzeitig verbessert wird, um zwischen den generierten und echten Daten zu unterscheiden.
    %TODO GRAFIK! LUKAS! GRAFIK!
    
    \subsection{Architekturen von GANs}
    
    Es gibt verschiedene Architekturen von GANs, die für verschiedene Arten von Anwendungen geeignet sind. 
    in Beispiel ist das Deep Convolutional GAN (DCGAN), das speziell für die Generierung von Bildern entwickelt wurde. %TODO Bild? ALso so architekkturbild oder so idk
    DCGAN nutzt Convolutional Neural Networks (CNNs) und Transposed Convolutional Neural Networks, um Bilder zu generieren, die visuell realistisch aussehen und strukturell konsistent sind.
    Ein weiteres Beispiel ist das CycleGAN, das für die Bildübersetzung zwischen verschiedenen Domänen verwendet werden kann. 
    CycleGAN nutzt einen Generator und einen Diskriminator sowie zusätzliche Cycle-Verlustfunktionen, um die Transformationen zwischen den Bildern in verschiedenen Domänen zu erlernen.%todo quellen vergessen
    
    \subsection{Anwendungen von GANs}
    
    GANs finden Anwendungen in verschiedenen Bereichen wie der Bildgenerierung, Style Transfer, der Verbesserung von Bildern und der Videoanalyse. 
    Zum Beispiel können GANs verwendet werden, um realistisch aussehende Bilder von Gesichtern, Landschaften oder anderen Objekten zu generieren. %TODO Beispielbild
    Style Transfer ermöglicht es, das visuelle Erscheinungsbild von Bildern zu verändern, indem der Stil von einem Bild auf ein anderes übertragen wird. %TODO Beispielbild
    GANs können auch verwendet werden, um Bilder mit höherer Auflösung oder besserer Qualität zu generieren, indem sie niedrig aufgelöste Bilder als Eingabe verwenden. 
    In der Videoanalyse können GANs verwendet werden, um Videosequenzen zu generieren oder zu verbessern.%TODO Quelle

    \subsection{Training von GANs und Evaluierung von generierten Ergebnissen}

    Das Training von GANs ist eine Herausforderung, da es sich um ein adversariales Lernverfahren handelt. 
    Das bedeutet, dass es zwei Netze gibt, die sich gegenseitig trainieren und verbessern. %TODO Grafik & Quellen
    Das generative Netzwerk versucht, Bilder zu erzeugen, die von einem diskriminierenden Netzwerk nicht von echten Bildern unterschieden werden können. 
    Das diskriminierende Netzwerk wird trainiert, um echte Bilder von den vom generativen Netzwerk generierten Bildern zu unterscheiden.
    Das Training von GANs erfolgt durch die Minimierung einer Verlustfunktion, die als GAN-Verlust bezeichnet wird. 
    Der GAN-Verlust besteh taus zwei Komponenten: dem Verlust des generativen Netzes und dem Verlust des diskriminierenden Netzes. Der Verlust des generativen Netzes wird minimiert, wenn dasd Netzwerk Bilder erzeugt, die vom diskriminierenden Netzwerk nicht als gefälscht erkannt werden. 
    Der Verlust des diskriminierenden Netzes wird minimiert, wenn das Netzwerk in der Lage ist, besonders zuverlässig und schneöö zwischen echten und generierten Bildern zu unterscheiden.
    Die Evaluierung von generierten Ergebnissen ist eine wichtige Aufgabe bei der Arbeit mit GANs. 
    Es gibt verschiedene Methoden zur Bewertung von GANs, wie beispielsweise die visuelle Bewertung, die qualitative Bewertung und die quantitative Bewertung. Die visuelle Bewertung beinhaltet das Betrachten der generierten Bilder, um zu beurteilen, ob sie realistisch aussehen oder nicht. Die qualitative Bewertung beinhaltet die Verwendung von Bewertungsskalen, um die Qualität der generierten Bilder zu bewerten. Die quantitative Bewertung beinhaltet die Verwendung von Metriken wie der Inception Score oder der Frechet Inception Distance, um die Qualität der generierten Bilder zu bewerten.

    \subsection{Ethische und soziale Implikationen von GANs}
    
    Obwohl GANs eine vielversprechende Technologie sind, gibt es auch ethische und soziale Implikationen, die berücksichtigt werden müssen. 
    Ein Problem bei der Verwendung von GANs ist, dass sie zur Erzeugung gefälschter Bilder oder Videos verwensdet werden können. 
    Dies kann zu Fälschungen und Manipulationen führen, die negative Auswirkungen auf die Gesellschaft haben können.
    Auch Rufschädigung kann durch GANs errleichtert werden.
    Ein weiteres Problem bei der Verwendung von GANs ist, dass sie möglicherweise nicht fair sind. 
    GANs können aufgrund ihrer Lernmethode unbewusste Vorurteile aufnehmen und in ihren generierten Ergebnissen widerspiegeln. 
    Dies kann zu diskriminierenden Ergebnissen führen, die unfaire Entscheidungen unterstützen.%TODO verweis auf gescheiterte Google AIs auf Twitter einbauen
    Es ist wichtig, dass bei der Verwendung von GANs Ethik und soziale Verantwortung berücksichtigt werden. 
    Es sollten Maßnahmen ergriffen werden, um sicherzustellen, dass GANs fair und ethisch korrekt arbeiten. 
    Zum Beispiel können spezielle Algorithmen entwickelt werden, um unbewusste Vorurteile zu minimieren. 
    Weiterhin können Regierungsbehörden und andere Organisationen Maßnahmen ergreifen, um den Missbrauch von GANs zu verhindern. 
    Das finden eines Kompromiss aus Forschung und politischer Einschränkung übersteigt jedoch den Rahmen dieser Arbeit. 
    
\section{Multiscale-Skalierung}

- Grundlagen von Multiscale-Skalierung
- Methoden zur Multiskalenanalyse (z.B. Wavelets, Pyramiden)
- Anwendungen von Multiscale-Skalierung (z.B. Texturanalyse, Bildkompression)
- Multiscale-Skalierung in Verbindung mit Deep Learning
- Limitationen und zukünftige Forschungsrichtungen von Multiscale-Skalierung

\section{Vor- und Nachteile der fortgeschrittenen Methoden}

- Vergleich der verschiedenen fortgeschrittenen Skalierungsmethoden
- Vorteile von fortgeschrittenen Methoden im Vergleich zu traditionellen Methoden
- Herausforderungen bei der Anwendung fortgeschrittener Methoden
- Auswirkungen von fortgeschrittenen Methoden auf die Leistung und Effizienz von Systemen
- Zukunftsaussichten für fortgeschrittene Skalierungsmethoden.

\newpage



\chapter{Evaluation von Skalierungsmethoden}

\section{Qualitätsmetriken von Skalierungsmethoden}

    Die Evaluation von Skalierungsmethoden in der Bildverarbeitung bedarf einer facettenreichen Palette an Qualitätsmetriken, welche eine präzise Analyse und Vergleichbarkeit unterschiedlicher Methoden ermöglichen. 
    Im Rahmen dieser Arbeit werden ausgewählte sowie zentrale Qualitätsmetriken für die Bewertung von Skalierungsmethoden erörtert und diskutiert.

    \subsection{\ac{PSNR}}
        Das Peak Signal-to-Noise Ratio \ac{PSNR} ist eine der am häufigsten verwendeten Metriken zur Bewertung der Qualität von Bildern. 
        Es misst die Qualität einer Bildrekonstruktion, indem es den Unterschied zwischen einem Originalbild und einem rekonstruierten Bild berechnet und diesen Unterschied durch das maximale Signal (Peak) und das Rauschen (Noise) im Originalbild teilt. 
        Je höher der PSNR-Wert, desto geringer ist der Unterschied zwischen Original- und Rekonstruktionsbildern und desto besser ist die Qualität der Rekonstruktion.
        \footfullcite{wang2004image}
        
        \subsubsection{Definition und Berechnung von \ac{PSNR}}
            Die Definition von PSNR lautet wie folgt:
            \begin{equation}
            PSNR = 20 \log_{10} \left( \frac{MAX_I}{\sqrt{MSE}} \right) 
            %TODO prüf mal ob die Formel stimmt. Bin verzweifelt und hab die irwo aus stackoverflow geholt
            \end{equation}

            Dabei ist $MAX_I$ der maximal mögliche Pixelwert des Bildes.
            Bei der Darstellung von Pixeln mit 8 Bits pro Abtastwert ist dieser Wert 255.
            Allgemeiner ausgedrückt: Wenn die Samples mit linearer PCM mit B Bits pro Sample dargestellt werden, ist MAXI $2^B - 1$\footfullcite{wiki:Peak_signal-to-noise_ratio}.
            
            % wobei Peak der höchste mögliche Wert des Signals ist, der meist auf 255 bei 8-Bit-Graustufenbildern oder 65535 bei 16-Bit-Graustufenbildern gesetzt wird. %nur so semi verstanden aber die Quellen haben da ünbereingestimmt (stackoverflow help me)

            \begin{equation}
                MSE = \frac{1}{n*m} \sum_{i=0}^{n-1} \sum_{i=0}^{m-1} (I(i,j) - K(i,j))^2
            \end{equation}
            
            Der Mean Squared Error \ac{MSE} ist definiert als der Durchschnitt der quadrierten Unterschiede zwischen jedem Pixel des Originalbildes und dem rekonstruierten Bild.
            Je höher der PSNR-Wert, desto geringer ist der \ac{MSE} und desto besser ist die Qualität der Rekonstruktion.
            
        %TODO Quelle für Formel (wWill kein Stackoverflow link angeben) 
        \subsubsection{Anwendung von \ac{PSNR} bei der Bewertung von Bildqualität}
            Obwohl \ac{PSNR} eine weit verbreitete Methode zur Bewertung der Qualität von Bildern ist, hat sie auch ihre Limitationen\footfullcite{sheikh2006image}.
            Zum einen berücksichtigt sie nur die Fehler zwischen Original- und Rekonstruktionsbildern und vernachlässigt andere Faktoren wie Bildverzerrungen, die durch Komprimierung oder Filterung entstehen können.
            Zum andern ist der \ac{PSNR}-Wert nicht sensitiv für menschlich wahrgenommene Verzerrungen\footfullcite{mittal2012no}, wie z.B. Farbverschiebungen oder Artefakte.
            Trotz dieser Limitationen bleibt \ac{PSNR} eine wichtige Metrik in der Bildverarbeitung und wird oft in der Praxis verwendet, um die Qualität von Bildrekonstruktionen zu bewerten und zu vergleichen.
            \footfullcite{korhonen2012peak}
    \subsection{Structural Similarity Index Measure (SSIM)}

        Der Structural Similarity Index \ac{SSIM} ist eine Metrik zur Bewertung der strukturellen Ähnlichkeit zwischen einem Originalbild und einem rekonstruierten Bild. 
        Im Gegensatz zum PSNR berücksichtigt der \ac{SSIM} nicht nur die Pixelwerte, sondern auch die Struktur und Textur des Bildes. 
        Der \ac{SSIM} berechnet die Ähnlichkeit zwischen den beiden Bildern anhand von drei Faktoren: Helligkeit, Kontrast und Struktur. Die Formel zur Berechnung von \ac{SSIM} ist wie folgt:
        
        \begin{equation}
            \text{SSIM}(x, y) = \frac{{(2\mu_x\mu_y + c_1)(2\sigma_{xy} + c_2)}}{{(\mu_x^2 + \mu_y^2 + c_1)(\sigma_x^2 + \sigma_y^2 + c_2)}}
        \end{equation}

        In dieser Formel repräsentieren \(x\) und \(y\) zwei verglichene Bilder.
        Die \ac{SSIM} misst die strukturelle Ähnlichkeit zwischen diesen Bildern.
        \(\mu_x\) und \(\mu_y\) sind die Mittelwerte von \(x\) und \(y\), während \(\sigma_x^2\) und \(\sigma_y^2\) die Varianzen von \(x\) und \(y\) darstellen.
        \(\sigma_{xy}\) repräsentiert die Kovarianz zwischen \(x\) und \(y\).
        Die Konstanten \(c_1\) und \(c_2\) sind kleine Werte, die zur Stabilisierung der Division hinzugefügt werden
        \footfullcite{chen2011fast}\footfullcite{wiki:Structural_similarity}.

        
        \subsubsection{Anwendung von SSIM bei der Bewertung von Bildqualität}
        
            \ac{SSIM} wird häufig verwendet, um die Qualität von Bildrekonstruktionen zu bewerten. Es hat sich gezeigt, dass \ac{SSIM} besser als PSNR die wahrgenommene Bildqualität wiederspiegelt. 
            Dies liegt daran, dass \ac{SSIM} die strukturelle Ähnlichkeit zwischen den beiden Bildern berücksichtigt, während \ac{PSNR} nur die Differenz der Pixelweite betrachtet.

        \subsubsection{Vorteile von SSIM im Vergleich zu PSNR}
        
            Im Vergleich zum \ac{PSNR} hat \ac{SSIM} mehrere Vorteile. Zum einen berücksichtigt es die strukturelle Ähnlichkeit zwischen den beiden Bildern, was zu einer besseren Bewertung der wahrgenommenen Bildqualität führt. 
            Zum anderen ist \ac{SSIM} in der Lage, Verzerrungen zu erkennen, die durch Kompression oder andere Arten von Bildverarbeitung verursacht werden, während \ac{PSNR} dies nicht tut.
        \footfullcite{5596999}
        
        \subsubsection{Limitationen von SSIM}
        
            Obwohl \ac{SSIM} eine bessere Metrik zur Bewertung der Bildqualität als \ac{PSNR} darstellt, hat es auch seine Limitationen. 
            \ac{SSIM} ist anfällig für Helligkeits- und Kontrastunterschiede zwischen den beiden Bildern und kann bei der Bewertung von stark komprimierten Bildern ungenau sein. 
            Auch bei der Anwendung auf Bilder mit unterschiedlichen Strukturen kann die \ac{SSIM} eine ungenaue Bewertung liefern.
        \footfullcite{1284395}
    \subsection{Mean Opinion Score (MOS)}
        Der Mean Opinion Score \ac{MOS} ist eine subjektive Metrik, die die wahrgenommene Qualität einer Bildrekonstruktion misst.
        MOS basiert auf der Bewertung durch menschliche Beobachter, die gebeten werden, die Qualität von Original- und Rekonstruktionsbildern auf einer Skala von 1 bis 5 oder 1 bis 10 zu bewerten.
        MOS ist eine wichtige Metrik, da sie die subjektive Wahrnehmung der Qualität eines Bildes durch den Betrachter berücksichtigt.
        \footfullcite{sheikh2006image}
        
    \subsection{Peak Signal-to-Noise Ratio der Y-Komponente (PSNR-Y)}
        Die Y-Komponente im YCbCr-Farbraum enthält die Helligkeitsinformationen des Bildes. 
        Das Peak Signal-to-Noise Ratio der Y-Komponente \ac{PSNR-Y} ist eine spezielle Version des \ac{PSNR}, die nur die Helligkeitsinformationen des Originalbildes und des rekonstruierten Bildes berücksichtigt. 
        \ac{PSNR}-Y ist eine wichtige Metrik zur Bewertung der Qualität von Skalierungsmethoden für Graustufen- oder Schwarz-Weiß-Bilder.        
        Diese Qualitätsmetriken sind wichtigeWerkzeuge für diew Bewertung und den Vergleich von Skalierungsmelthoden in der Bildverarbeitung.
        Es ist jedoch wichtig zu beachten, dass keine einzelne Metrik alle Aspekte der Bildqualität abdeckt. Eine umfassende Bewertung sollte mehrere Metriken kombinieren und auch die subjektive Wahrnehmung 
        \footfullcite{huang2010new}
        
    \subsection{Computational Speed}
    
        Die Wahl einer Skalierungsmethode hängt nicht nur von der Bildqualität, sondern auch von der benötigten Rechenleistung ab. 
        Die Computational Speed ist daher ein wichtiger Faktor bei der Auswahl einer Skalierungsmethode.
        Es gibt verschiedene Methoden zur Messung von Computational Speed, wie z.B. die Messung der benötigten Zeit, um ein Bild zu skalieren oder die Berechnung von \ac{FLOPS}. 
        Eine genauere Messung kann durch die Verwendung von Benchmarks erreicht werden, die es ermöglichen, verschiedene Skalierungsmethoden auf derselben Hardware zu vergleichen.
        Jedoch muss bei der Messung der Computational Speed berücksichtigt werden, dass die Leistung des verwendeten Computers die Ergebnisse beeinflussen kann. 
        Ein schnellerer Computer kann eine Methode schneller ausführen als ein langsamerer Computer. 
        Daher ist es wichtig, die Messungen auf einem vergleichbaren Computer durchzuführen und die Ergebnisse zu normalisieren.
        Ein wichtiger Trade-off besteht zwischen der Bildqualität und der Computational Speed. 
        Eine Methode, die eine höhere Bildqualität liefert, benötigt in der Regel mehr Rechenleistung und ist daher langsamer als eine Methode mit niedrigerer Bildqualität. 
        Es ist daher wichtig, die gewünschte Bildqualität und die verfügbare Rechenleistung abzuwägen und eine Methode zu wählen, die den Anforderungen am besten entspricht. 
        Es können auch Techniken wie progressiver Skalierung verwendet werden, um eine gute Bildqualität mit einer angemessenen Geschwindigkeit zu erreichen.
        \footfullcite{xu2017efficient,choi2019lightweight}
        
    \subsection{Root-Mean-Square Error (RMSE)}
    
        Der Root-Mean-Square Error \ac{RMSE} ist eine Metrik zur Bewertung der Qualität von Bildrekonstruktionen. 
        Ein Nachteil von RMSE ist, dass er nur den Fehler zwischen Pixelwerten betrachtet und keine Berücksichtigung von strukturellen Unterschieden im Bild nimmt. 
        Daher wird RMSE oft in Kombination mit anderen Metriken wie \ac{PSNR} und \ac{SSIM} verwendet, um ein umfassenderes Bild der Bildqualität zu erhalten.
        Abschließend ist zu sagen, dass beider Bewertung von Bildqualität die Wahl der Metrik von der spezifischen Anwendung abhängt. 
        Während \ac{PSNR} und \ac{SSIM} für die Bewertung von Bildern in vielen Anwendungen ausreichend sein können, kann in anderen Fällen die subjektive Wahrnehmung des menschlichen Betrachters durch MOS eine bessere Metrik sein. 
        Darüber hinaus ist bei der Wahl einer Skalierungsmethode auch die Messung von Computational Speed und das Abwägen von Trade-offs zwischen Bildqualität und Geschwindigkeit von entscheidender Bedeutung.
        \footfullcite{morad1996role}
        
\section{Kriterien zur Auswahl der Skalierungsmethode} %TODO Hier können wir viele Bilder einbauen um die Vergleichsgrafiken besser zu erklären (@MARC)

    \subsection{Bildvergleich und visuelle Bewertung}
        
        Die Evaluation von Skalierungsmethoden in der Bildverarbeitung erfordert eine umfassende und präzise Analyse verschiedener Kriterien, um die optimale Methode auszuwählen. 
        Ein wesentlicher Aspekt bei dieser Bewertung ist der Bildvergleich und die visuelle Bewertung, die auf der subjektiven Wahrnehmung von menschlichen Beobachtern beruhen.
        Die visuelle Bewertung von Bildern durch menschliche Beobachter ermöglicht eine direkte und subjektive Beurteilung der Bildqualität basierend auf wahrgenommenen visuellen Eigenschaften. 
        Verschiedene Methoden des Bildvergleichs werden eingesetzt, wie beispielsweise der Side-by-Side-Vergleich, bei dem zwei Bilder direkt miteinander verglichen werden, oder der Triple-Stimulus-Test, bei dem ein Originalbild mit zwei rekonstruierten Bildern verglichen wird.
        Diese Methoden des Bildvergleichs und der visuellen Bewertung werden auch bei der Evaluierung von Skalierungsmethoden angewendet. 
        Durch den Vergleich der rekonstruierten Bilder mit dem Originalbild können verschiedene Skalierungsmethoden anhand ihrer visuellen Qualität beurteilt und verglichen werden. 
        dies ermöglicht eine umfangreiche Analyse und eine direkte Gegenüberstellung der unterschiedlichen Methoden.
        Es ist jedoch von entscheidender Bedeutung, die Limitationen und Vorbehalte im Zusammenhang mit der visuellen Bewertung von Bildern zu berücksichtigen. 
        Die subjektive Wahrnehmung kann von Person zu Person variieren und wird auch von anderen Faktoren wie Beleuchtung, Betrachtungsabstand und individuellen Präferenzen beeinflusst. 
        Darüber hinaus können komplexe visuelle Verzerrungen nicht immer präzise durch die visuelle Bewertung erfasst werden. 
        Daher sollten bei der Bewertung von Skalierungsmethoden auch objektive Metriken und weitere Evaluierungskriterien einbezogen werden.
        Insgesamt spielt der Bildvergleich und die visuelle Bewertung eine bedeutende Rolle bei der Evaluierung von Skalierungsmethoden in der Bildverarbeitung.
        Durch die Kombination der subjektiven visuellen Bewertung mit objektiven Metriken können fundierte Entscheidungen getroffen werden, um die optimale Methode gemäß den spezifischen Anforderungen der Anwendung auszuwählen. 
        Die sorgfältige Berücksichtigung der genannten Limitationen trägt dazu bei, zuverlässige und aussagekräftige Ergebnisse zu erzielen und die Qualität von Bildrekonstruktionen bestmöglich zu bewerten.
        \footfullcite{ye2012no,wang2004image,eckert1998perceptual}
    
    \subsection{Effektivität und Effizienz}

        Die Bewertung von Skalierungsmethoden in der Bildverarbeitung erfordert eine umfassende Betrachtung sowohl der Effektivität, also der Bildqualität, als auch der Effizienz, insbesondere der Computational Speed. 
        Es besteht ein inhärenter Trade-off zwischen der erzielten Bildqualität und der benötigten Rechenleistung, der bei der Auswahl der optimalen Skalierungsmethode berücksichtigt werden muss.
        Die Effektivität einer Skalierungsmethode bezieht sich auf die erzielte Bildqualität der rekonstruierten Bilder. 
        Eine Methode, die eine höhere Bildqualität liefert, wird in der Regel auch mehr Rechenleistung benötigen und somit langsamer sein als eine Methode mit geringerer Bildqualität. 
        Daher ist es von großer Bedeutung, die gewünschte Bildqualität und die verfügbare Rechenleistung sorgfältig abzuwägen, um die beste Skalierungsmethode zu wählen.


        \begin{table}[!h]
        \centering
        \begin{tabular}{ccccccc}
            \textbf{Model} & \textbf{PLCC} & \textbf{MAE} & \textbf{RMS} & \textbf{SRCC} & \textbf{KRCC} & \textbf{Computation time} \\ \hline
            PSNR           & 0.9062        & 7.4351       & 9.8191       & 0.8804        & 0.6886        & 1                                     \\
            SSIM           & 0.9253        & 6.9203       & 8.8069       & 0.9014        & 0.7246        & 22.65                                 \\
            MS-SSIM        & 0.8945        & 8.1969       & 10.384       & 0.8619        & 0.6605        & 48.49                                 \\
            SSIMplus       & 0.9732        & 4.3192       & 5.3451       & 0.9349        & 0.7888        & 7.83                                  
        \end{tabular}
        \caption{Leistungsvergleich zwischen PSNR, SSIM, MS-SSIM, VQM, PQR-Tek, DMOS-Tek, JND-VC,
DMOS-VC und SSIMplus einschließlich aller Geräte\footfullcite{Rehman_Zeng_Wang_2015}.}
        \label{tab:Leistungsvergleich}
        \end{table}


        
        Die Evaluierung der Trade-offs zwischen Effektivität und Effizienz erfordert eine gründliche Analyse der Leistungsmerkmale verschiedener Skalierungsmethoden. 
        Hierbei können Kosten-Nutzen-Analysen hilfreich sein, um die Auswirkungen der gewählten Methode auf die Bildqualität und die erforderliche Rechenleistung abzuschätzen. 
        Indem die Kosten in Bezug auf die erzielte Bildqualität bewertet werden, kann die optimale Methode entsprechend den spezifischen Anforderungen des Anwendungsbereichs ausgewählt werden.
        Die Relevanz von Effektivität und Effizienz variiert je nach Anwendungsbereich der Bildverarbeitung. 
        In einigen Fällen, wie beispielsweise medizinischen Bildgebungsverfahren oder hochauflösenden Videoanwendungen, ist eine hohe Bildqualität von größter Bedeutung, selbst wenn dies mit einem höheren Rechenaufwand einhergeht. 
        In anderen Szenarien, wie in Echtzeit-Anwendungen oder mobilen Geräten, kann die Effizienz und die schnelle Verarbeitung der Bilder priorisiert werden, auch wenn dies zu einer gewissen Einbuße der Bildqualität führt.
        Insgesamt ist es von entscheidender Bedeutung, sowohl die Effektivität als auch die Effizienz bei der Wahl der besten Skalierungsmethode zu berücksichtigen. 
        Die richtige Balance zwischen Bildqualität und erforderlicher Rechenleistung zu finden, ermöglicht die optimale Nutzung von Ressourcen und die Erfüllung der spezifischen Anforderungen des jeweiligen Anwendungsbereichs der Bildverarbeitung.
        \footfullcite{zhang2019efficiency}\footfullcite{gu2019comprehensive}\footfullcite{hu2020efficient}

    \subsection{Zukunftsaussichten und Herausforderungen}
    
        Die kontinuierliche Entwicklung von Technologien stellt neue Herausforderungen bei der Evaluierung von Skalierungsmethoden in der Bildverarbeitung dar.
        Fortschritte in der Hardware, wie leistungsstärkere Prozessoren und spezialisierte Beschleuniger, können die Rechenleistung erhöhen und somit neue Möglichkeiten für komplexe Skalierungsalgorithmen eröffnen. Gleichzeitig führen neue Technologien wie Virtual Reality, Augmented Reality und 8K-Bildschirme zu höheren Anforderungen an die Bildqualität und erfordern präzisere Bewertungsmethoden.
        Spezifische Anwendungsbereiche, wie die medizinische Bildgebung oder die Videokompression, stellen weitere Herausforderungen bei der Evaluierung von Skalierungsmethoden dar. 
        In der medizinischen Bildgebung ist die genaue Beurteilung der Bildqualität von entscheidender Bedeutung für die richtige Diagnosestellung und Behandlungsplanung. 
        Hier müssen spezielle Metriken und Evaluierungsverfahren entwickelt werden, um den Anforderungen dieses Bereichs gerecht zu werden. 
        Ebenso erfordert die Videokompression eine sorgfältige Evaluierung der Skalierungsmethoden, um eine ausreichende Qualität der komprimierten Videos zu gewährleisten und gleichzeitig die Effizienz der Kompression zu maximieren.
        Ein vielversprechendes Potenzial für die zukünftige Evaluierung von Skalierungsmethoden liegt in der Anwendung von KI-basierten Evaluierungsmethoden. 
        Durch den Einsatz von maschinellem Lernen und neuronalen Netzwerken können automatisierte Bewertungsalgorithmen entwickelt werden, die die menschliche Wahrnehmung von Bildqualität besser simulieren. 
        Diese KI-basierten Ansätze können eine schnellere und objektivere Evaluierung ermöglichen und den Entwicklungsprozess von Skalierungsmethoden beschleunigen.
        Bei der Bewertung von Skalierungsmethoden sollten jedoch auch ethische und soziale Implikationen berücksichtigt werden. 
        Die Entwicklung von immer leistungsfähigeren Skalierungsmethoden kann auch potenzielle Risiken mit sich bringen, wie zum Beispiel die Manipulation von Bildern oder die Schaffung von gefälschten Inhalten. 
        Es ist daher wichtig, entsprechende Schutzmechanismen zu entwickeln, um den Missbrauch solcher Technologien zu verhindern und die Privatsphäre und Integrität von Personen zu wahren.
        Zusammenfassend lassen sich die Zukunftsaussichten für die Evaluierung von Skalierungsmethoden als vielversprechend betrachten. 
        Durch die Berücksichtigung technologischer Entwicklungen, die Bewältigung spezifischer Herausforderungen in verschiedenen Anwendungsbereichen, die Nutzung von KI-basierten Evaluierungsmethoden und die Beachtung ethischer und sozialer Implikationen können wir die Qualität und Effizienz von Skalierungsmethoden weiter verbessern und gleichzeitig sicherstellen, dass sie verantwortungsvoll und zum Wohl der Gesellschaft eingesetzt werden.
        \footfullcite{zhang2015learning}\footfullcite{blau2018perception}\footfullcite{liu2016ssd,zhang2018split}
\newpage

\chapter{Zusammenfassung und Ausblick}

\section{Auswertung der Ergebnisse}
\section{Diskussion offener Fragen und zukünftiger Forschungsbedarf}
\chapter{Einführung in Deep Learning}

    Deep Learning ist eine Unterkategorie des maschinellen Lernens und beschäftigt sich mit der Entwicklung und Anwendung von neuronalen Netzwerken mit mehreren Schichten, um komplexe Probleme zu lösen. Es verwendet künstliche Intelligenz, um aus großen Datenmengen Muster zu erkennen und Entscheidungen zu treffen. 
    In der heutigen Welt hat Deep Learning Anwendungen in Bereichen wie Spracherkennung, Bilderkennung, Robotik, Medizin und Autonomen Fahrzeugen gefunden. 
    Die Geschichte des Deep Learning reicht zurück bis in die 1940er Jahre, aber erst in den 2000er Jahren wurden die Algorithmen und Computerleistung ausreichend entwickelt, um Deep Learning erfolgreich anzuwenden. 
    Im Jahr 2012 gewann ein Deep Learning-Modell namens AlexNet den ImageNet-Wettbewerb, was als Durchbruch für die Anwendung von Deep Learning in Computer Vision gilt.
    Traditionelles maschinelles Lernen verwendet in der Regel flache neuronale Netzwerke, die nur eine Schicht haben, um Daten zu analysieren. 
    Im Gegensatz dazu verwenden Deep-Learning-Modelle mehrere Schichten, um komplexere Merkmale der Daten zu identifizieren. 
    Dies ermöglicht eine höhere Genauigkeit und Effektivität bei der Lösung komplexer Probleme.
    Beispiele für reale Anwendungen von Deep Learning sind unter anderem die Bilderkennung in sozialen Medien, die Spracherkennung in Smart Speakern wie Amazon Echo und Google Home, sowie die automatisierte Diagnose von medizinischen Bildern wie CT-Scans und MRT-Bildern.

\section{Motivation hinter der Studie von CNNs und deren Training}
    
    Traditionelle maschinelle Lernalgorithmen sind nicht effektiv bei der Verarbeitung komplexer Datentypen wie Bildern. Hier kommen Convolutional Neural Networks (CNNs) ins Spiel. 
    CNNs sind eine spezielle Art von neuronalem Netzwerk, die speziell für die Bilderkennung und Computer Vision entwickelt wurden. 
    CNNs arbeiten durch die Verwendung von Filtern, die über das Eingabebild geschoben werden, um Merkmale wie Kanten, Farben und Texturen zu extrahieren. 
    Diese Merkmale werden dann von Schichten von Neuronen verwendet, um die Merkmale zu kombinieren und eine Vorhersage zu treffen. 
    Eine der größten Herausforderungen beim Training von CNNs ist das Overfitting, bei dem das Modell zu stark auf die Trainingsdaten optimiert wird und dadurch bei neuen Daten schlecht abschneidet. 
    Eine weitere Herausforderung ist das Problem der verschwindenden Gradienten, bei dem die Gradienten, die zur Anpassung der Gewichte verwendet werden, während des Trainings immer kleiner werden und das Modell nicht mehr lernen kann. 
    Um diese Probleme zu lösen, werden Optimierungsalgorithmen wie der stochastische Gradientenabstieg verwendet, um die Gewichte des Modells anzupassen und das Modell an die Daten anzupassen. 
    Es gibt auch Techniken wie Dropout und Data Augmentation, die dazu beitragen können, Overfitting zu reduzieren. 
    Es ist wichtig, CNNs zu verstehen und effektiv zu trainieren, um die volle Leistungsfähigkeit von Deep Learning in der Bilderkennung und Computer Vision zu nutzen. 
    
\section{Struktur de Arbeit}

    Diese Studienarbeit ist in drei Hauptabschnitte unterteilt.
    Im ersten Abschnitt wird eine Einführung in Deep Learning gegeben, einschließlich der Bedeutung von Deep Learning in der heutigen Welt, der Geschichte und Entwicklung des Deep Learning, der Unterschiede zwischen Deep Learning und traditionellem maschinellem Lernen und Beispielen für reale Anwendungen von Deep Learning.
    
    Im zweiten Abschnitt wird die Motivation hinter der Studie von CNNs und deren Training erläutert. Dabei werden die Einschränkungen von traditionellen maschinellen Lernalgorithmen bei der Verarbeitung komplexer Datentypen wie Bildern diskutiert und die Bedeutung von CNNs bei der Bilderkennung und Computer Vision erklärt. Weiterhin werden die Herausforderungen beim Training von CNNs wie Overfitting und verschwindende Gradienten sowie die Rolle von Optimierungsalgorithmen wie dem stochastischen Gradientenabstieg beim Training von CNNs beschrieben.
    
    Im dritten Abschnitt werden die Ergebnisse und Beiträge dieses Papiers skizziert. Dies beinhaltet eine Zusammenfassung der wichtigsten Erkenntnisse und Empfehlungen für das effektive Training von CNNs.
\chapter{Grundlagen von Convolutional Neural Networks}

\section{Wie lernen Maschinen?}

    In diesem Abschnitt werden die fundamentalen Konzepte des maschinellen Lernens und des Deep Learnings erläutert. 
    Das maschinelle Lernen bezieht sich auf einen Prozess, bei dem maschinelle Systeme mithilfe von statistischen Modellen und Algorithmen automatisch aus Erfahrung lernen, ohne dass eine explizite Programmierung erforderlich ist. 
    Das Deep Learning stellt einen spezifischen Teilbereich des maschinellen Lernens dar und konzentriert sich auf die Anwendung von neuronalen Netzwerken mit zahlreichen Schichten, um komplexe abstrakte Repräsentationen zu erlernen.
    
\subsection{Definition von maschinellem Lernen}
    
    Das maschinelle Lernen ist ein Prozess, bei dem computergestützte Systeme mithilfe von statistischen Methoden und Algorithmen aus Erfahrung lernen, ohne dass eine explizite Programmierung erforderlich ist. 
    Im Gegensatz zur traditionellen Programmierung, bei der eine festgelegte Abfolge von Anweisungen gegeben wird, basiert das maschinelle Lernen auf der Analyse und Verarbeitung großer Mengen an Daten. 
    Durch diesen iterativen Lernprozess können Maschinen Muster, Zusammenhänge und Regeln erkennen, um Vorhersagen zu treffen oder komplexe Aufgaben zu automatisieren.
    Der Kern des maschinellen Lernens liegt in der Verwendung von Algorithmen und statistischen Modellen, um aus den vorhandenen Daten zu lernen. 
    Diese Modelle werden trainiert, indem sie mit Eingabedaten gefüttert werden und ihre internen Parameter so angepasst werden, dass sie die gewünschten Ausgabewerte erzeugen. 
    Der Trainingsprozess erfolgt durch die Optimierung von Modellparametern unter Verwendung von mathematischen Optimierungsmethoden wie dem Gradientenabstieg.
    Durch wiederholtes Training und Feinabstimmung der Modelle können sie kontinuierlich verbessert werden, um präzisere Vorhersagen zu generieren oder überlegene Leistungen bei bestimmten Aufgaben zu erzielen.

\subsection{Deep Learning}

    Deep Learning ist ein Teilbereich des maschinellen Lernens, der sich auf die Verwendung neuronaler Netzwerke mit vielen Schichten konzentriert. 
    Neuronale Netzwerke sind mathematische Modelle, die biologische Neuronen simulieren und miteinander verbundene Schichten von Neuronen enthalten. 
    Jede Schicht nimmt Eingaben von der vorherigen Schicht entgegen und gibt Ausgaben an die nächste Schicht weiter.
    Der Hauptvorteil von Deep Learning liegt in der Fähigkeit, automatisch Merkmale aus den Daten zu extrahieren, anstatt dass diese manuell von einem Experten definiert werden müssen. 
    Durch die Kombination vieler Schichten können neuronale Netzwerke komplexe Muster und abstrakte Darstellungen erfassen, wodurch sie in der Lage sind, hochdimensionale Daten effektiv zu verarbeiten.
    Das Training von Deep-Learning-Modellen erfolgt in der Regel mit Hilfe von großen Datensätzen und leistungsstarken GPUs, um die erforderlichen Berechnungen effizient durchführen zu können. 
    Durch das iterative Training der Netzwerke werden die Gewichtungen und Parameter der einzelnen Neuronen angepasst, um die Fähigkeit des Modells zu verbessern, genaue Vorhersagen zu treffen oder komplexe Aufgaben zu lösen.
    In den folgenden Abschnitten werden wir uns mit den grundlegenden Komponenten und Funktionen von Convolutional Neural Networks (CNNs) befassen, die eine spezielle Art von neuronalen Netzwerken sind, die insbesondere in der Bildverarbeitung weit verbreitet sind.
    \footfullcite{lecun2015deep,schmidhuber2015deep}

\subsection{Grundlagen von Convolutional Neural Networks}

    Convolutional Neural Networks (CNNs) sind eine spezielle Art von neuronalen Netzwerken, die in der Lage sind, Muster und Merkmale in Bildern effektiv zu erkennen. 
    Sie zeichnen sich durch ihre Fähigkeit aus, lokale Verbindungen und Gewichtungen zwischen den Neuronen herzustellen und so räumliche Informationen in den Daten zu berücksichtigen.
    Die grundlegende Architektur eines CNNs besteht aus mehreren Schichten, darunter Convolutional Layers, Pooling Layers und Fully Connected Layers. 
    In den Convolutional Layers werden Filter verwendet, um lokale Muster und Merkmale zu extrahieren, indem sie über das Eingabebild verschoben werden und die Faltungsoperation durchgeführt wird. 
    Diese Filter können beispielsweise Kanten, Texturen oder spezifische Formen erfassen.
    Nach den Convolutional Layers folgen Pooling Layers, die dazu dienen, die räumliche Dimension der Merkmale zu reduzieren und die wichtigsten Informationen beizubehalten. 
    Typische Pooling-Operationen umfassen das Max-Pooling, bei dem der maximale Wert in einem Bereich ausgewählt wird, und das Average-Pooling, bei dem der Durchschnittswert berechnet wird.
    Die Ausgabe der Pooling Layers wird dann an die Fully Connected Layers weitergeleitet, die eine traditionelle neuronale Netzwerkstruktur aufweisen. Diese Schichten dienen dazu, die erfassten Merkmale zu klassifizieren oder eine Vorhersage zu treffen, indem sie die Gewichtungen der Neuronen anpassen und die Aktivierungsfunktionen anwenden.
    Die Stärke von CNNs liegt in ihrer Fähigkeit, hierarchische Merkmale in den Daten zu erfassen. 
    Die früheren Schichten lernen einfache Merkmale wie Kanten und Ecken, während die späteren Schichten komplexere Merkmale und abstrakte Darstellungen lernen. 
    Durch das Training des Netzwerks mit großen Datensätzen können CNNs die Gewichtungen ihrer Neuronen so anpassen, dass sie die gewünschte Aufgabe, wie beispielsweise die Klassifizierung von Bildern, mit hoher Genauigkeit erfüllen können.
    CNNs haben eine breite Anwendungspalette in der Bildverarbeitung und sind in der Lage, komplexe Aufgaben wie Objekterkennung, Gesichtserkennung, Semantische Segmentierung und Bildgenerierung zu bewältigen. 
    Ihre Leistungsfähigkeit beruht auf der Fähigkeit, automatisch relevante Merkmale aus den Daten zu extrahieren und sie in einer hierarchischen Weise zu verarbeiten.
    In den nächsten Abschnitten werden wir uns genauer mit den einzelnen Komponenten von CNNs befassen, ihre Funktionsweise im Detail untersuchen und verschiedene Architekturen und Techniken kennenlernen, die in der Praxis weit verbreitet sind.
    \footfullcite{lecun1998gradient,krizhevsky2012imagenet,simonyan2014very,he2016deep}

[Graph mit 3 Kreisen: KI > ML > DL]

    \subsection{Definition und Anwendungen von Convolutional Neural Networks (CNNs)}
    
        Convolutional Neural Networks (CNNs) sind eine Art von künstlichen neuronalen Netzwerken, die speziell für die Verarbeitung von Daten mit räumlicher Struktur entwickelt wurden. 
        Sie zeichnen sich durch ihre Fähigkeit aus, hierarchische Muster und Merkmale in komplexen Daten wie Bildern, Videos oder Tonaufnahmen zu erkennen. 
        CNNs haben in den letzten Jahren enorme Fortschritte in verschiedenen Bereichen der Informatik gemacht, insbesondere in der Bildverarbeitung, Computer Vision und Mustererkennung.
        
        \subsubsection{Einsatz von Convolutional Neural Networks in der Bildverarbeitung}
    
            Convolutional Neural Networks (CNNs) spielen eine entscheidende Rolle bei der automatischen Analyse und Verarbeitung von visuellen Daten in der Bildverarbeitung. 
            Durch ihre einzigartige Architektur sind sie in der Lage, Bilder in ihre Bestandteile zu zerlegen und räumliche Merkmale wie Kanten, Formen und Texturen zu extrahieren. 
            Dies geschieht durch die Kombination mehrerer Convolutional Layers und Pooling Layers, wodurch komplexe visuelle Muster erlernt und hochdimensionale Daten effektiv verarbeitet werden können. Als Ergebnis haben sich zahlreiche Anwendungen wie Objekterkennung, Gesichtserkennung, semantische Segmentierung und Bildgenerierung entwickelt. 
            Die fortschrittlichen Fähigkeiten von CNNs haben die Leistungsfähigkeit von Bildverarbeitungssystemen erheblich verbessert und zu bahnbrechenden Fortschritten in Bereichen wie autonomes Fahren, medizinische Bildgebung und Überwachungstechnologie geführt.
        
        \subsubsection{Anwendungen von Convolutional Neural Networks in der Computer Vision}
        
            Die Computer Vision ist ein interdisziplinäres Forschungsfeld, das sich mit der Entwicklung von Algorithmen und Techniken zur Erfassung, Interpretation und Verarbeitung von visuellen Informationen befasst. 
            In diesem Bereich haben Convolutional Neural Networks (CNNs) eine Revolution ausgelöst, da sie die Fähigkeit besitzen, automatisch relevante visuelle Merkmale aus Bildern oder Videosequenzen zu extrahieren. 
            Durch das Training mit großen Datensätzen können CNNs lernen, Objekte und Szenen zu erkennen, Gesichter zu identifizieren, Gesten zu verstehen und komplexe visuelle Aufgaben zu lösen. 
            Die Anwendungen von CNNs in der Computer Vision sind vielfältig und umfassen Bereiche wie automatische Fahrzeugerkennung, Augmented Reality, Robotik und intelligente Überwachungssysteme. 
            Durch den Einsatz von CNNs wurden die Grenzen der Computer Vision erweitert und Computer sind nun in der Lage, die visuelle Welt um uns herum besser zu verstehen und mit ihr zu interagieren.
        
        \subsubsection{Mustererkennung und Klassifikation mit Convolutional Neural Networks}
    
            Ein weiterer bedeutender Anwendungsbereich von Convolutional Neural Networks (CNNs) liegt in der Mustererkennung und Klassifikation. 
            Durch das Training mit annotierten Datensätzen sind CNNs in der Lage, Muster und Zusammenhänge in den Daten zu erkennen und verschiedene Klassen oder Kategorien von Objekten zu unterscheiden. 
            Dies ermöglicht die automatische Klassifizierung von Bildern, Texten, Tonaufnahmen und vielen anderen Datenarten. 
            In den Bereichen der Texterkennung, Spracherkennung, Sprachübersetzung und anderen Bereichen der Mustererkennung haben CNNs bahnbrechende Fortschritte erzielt. 
            Sie demonstrieren bemerkenswerte Fähigkeiten zur Bewältigung komplexer Klassifikationsaufgaben mit hoher Genauigkeit und haben somit die Effizienz und Genauigkeit von maschinellen Lernsystemen in erheblichem Maße verbessert.
            
        \subsubsection{Weitere Anwendungen}
    
    Neben den oben genannten Bereichen finden Convolutional Neural Networks auch in vielen anderen Bereichen Anwendung. Hier sind einige Beispiele:
    \begin{itemize}
    \item Medizinische Bildgebung: CNNs werden verwendet, um medizinische Bilder wie Röntgenaufnahmen, MRT-Scans und CT-Scans zu analysieren und Krankheiten zu diagnostizieren. Sie können Tumore, Anomalien oder andere gesundheitliche Zustände identifizieren, was Ärzten bei der genauen Diagnose und Behandlung unterstützt.

    \item Sprachverarbeitung: CNNs können in der automatischen Spracherkennung und Sprachsynthese eingesetzt werden. Sie können Audiosignale analysieren und Transkriptionen von gesprochener Sprache generieren. Dies ermöglicht Anwendungen wie Sprachsteuerungssysteme, Sprachassistenten und Untertitelungsdienste.

    \item Finanzanalyse: CNNs werden zur Vorhersage von Finanzmärkten und zur Erkennung von betrügerischen Transaktionen eingesetzt. Sie können komplexe Muster in historischen Finanzdaten erkennen und Modelle entwickeln, um zukünftige Trends oder Risiken vorherzusagen.

    \item Naturwissenschaften: In den Naturwissenschaften werden CNNs zur Analyse von großen Datensätzen aus Bereichen wie Astronomie, Genetik und Teilchenphysik eingesetzt. Sie können dabei helfen, Muster, Zusammenhänge und neue Erkenntnisse in komplexen wissenschaftlichen Daten zu identifizieren.

    \end{itemize}
    
    Insgesamt bieten Convolutional Neural Networks eine vielfältige Bandbreite an Anwendungen, die von der Bildverarbeitung über die Computer Vision bis hin zur Mustererkennung reichen. Ihre Fähigkeit, komplexe Muster in hochdimensionalen Daten zu erfassen, hat die Möglichkeiten der automatisierten Datenverarbeitung und -analyse erweitert und damit zahlreiche innovative Lösungen in verschiedenen Bereichen ermöglicht.
    
\subsection{Wie unterscheiden sich CNNs von anderen neuronalen Netzwerkarchitekturen?}

    Convolutional Neural Networks (CNNs) stellen eine spezifische Architektur von künstlichen neuronalen Netzwerken dar, die sich in einigen wesentlichen Punkten von anderen Netzwerkarchitekturen unterscheiden. 
    Im Folgenden werden zwei wichtige Unterschiede hervorgehoben:

\subsubsection{Räumlich-sensitive Neuronen und feste Gewichtungen}

    Im Gegensatz zu vollständig verbundenen neuronalen Netzwerken verwenden CNNs räumlich-sensitive Neuronen und feste Gewichtungen, um lokale Muster in den Eingabedaten zu erkennen. 
    Bei einem vollständig verbundenen Netzwerk sind alle Neuronen einer Schicht mit allen Neuronen der nächsten Schicht verbunden. 
    Dies führt dazu, dass die räumliche Struktur der Daten nicht berücksichtigt wird und das Netzwerk möglicherweise nicht in der Lage ist, lokale Muster und Zusammenhänge effektiv zu erfassen. 
    CNNs hingegen verwenden sogenannte Filter oder Kernel, die über das Eingabebild verschoben werden, um lokale Merkmale zu extrahieren. 
    Diese Filter haben feste Gewichtungen, die während des Trainings angepasst werden, um auf spezifische Muster zu reagieren. 
    Durch die Verwendung räumlich-sensitiver Neuronen und fester Gewichtungen sind CNNs in der Lage, lokalisierte Merkmale wie Kanten, Texturen und spezifische Formen zu erfassen.

    \subsubsection{Pooling-Operationen zur Dimensionalitätsreduktion und Erhöhung der Robustheit}
    
        Ein weiterer Unterschied besteht darin, dass CNNs Pooling-Operationen verwenden, um die Dimensionalität der Eingabedaten zu reduzieren und die Robustheit gegenüber Translationen zu erhöhen. 
        Nach den Convolutional Layers folgen in einem CNN typischerweise Pooling Layers. 
        In diesen Schichten werden lokal aggregierte Informationen aus den vorherigen Schichten extrahiert, indem beispielsweise der maximale Wert (Max-Pooling) oder der Durchschnittswert (Average-Pooling) innerhalb eines bestimmten Bereichs ausgewählt wird. 
        Durch das Anwenden von Pooling-Operationen wird die räumliche Auflösung der Merkmalskarten reduziert, während wichtige Informationen beibehalten werden. 
        Dies führt zu einer effizienteren Verarbeitung der Daten und einer erhöhten Robustheit gegenüber kleinen Translationen in den Eingabedaten. 
        Diese Eigenschaft ist besonders vorteilhaft, wenn die genaue Position der Merkmale in den Daten nicht von entscheidender Bedeutung ist.
        Die Verwendung räumlich-sensitiver Neuronen und fester Gewichtungen sowie die Integration von Pooling-Operationen sind charakteristische Merkmale von Convolutional Neural Networks, die es ihnen ermöglichen, räumliche Informationen zu berücksichtigen, lokale Muster zu erkennen und komplexe visuelle Aufgaben effektiv zu bewältigen.

\subsection{Warum sind Convolutional Neural Networks besonders nützlich für Bild- und Videodaten?}

    Convolutional Neural Networks (CNNs) haben sich als äußerst nützlich und effektiv bei der Verarbeitung von Bild- und Videodaten erwiesen. Im Folgenden werden einige Gründe dafür erläutert:

\subsubsection{Automatische Merkmalsextraktion}

    CNNs haben die Fähigkeit, automatisch Merkmale aus Bildern und Videos zu extrahieren, ohne dass eine manuelle Merkmalsextraktion erforderlich ist. 
    Traditionelle Ansätze zur Verarbeitung von visuellen Daten erforderten oft die Definition und Konstruktion von spezifischen Merkmalsvektoren durch Experten. 
    Im Gegensatz dazu können CNNs lernen, hierarchische Merkmale direkt aus den Rohdaten zu extrahieren. 
    Durch die Kombination von Faltungs- und Pooling-Schichten sind CNNs in der Lage, lokale Muster wie Kanten, Texturen und Formen zu erkennen, die zur Identifizierung von Objekten und zur Klassifizierung von Bildern wesentlich sind.

\subsubsection{Lernen räumlicher Hierarchien von Merkmalen}

    Die Verwendung von faltenden Schichten ermöglicht es CNNs, räumliche Hierarchien von Merkmalen zu lernen, was sie besonders effektiv für die Objekterkennung und Klassifizierung macht. 
    Durch die schrittweise Anwendung von Faltung und Pooling in aufeinanderfolgenden Schichten können CNNs komplexe visuelle Konzepte erfassen. 
    Anfängliche Schichten lernen einfache Merkmale wie Kanten und Farbkontraste, während tiefere Schichten komplexere Merkmale wie Formen, Strukturen und Objekte auf höheren Abstraktionsebenen erkennen können. 
    Diese Hierarchie von Merkmalen ermöglicht es CNNs, Objekte mit hoher Genauigkeit zu identifizieren und Bilder korrekt zu klassifizieren.

\subsubsection{Anpassung auf weitere Anwendungsbereiche}

    Ein weiterer Vorteil von CNNs ist ihre Fähigkeit, ohne viel Aufwand und vortrainiertes Wissen auf verschiedene Anwendungsbereiche angepasst zu werden. 
    Ein bereits vortrainiertes CNN für die Segmentierung von Menschen kann beispielsweise für die Erkennung von Objekten wiederverwendet werden. 
    Indem man das Netzwerk auf neue Daten feinabstimmt (Fine-Tuning), kann es auf spezifische Aufgaben oder Domänen angepasst werden, ohne von Grund auf neu trainiert werden zu müssen.
    Dies spart Zeit und Ressourcen und ermöglicht es, CNNs für eine Vielzahl von Bild- und Videodaten-Anwendungen einzusetzen.

\subsubsection{Toleranz gegenüber Translationen, Skalierungen und Verzerrungen}

    CNNs sind auch in der Lage, Translationen, Skalierungen und Verzerrungen in den Eingabedaten zu tolerieren, was für die Verarbeitung von Bildern und Videos von Vorteil ist. 
    Aufgrund der Verwendung von Faltungsschichten und Pooling-Operationen sind CNNs invariant gegenüber kleinen räumlichen Verschiebungen in den Eingabedaten. 
    Dies bedeutet, dass ein Objekt, das sich leicht innerhalb eines Bildes bewegt oder skaliert oder verzerrt wird, immer noch von einem CNN erkannt und klassifiziert werden kann. 
    Dies ist besonders vorteilhaft für Anwendungen wie die Objekterkennung in Videos oder die Klassifizierung von Bildern, bei denen die Position, Größe oder Ausrichtung der Objekte variieren können.
    Durch die Toleranz gegenüber solchen Variationen wird die Robustheit und Zuverlässigkeit von CNNs in der Verarbeitung von Bild- und Videodaten verbessert. 
    Sie können auch mit unterschiedlichen Auflösungen von Bildern umgehen und sind in der Lage, Informationen auf verschiedenen Skalenebenen zu extrahieren.
    Diese Fähigkeiten machen CNNs zu einem effektiven Werkzeug für eine Vielzahl von Bild- und Videodaten-Anwendungen. 
    Von der automatischen Erkennung von Objekten und Gesichtern bis hin zur semantischen Segmentierung und Bildgenerierung haben CNNs die Grenzen der Bildverarbeitung und Computer Vision erweitert und bahnbrechende Fortschritte in Bereichen wie autonomes Fahren, medizinische Bildgebung und Überwachungstechnologie ermöglicht.
    Insgesamt sind Convolutional Neural Networks aufgrund ihrer automatischen Merkmalsextraktion, des Lernens räumlicher Hierarchien von Merkmalen, der Anpassungsfähigkeit auf verschiedene Anwendungsbereiche und der Toleranz gegenüber Translationen, Skalierungen und Verzerrungen besonders nützlich für die Verarbeitung und Analyse von Bild- und Videodaten.

\section{Anwendungen von CNNs}

	Convolutional Neural Networks (CNNs) finden in vielen Anwendungsgebieten Anwendung, insbesondere in der Bildverarbeitung und Mustererkennung.
	Durch ihre Fähigkeit, komplexe Merkmale zu erkennen und Bilder automatisch zu klassifizieren, haben sie bahnbrechende Fortschritte in verschiedenen Bereichen erzielt.

\subsection{Bildklassifikation}

    Bildklassifikation ist eine der grundlegenden Anwendungen von CNNs. 
    Mit Hilfe von trainierten Modellen können CNNs Bilder automatisch in verschiedene Kategorien klassifizieren, wie beispielsweise Hund vs. Katze, Auto vs. Fahrrad usw. Diese Fähigkeit beruht auf der Nutzung tieferer Schichten in einem CNN. 
    Durch diese tieferen Schichten können komplexe Merkmale wie Texturen, Formen und Strukturen erkannt werden, was zu einer verbesserten Bildklassifikation führt. 
    Indem CNNs lernen, spezifische Merkmale auf verschiedenen Abstraktionsebenen zu erfassen, können sie komplexe visuelle Muster in Bildern effizient identifizieren und analysieren.

\subsection{Objekterkennung}

	CNNs werden häufig für die Erkennung und Lokalisierung von Objekten in Bildern eingesetzt.
	Durch die Nutzung tieferer Schichten können CNNs komplexe Merkmale wie Texturen, Formen und Strukturen erkennen, was die Genauigkeit der Bildklassifikation verbessert.
	Zusätzlich ermöglicht der Einsatz von Region Proposal Networks (RPNs) den CNNs, die genauen Begrenzungsrahmen der erkannten Objekte zu berechnen.

\subsection{Gesichtserkennung}

	CNNs haben in der Gesichtserkennung signifikante Fortschritte gemacht und werden häufig für die Identifizierung von Personen in Bildern und Videos eingesetzt.
	Durch den Einsatz von CNNs können Gesichtsmerkmale wie Augen, Nase und Mund erkannt und zur Identifizierung von Personen verwendet werden.
	Diese Technologie findet Anwendung in Zugangskontrollen, Überwachungssystemen und biometrischer Authentifizierung.

\subsection{Natural Language Processing (NLP)}

	Obwohl CNNs hauptsächlich für die Verarbeitung von Bildern entwickelt wurden, können sie auch in bestimmten Anwendungen des Natural Language Processing (NLP) eingesetzt werden.
	In der NLP können CNNs zur Klassifizierung von Texten, Sentimentanalyse, maschinellen Übersetzung und Textgenerierung eingesetzt werden.
	Durch die Anwendung von Faltungsschichten auf Textsequenzen können CNNs relevante Merkmale extrahieren und wichtige Informationen für die Klassifizierung liefern.

\subsection{Weitere Anwendungen}

	Neben den oben genannten Anwendungen finden CNNs in vielen anderen Bereichen Anwendung, wie zum Beispiel in der medizinischen Bildgebung, bei autonomen Fahrzeugen und der Spracherkennung.
	Sie ermöglichen die Automatisierung und Verbesserung verschiedener Aufgaben, indem sie Muster und Zusammenhänge in den Daten erkennen.
	Mit der kontinuierlichen Weiterentwicklung von CNNs eröffnen sich ständig neue Anwendungsmöglichkeiten in verschiedenen Branchen.

\section{Architektur von CNNs}

\subsection{Input Layer}

Die Input Layer eines Convolutional Neural Networks (CNNs) spielt eine entscheidende Rolle bei der Aufnahme und Vorverarbeitung der Rohdaten, die in der Regel Bilder oder Videos sind. 
Sie ist die erste Schicht des Netzwerks und ermöglicht den Datenfluss in das CNN.
Die Eingabeschicht besteht aus einer Anordnung von Neuronen, wobei jedes Neuron mit einem bestimmten Bereich des Eingabebildes verbunden ist. 
Diese Verbindungen stellen sicher, dass jedes Neuron Informationen aus einem begrenzten lokalen Bereich des Bildes erhält. Auf diese Weise kann das CNN räumliche Informationen über die Daten berücksichtigen und lokalisierte Merkmale erfassen.
Die Größe der Input Layer ist normalerweise fest vorgegeben, um eine konsistente Verarbeitung der Daten sicherzustellen. 
Für RGB-Bilder beträgt die typische Größe beispielsweise 244x244x3, wobei die ersten beiden Dimensionen die Breite und Höhe des Bildes repräsentieren und die letzte Dimension die Farbkanäle (Rot, Grün, Blau) darstellt.
Die Input Layer spielt eine wichtige Rolle bei der Einführung der Daten in das CNN und ermöglicht die weitere Verarbeitung und Extraktion von Merkmalen in den nachfolgenden Schichten. 
Durch die Struktur der Input Layer können CNNs komplexe visuelle Muster in den Daten erkennen und in der Lage sein, komplexe Aufgaben wie Objekterkennung, Gesichtserkennung und Semantische Segmentierung effektiv zu bewältigen.


\subsection{Hidden Layers}

%https://pyimagesearch.com/2021/05/14/convolutional-neural-networks-cnns-and-layer-types/
%TODO: Grafiken einbauen

Die Hidden Layers bilden die Hauptkomponente eines Convolutional Neural Networks (CNNs) und bestehen standardmäßig aus Convolutional-Layern, Pooling-Layern, vollständig verbundenen Layern und Aktivierungsfunktionen. 
Diese Schichten spielen eine entscheidende Rolle bei der Verarbeitung und Extraktion von Merkmalen aus den Eingabedaten.
Die Convolutional-Layer führen Faltungsoperationen auf den Eingabedaten durch, um lokale Muster und Merkmale zu extrahieren. Durch die Anwendung von Filtern auf die Eingabedaten werden wichtige Informationen hervorgehoben und irrelevante Details unterdrückt. 
Dies ermöglicht es dem CNN, Merkmale wie Kanten, Texturen und Formen zu erkennen, die für die spätere Klassifizierung oder Segmentierung von entscheidender Bedeutung sind.
Die Pooling-Layer sind für die Reduzierung der Dimensionalität der Daten verantwortlich. 
Sie ermöglichen es, die Größe der Merkmalskarten zu verkleinern und die Anzahl der Parameter im Modell zu reduzieren. 
Der Hauptzweck des Pooling besteht darin, die Rechenkomplexität des Modells zu verringern, indem weniger Merkmale beibehalten werden. 
Durch das Zusammenfassen von Informationen wird auch die Lokalisierungsinvarianz verbessert, da die genaue Position eines bestimmten Merkmals in den Merkmalskarten weniger wichtig wird.
Jedoch erhöhen Pooling-Layer nicht unbedingt die Translationssicherheit. Während des Pooling-Prozesses kann ein gewisser Informationsverlust auftreten, da nur die dominanten Merkmale beibehalten werden. 
Dadurch können Feinheiten und detaillierte Informationen verloren gehen, die möglicherweise für die genaue Klassifizierung oder Segmentierung von Objekten von Bedeutung sind.
Insgesamt sind Pooling-Layer eine wichtige Komponente von CNNs, da sie die Dimensionalität reduzieren und die Rechenressourcen effizienter nutzen können. 
Es ist jedoch wichtig, sie mit Bedacht einzusetzen, abhängig von den Anforderungen der spezifischen Aufgabe und dem Trade-off zwischen Dimensionsreduktion und dem Erhalt wichtiger Informationen. 
Die Auswahl der richtigen Pooling-Strategie und -Parameter ist entscheidend, um ein optimales Gleichgewicht zwischen Dimensionalitätsreduktion und Informationsbewahrung zu erreichen.

\subsection{Output Layers}

Die Output Layers bilden die letzte Schicht eines Convolutional Neural Networks (CNNs) und sind für die Generierung der endgültigen Vorhersage, Klassifizierung oder Segmentierung verantwortlich. 
Die Ausgabeschicht ist entscheidend für die Interpretation der durch das CNN ermittelten Merkmale und ermöglicht die Ausgabe der gewünschten Ergebnisse.
Ähnlich wie bei den Hidden Layers führen Convolutional-Layer auch in der Ausgabeschicht Faltungsoperationen auf den vorverarbeiteten Eingabedaten durch. 
Diese Operationen dienen dazu, die relevanten Merkmale zu extrahieren und sie für die weitere Verarbeitung und Interpretation verfügbar zu machen. Die in den vorherigen Schichten erlernten Merkmale werden hier zusammengeführt, um eine aussagekräftige Vorhersage oder Klassifizierung zu ermöglichen.
Die Form der Ausgabeschicht variiert je nach Anwendungsfall. Bei binärer Klassifizierung besteht die Ausgabeschicht in der Regel aus einem einzelnen Neuron, das eine Wahrscheinlichkeit oder Entscheidung für eine der beiden Klassen ausgibt. 
Bei Multi-Klassen-Klassifizierung hingegen werden mehrere Neuronenausgaben verwendet, um die Wahrscheinlichkeiten oder Entscheidungen für jede einzelne Klasse zu liefern.
In unserem spezifischen Projekt ist die Ausgabeschicht ein weiteres Bild, das genau die gleiche Größe wie das Eingangsbild hat. 
Dies ermöglicht die Segmentierung des Eingangsbildes in verschiedene Bereiche oder die Generierung eines verarbeiteten Ausgabebildes mit spezifischen Eigenschaften.
Die Ausgabeschicht stellt somit das Endergebnis des CNNs dar und ist entscheidend für die Interpretation und Verwendung der ermittelten Merkmale. 
Durch eine passende Auswahl der Ausgabeschicht können wir die gewünschten Ergebnisse erzielen, sei es eine Klassifizierung, Segmentierung oder die Generierung eines verarbeiteten Ausgabebildes.

\section{Convolutional layers}

\subsection{Pooling-Layers}
\begin{itemize}
  \item Pooling-Layers reduzieren die Dimensionalität der Daten, indem sie die Aktivierungswerte in bestimmten Bereichen zusammenfassen.
  \item Typische Pooling-Operationen sind das Max-Pooling und das Average-Pooling, die die maximalen bzw. durchschnittlichen Werte in einem Bereich auswählen.
  \item Bei CNNs werden im allgemeinen Max-Pooling layers genutzt.
  \item Pooling-Schichten helfen dabei, die Anzahl der Parameter im Netzwerk zu reduzieren und die räumliche Invarianz gegenüber kleinen Translationen zu erreichen.
\end{itemize}

\subsection{Fully-Connected-Layers}
\begin{itemize}
  \item Fully Connected Layers sind traditionelle neuronale Netzwerk-Schichten, bei denen alle Neuronen mit allen Neuronen der vorherigen Schicht verbunden sind.
  \item In CNNs werden vollständig verbundene Schichten normalerweise am Ende des Netzwerks verwendet, um die extrahierten Merkmale zu klassifizieren.
  \item Die Anzahl der Neuronen in den vollständig verbundenen Schichten hängt von der Anzahl der Klassen oder der spezifischen Aufgabe ab.
  \item Dropout können die Vernetzung der Neuronen minimieren, sodass kein overfitting passiert.

  "The last layer type we are going to discuss is dropout. Dropout is actually a form of regularization that aims to help prevent overfitting by increasing testing accuracy, perhaps at the expense of training accuracy. For each mini-batch in our training set, dropout layers, with probability p, randomly disconnect inputs from the preceding layer to the next layer in the network architecture." source: siehe link oben
\end{itemize}

\subsection{Aktivierungsfunktionen}
\begin{itemize}
  \item Aktivierungsfunktionen werden auf die Ausgaben der Neuronen angewendet und führen nichtlineare Transformationen durch.
  \item Gängige Aktivierungsfunktionen in CNNs sind die ReLU (Rectified Linear Unit), die Sigmoid-Funktion und die tanh-Funktion.
  \item Aktivierungsfunktionen helfen dabei, die Fähigkeit des Netzwerks zur Modellierung komplexer Zusammenhänge zu verbessern.
\end{itemize}

\subsection{Verlustfunktionen}
\begin{itemize}
  \item Verlustfunktionen messen den Unterschied zwischen den Vorhersagen des Modells und den tatsächlichen Werten der Daten.
  \item In der Klassifizierung werden häufig Verlustfunktionen wie die Cross-Entropy-Loss-Funktion verwendet.
  \item Verlustfunktionen dienen als Grundlage für die Berechnung des Fehlersignals und die Aktualisierung der Gewichte im Netzwerk während des Trainingsprozesses.
\end{itemize}

So sieht es bei uns aus:

Training [   0/10000] ..........0 ) Loss= 1.1146412
Training [   1/10000] ..........1 ) Loss= 1.0547378
Training [   2/10000] ..........2 ) Loss= 1.1281569
Training [   3/10000] ..........3 ) Loss= 1.1044791

...

Training [7077/10000] ..........7077 ) Loss= 0.059984926
Training [7078/10000] ..........7078 ) Loss= 0.073117875
Training [7079/10000] ..........7079 ) Loss= 0.05900609
Training [7080/10000] ..........7080 ) Loss= 0.040167235



% Das soll als fazit dienen, keine ahnung du... 
Indem wir die Architektur und die verschiedenen Schichten eines Convolutional Neural Networks verstehen, können wir die Funktionsweise und die Fähigkeiten dieser leistungsstarken neuronalen Netzwerkarchitektur besser erfassen.

\chapter{Datenverarbeitung (Data Preprocessing)}

Vorab: wir müssten davon nichts machen, da unser Datensatz schon vorbereitet war. Aber stellt sich die frage für uns, ob der Datensatz passt (da könntest auch eine subsection daraus bauen am Ende. Idk, using a pre-aggregated Dataset).

\section{Was sind Daten?}

\section{Bedeutung der richtigen Daten}

\subsection{Datenbereinigung (Data Cleaning)}

    In der Welt der Datenanalyse und des maschinellen Lernens ist die Qualität der Daten von entscheidender Bedeutung. 
    Daten werden oft in verschiedenen Formen und aus verschiedenen Quellen gesammelt, und sie können fehlerhaft, inkonsistent oder unvollständig sein. 
    Um verlässliche und aussagekräftige Ergebnisse zu erzielen, ist es daher wichtig, die Daten zu bereinigen und sicherzustellen, dass sie für Analysen und Modellierung geeignet sind. 
    Dieser Prozess wird als Datenbereinigung oder Data Cleaning bezeichnet.    
    Die Datenbereinigung bezieht sich auf den Prozess der Entfernung von fehlerhaften, inkonsistenten oder unvollständigen Daten aus einem Datensatz. 
    Ziel ist es, die Qualität der Daten zu verbessern und sicherzustellen, dass sie für die weiteren Schritte der Datenanalyse und des maschinellen Lernens verwendet werden können. 
    Eine gründliche Datenbereinigung ist entscheidend, da fehlerhafte oder inkonsistente Daten zu falschen Analysen oder Modellen führen können.    
    Die Schritte der Datenbereinigung umfassen in der Regel das Entfernen von Duplikaten, das Behandeln von fehlenden Werten und das Korrigieren von inkonsistenten Dateneinträgen. 
    Duplikate können die Analyseergebnisse verfälschen und sollten daher entfernt werden. 
    Fehlende Werte sind ein häufiges Problem in Datensätzen und müssen angemessen behandelt werden, entweder durch das Auffüllen der fehlenden Werte oder das Entfernen der betroffenen Datensätze. 
    Inkonsistente Dateneinträge, wie zum Beispiel unterschiedliche Schreibweisen oder Formatierungen, sollten korrigiert werden, um eine einheitliche und konsistente Datenbasis zu gewährleisten.  
    Die Bedeutung der richtigen Datenbereinigung kann nicht unterschätzt werden. 
    Sie trägt maßgeblich zur Zuverlässigkeit und Genauigkeit der Analyseergebnisse bei und bildet die Grundlage für fundierte Entscheidungen. 
    Eine unzureichende Datenbereinigung kann zu falschen Schlussfolgerungen führen und das Vertrauen in die Analyse oder das Modell beeinträchtigen.    
    Insgesamt ist die Datenbereinigung ein wesentlicher Schritt in der Datenverarbeitung, der sicherstellt, dass die Daten von hoher Qualität sind und für Analysen und Modellierungszwecke geeignet sind. 
    Durch das Entfernen von fehlerhaften, inkonsistenten oder unvollständigen Daten wird die Zuverlässigkeit der Ergebnisse verbessert und die Basis für eine fundierte Datenanalyse geschaffen.
    \footfullcite{rahm2000data,chapman2005principles}

\subsection{Datennormalisierung (Data Normalization)}

    Die Datennormalisierung ist ein wichtiger Schritt in der Datenverarbeitung, der dazu dient, die Daten auf eine einheitliche Skala oder Verteilung zu transformieren. 
    Oftmals enthalten Datensätze verschiedene Merkmale mit unterschiedlichen Skalen oder Einheiten. 
    Durch die Normalisierung der Daten werden diese in einen bestimmten Wertebereich gebracht, um Probleme aufgrund von Skalenunterschieden oder unterschiedlichen Einheiten zu vermeiden.
    Der Prozess der Datennormalisierung spielt eine entscheidende Rolle in der Vorverarbeitung von Daten, da er die Grundlage für viele statistische Analysen und maschinelle Lernverfahren bildet. 
    Durch die Normalisierung der Daten können Muster und Zusammenhänge besser erkannt werden, da keine Verzerrungen aufgrund unterschiedlicher Skalen auftreten. 
    Darüber hinaus kann die Normalisierung die Leistung von Algorithmen verbessern und die Konvergenz während des Trainingsprozesses beschleunigen.
    Es gibt verschiedene Methoden zur Datennormalisierung, die je nach Anwendungsfall und Art der Daten verwendet werden können. 
    Eine gängige Methode ist die Min-Max-Normalisierung, bei der die Daten auf einen bestimmten Wertebereich transformiert werden. 
    Dabei werden die Daten auf einen Wertebereich zwischen 0 und 1 skaliert, wobei der kleinste Wert auf 0 und der größte Wert auf 1 abgebildet wird. 
    Diese Methode eignet sich gut, wenn die genaue Verteilung der Daten nicht von großer Bedeutung ist.
    Eine weitere Methode ist die Z-Score-Normalisierung, bei der die Daten auf ihre Standardabweichung transformiert werden. 
    Hierbei werden die Daten so verschoben und skaliert, dass sie einen Mittelwert von 0 und eine Standardabweichung von 1 haben. 
    Diese Methode berücksichtigt die Verteilung der Daten und ist insbesondere dann nützlich, wenn die Daten einer Normalverteilung folgen.
    Eine weitere gängige Methode ist die Skalierung auf den Einheitsvektor, bei der die Daten auf eine Länge von 1 normiert werden. 
    Diese Methode wird häufig in maschinellen Lernalgorithmen verwendet, bei denen die Richtung der Daten von Bedeutung ist, aber nicht die genaue Länge.
    Die Auswahl der geeigneten Normalisierungsmethode hängt von der Art der Daten und den Anforderungen des spezifischen Anwendungsfalls ab. 
    Es ist wichtig, die Daten vor der Normalisierung sorgfältig zu analysieren und zu verstehen, um die richtige Methode auszuwählen.
    Insgesamt spielt die Datennormalisierung eine wichtige Rolle in der Datenverarbeitung, um die Daten auf eine einheitliche Skala oder Verteilung zu bringen. 
    Durch die Anwendung geeigneter Normalisierungsmethoden können Verzerrungen aufgrund von Skalenunterschieden oder unterschiedlichen Einheiten vermieden werden, was zu besseren Analysen und Modellen führt. 
    Es ist entscheidend, die Daten sorgfältig zu analysieren und die richtige Normalisierungsmethode entsprechend den Anforderungen des Anwendungsfalls auszuwählen.

\subsection{Datenaugmentierung (Data Augmentation)}

    Die Datenaugmentierung bezieht sich auf den Prozess der künstlichen Erweiterung des Trainingsdatensatzes durch Anwendung von Transformationen oder Manipulationen auf die vorhandenen Daten. Sie spielt eine wichtige Rolle in der Datenverarbeitung, insbesondere bei begrenzten Datenmengen oder wenn das Modell eine größere Vielfalt an Beispielen lernen soll.    
    Durch die Datenaugmentierung wird die Varianz im Datensatz erhöht, indem verschiedene Transformationen auf die Daten angewendet werden. Dadurch erhält das Modell Zugang zu einer größeren Bandbreite an Datenmustern und kann robuster und vielfältiger werden. 
    Die Datenaugmentierung hilft dabei, Überanpassung (Overfitting) zu vermeiden und die allgemeelle Fähigkeit des Modells zur Verallgemeinerung auf neue Daten zu verbessern.    
    Es gibt verschiedene Techniken der Datenaugmentierung, die je nach Anwendungsfall und Art der Daten angewendet werden können. 
    Ein häufig verwendetes Verfahren ist das Zufällige Zuschneiden (Random Cropping), bei dem zufällige Ausschnitte aus den Bildern genommen werden, um den Trainingsdatensatz zu erweitern. 
    Dadurch wird das Modell in der Lage, Objekte in unterschiedlichen Positionen und Größen zu erkennen.    
    Eine weitere Technik ist das Horizontale Spiegeln (Horizontal Flipping), bei dem die Bilder horizontal gespiegelt werden. Dadurch werden die Daten umgekehrt und das Modell lernt, Objekte aus verschiedenen Blickwinkeln zu erkennen. 
    Diese Technik ist besonders nützlich, wenn die Orientierung der Objekte in den Bildern nicht von Bedeutung ist.    
    Das Hinzufügen von Rauschen (Noise Addition) ist eine weitere Methode der Datenaugmentierung, bei der Rauschen in die Daten eingefügt wird.
    Dies kann helfen, das Modell widerstandsfähiger gegenüber Störungen zu machen und es auf den Umgang mit realen Daten vorzubereiten.    
    Weitere Techniken umfassen das Skalieren, Drehen, Farbveränderungen und das Hinzufügen von Text oder Objekten zu den Bildern. 
    Diese Methoden ermöglichen es, den Trainingsdatensatz zu diversifizieren und sicherzustellen, dass das Modell auf verschiedene Situationen und Variationen vorbereitet ist.    
    Insgesamt spielt die Datenaugmentierung eine bedeutende Rolle in der Datenverarbeitung, da sie die Qualität und Vielfalt der Trainingsdaten verbessert. 
    Durch die Anwendung verschiedener Transformationen auf die Daten kann das Modell robustere und leistungsfähigere Modelle erstellen. 
    Es ist wichtig, die richtigen Techniken der Datenaugmentierung entsprechend der Domäne und des Anwendungsfalls auszuwählen, um die Leistung des Modells zu verbessern.
    \footfullcite{shorten2019data,}

\subsection{Fazit}

    Die richtige Vorverarbeitung der Daten ist entscheidend für den Erfolg von Modellen und Analysen. 
    Durch Datenbereinigung, Datennormalisierung und Datenaugmentierung kann die Qualität und Aussagekraft der Daten verbessert werden, was zu besseren Ergebnissen führt. 
    Die Datenbereinigung ermöglicht es, fehlerhafte oder unvollständige Daten zu


\chapter{Architektur des DeepLabV3+ Modells mit ResNet-50 Backbone}

Das DeepLabV3+ Modell ist eine leistungsstarke Architektur für die semantische Segmentierung von Bildern. In diesem Kapitel werden wir uns näher mit der Architektur des Modells befassen, insbesondere mit dem ResNet-50 Backbone.

\section{Backbone-Netzwerk}
    Das Backbone-Netzwerk ist für die Extraktion aussagekräftiger Merkmale aus dem Eingangsbild verantwortlich. In diesem Fall basiert das Backbone-Netzwerk auf der ResNet-50 Architektur. ResNet-50 ist ein tiefes CNN, das in verschiedenen Computer Vision Aufgaben herausragende Leistung erzielt hat. Es besteht aus mehreren Faltungs­schichten, die in verschiedene Stufen gruppiert sind, wobei Residualverbindungen zwischen ihnen verwendet werden, um das Problem des verschwindenden Gradienten zu lösen. Diese Residualverbindungen ermöglichen es dem Netzwerk, effektiver zu lernen, indem sie Gradienten über Verbindungen mit geringerer Tiefe propagieren.

\section{Prediction-Head}
    Der Prediction-Head nimmt die von dem Backbone-Netzwerk extrahierten Merkmale und generiert die endgültigen Vorhersagen. Im DeepLabV3+ Modell verwendet der Prediction-Head atrous (oder dilatierte) Faltungen mit unterschiedlichen Dilationsraten, um mehrskalige Informationen zu erfassen. Dadurch kann das Modell sowohl detaillierte lokale Informationen als auch Kontextinformationen berücksichtigen.
    
    Das ursprüngliche DeepLabV3+ Modell wird auf dem ImageNet-Datensatz vortrainiert, der Millionen von gelabelten Bildern aus Tausenden von Kategorien enthält. Dieses Vortraining hilft dem Modell, generische visuelle Merkmale zu erlernen, die für spezifische Aufgaben feinabgestimmt werden können.
    
    Im bereitgestellten Code wird das vortrainierte DeepLabV3+ Modell mit dem ResNet-50 Backbone geladen. Das bedeutet, dass das Backbone die anfänglichen Schichten von ResNet-50 umfasst, die für die Merkmalsextraktion verantwortlich sind. Die genauen Schichten und ihre Konfigurationen können recht komplex sein, aber sie beinhalten in der Regel mehrere Faltungsschichten, Pooling-Schichten zur Verkleinerung der Auflösung und Residualverbindungen.
    
    Die Modifikation im Code betrifft die letzte Schicht des Modells, den Klassifizierer. Im ursprünglichen DeepLabV3+ Modell handelt es sich dabei um eine 1x1-Faltungsschicht, die einen Tensor mit den Dimensionen [Batchgröße, Anzahl der Klassen, Höhe, Breite] erzeugt. Die Anzahl der Klassen im Originalmodell beträgt 21, da es auf dem COCO-Datensatz trainiert wurde, der Objekte aus 21 verschiedenen Kategorien enthält.
    
    Im bereitgestellten Code wird die letzte Schicht jedoch durch eine andere 1x1-Faltungsschicht mit einer geänderten Anzahl von Ausgabekanälen ersetzt. Die ursprüngliche Anzahl von Ausgabekanälen beträgt 256, was der Anzahl der vom Backbone-Netzwerk erlernten Merkmale entspricht. In der modifizierten Version des Codes wird jedoch die Anzahl der Ausgabekanäle auf 3 gesetzt, was darauf hinweist, dass das Modell einen Tensor mit den Dimensionen [Batchgröße, 3, Höhe, Breite] ausgibt. Diese Änderung der Anzahl der Ausgabekanäle erfolgt, um das Modell für eine spezifische Aufgabe mit drei Klassen anstelle der ursprünglichen 21 Klassen anzupassen.
    
    Insgesamt handelt es sich bei dem DeepLabV3+ Modell mit ResNet-50 Backbone um eine leistungsstarke Architektur für semantische Segmentierungsaufgaben. Durch die Modifikation der letzten Schicht passt der Code das Modell an, um Pixel in eine von drei Klassen zu klassifizieren.
    
    \textbf{Code-Beispiel:}
    
    \begin{lstlisting}[language=Python]
    Net = torchvision.models.segmentation.deeplabv3_resnet50(pretrained=True)  # Modell laden
    Net.classifier[4] = torch.nn.Conv2d(
        256,
        3,
        kernel_size=(1, 1),
        stride=(1, 1)
    )  # Letzte Schicht auf 3 Klassen ändern
    Net = Net.to(device)
    
    optimizer = torch.optim.Adam(params=Net.parameters(), lr=Learning_Rate)  # Adam-Optimizer erstellen
    \end{lstlisting}
    
\section{Laden des Modells}
    Der Code beginnt damit, das DeepLabV3+ Modell mit ResNet-50 Backbone mithilfe der torchvision-Bibliothek zu laden. DeepLabV3+ ist ein beliebtes Modell für die semantische Segmentierung, d.h., es weist jedem Pixel in einem Eingangsbild eine Klassenbezeichnung zu. Das ResNet-50 Backbone ist ein Typ von Faltungsneuronalem Netzwerk (CNN), das für seine Effektivität bei der Extraktion von Merkmalen aus Bildern bekannt ist.
    
\section{Modifikation der letzten Schicht}
    Anschließend wird die letzte Schicht des geladenen Modells modifiziert. Im Originalmodell ist die letzte Schicht ein Klassifizierer, der eine Wahrscheinlichkeitsverteilung über 21 verschiedene Klassen (wie "Person", "Auto", "Hund", usw.) erzeugt. In diesem Code wird die letzte Schicht jedoch durch eine 1x1-Faltungsschicht ersetzt. Diese Modifikation ändert die Ausgabe des Modells so, dass eine Wahrscheinlichkeitsverteilung über 3 Klassen anstelle von 21 erzeugt wird. Die spezifischen Werte für die Kernelgröße und den Stride bestimmen das Verhalten der Faltung.

\section{Verschieben des Modells auf ein Gerät}
    Das modifizierte Modell wird dann auf ein spezifiziertes Gerät verschoben, das entweder die CPU oder eine GPU sein kann. Dieser Schritt stellt sicher, dass die Berechnungen, die das Modell durchführt, auf dem ausgewählten Gerät durchgeführt werden. Die Nutzung einer GPU kann die Schulung und Inferenz von Deep Learning Modellen erheblich beschleunigen.

\section{Erstellen des Optimierers}
    Ein Optimierer ist ein Algorithmus, der die Parameter des Modells während des Trainingsprozesses anpasst, um die Verlustfunktion zu minimieren. In diesem Code wird ein Adam-Optimierer erstellt, der das Modellparameter (erhalten durch \texttt{Net.parameters()}) und die Lernrate (angegeben durch \texttt{Learning\_Rate}) als Eingabe verwendet. Die Lernrate bestimmt die Schrittgröße, mit der der Optimierer die Modellparameter basierend auf den berechneten Gradienten während der Rückwärtspropagation aktualisiert.
    
    Indem Sie dieser Architektur folgen, haben Sie ein modifiziertes DeepLabV3+ Modell mit ResNet-50 Backbone, das Bilder in eine von drei Klassen segmentieren kann. Die Modellparameter werden während des Trainings mit dem Adam-Optimierer und der angegebenen Lernrate optimiert.



\chapter{Convolutional Neural Network Trainings Prozess}

\section{Was ist Training?}
    Bevor wir uns mit dem Trainingsprozess von Convolutional Neural Networks (CNNs) befassen, ist es sinnvoll, das Konzept des Trainings von neuronalen Netzwerken kurz aufzufrischen. Training bezieht sich auf den Prozess des Anpassens der Gewichte und Bias-Werte eines neuronalen Netzwerks, um eine bestimmte Aufgabe zu erlernen. Im Fall von CNNs besteht das Ziel darin, das Netzwerk auf eine bestimmte Bildklassifizierung oder ein anderes visuelles Erkennungsproblem vorzubereiten.
    
    Der Trainingsprozess ist von entscheidender Bedeutung für die Leistungsfähigkeit von CNNs. Während des Trainings lernt das Netzwerk, relevante Merkmale aus den Trainingsdaten zu extrahieren und Muster zu erkennen. Durch die Optimierung der Gewichte und Bias-Werte kann das Netzwerk lernen, geeignete Entscheidungen zu treffen und präzise Vorhersagen zu treffen.

\section{Trainingsprozess}
    Der Trainingsprozess eines CNNs lässt sich grob in verschiedene Schritte und Phasen unterteilen. Zunächst werden die Trainingsdaten geladen und gegebenenfalls vorverarbeitet, um eine optimale Eingabe für das Netzwerk zu gewährleisten. Dies kann Aufgaben wie das Skalieren der Bilder, das Normalisieren der Daten oder das Anwenden von Data Augmentation-Techniken umfassen.
    Während des Trainingsprozesses werden die Eingabedaten durch das CNN propagiert, wobei die Convolutional-Schichten und Pooling-Schichten verwendet werden, um Merkmale auf verschiedenen Abstraktionsebenen zu extrahieren. Aktivierungsfunktionen wie die ReLU-Funktion werden angewendet, um die Nichtlinearität des Netzwerks zu erhöhen.
    
    Ein entscheidender Schritt ist die Berechnung des Verlusts (Loss), der den Unterschied zwischen den vom Netzwerk vorhergesagten Ausgaben und den tatsächlichen Labels misst. Die Wahl des geeigneten Verlustmaßes hängt von der spezifischen Aufgabe ab, beispielsweise der Kreuzentropie-Verlust für Klassifizierungsaufgaben.
    Um den Verlust zu minimieren und die Gewichte anzupassen, wird der Backpropagation-Algorithmus angewendet. Dabei werden die Gradienten der Verlustfunktion bezüglich der Gewichte berechnet und anschließend mittels eines Optimierungsverfahrens, wie zum Beispiel dem Stochastic Gradient Descent (SGD), verwendet, um die Gewichte in die richtige Richtung zu aktualisieren.
    
    Der Trainingsprozess wird in der Regel über mehrere Epochen durchgeführt, wobei jede Epoche eine Durchlauf der gesamten Trainingsdaten umfasst. Dies ermöglicht es dem Netzwerk, schrittweise zu lernen und seine Leistung zu verbessern. Es ist auch üblich, das Netzwerk regelmäßig auf einem separaten Validierungsdatensatz zu testen, um die Überanpassung (Overfitting) zu vermeiden.

\section{Stochastic Gradient Descent (SGD)}
    Der Stochastic Gradient Descent\footfullcite{amari1993backpropagation} (SGD) ist ein Optimierungsverfahren, das eine zentrale Rolle im Trainingsprozess von CNNs spielt. Im Gegensatz zum Gradientenabstiegsverfahren (Gradient Descent), bei dem der Verlust über den gesamten Trainingsdatensatz berechnet wird, verwendet SGD eine zufällige Teilmenge der Daten, um den Gradienten zu approximieren.
    
    Der Einsatz von SGD hat mehrere Vorteile. Erstens ermöglicht er eine schnellere Berechnung des Gradienten, da nur eine Teilmenge der Daten betrachtet wird. Zweitens macht die zufällige Auswahl der Daten das Verfahren robuster gegenüber lokalen Minima, da das Netzwerk unterschiedliche Datenmuster im Verlauf des Trainingsprozesses betrachtet.
    
    \begin{figure}[h]
        \centering
        \includegraphics[width=0.8\textwidth]{img/gradient_descent.png}
        \caption{Beispiel von Stochastic Gradient Descent mit 30 Lernschriten.}
        \label{fig:Stochastic_Gradient_Descent}
    \end{figure}
    
    Die Anpassung der Lernrate ist ein wichtiger Aspekt im SGD-Algorithmus. Die Lernrate bestimmt die Größe der Aktualisierungen der Gewichte und beeinflusst somit die Konvergenzgeschwindigkeit des Netzwerks. Eine zu hohe Lernrate kann zu instabilen oder schlecht konvergierenden Lösungen führen, während eine zu niedrige Lernrate den Trainingsprozess verlangsamen kann. Es ist daher oft erforderlich, die Lernrate im Laufe des Trainings anzupassen, beispielsweise durch den Einsatz von Lernrateplanern oder adaptiven Methoden wie Adam.

\section{Backpropagation}
    Backpropagation\footfullcite{amari1993backpropagation} ist ein wesentlicher Bestandteil des Trainingsprozesses von neuronalen Netzwerken, einschließlich CNNs. Es handelt sich um ein Verfahren zur Berechnung der Gradienten der Verlustfunktion bezüglich der Gewichte des Netzwerks.
    
    Der Backpropagation-Algorithmus funktioniert, indem er die Fehlerinformationen vom Ausgabeneuron zurück durch das Netzwerk propagiert. Dabei werden die partiellen Ableitungen der Verlustfunktion nach den Gewichten in den Schichten berechnet. Dies ermöglicht es, den Gradienten des Verlusts bezüglich der Gewichte zu bestimmen und somit die Gewichte entsprechend anzupassen.
    
    Dank der effizienten Berechnung des Backpropagation-Algorithmus ist es möglich, CNNs mit vielen Schichten zu trainieren. Die Gradienten werden schichtweise berechnet, wodurch eine effektive Ausbreitung der Fehler ermöglicht wird. Dieser Prozess des Gradientenabstiegs ermöglicht es dem Netzwerk, seine Gewichte so anzupassen, dass der Verlust minimiert wird.

\section{Hyperparameter-Tuning}
    Hyperparameter\footfullcite{mantovani2016hyper} sind Parameter, die nicht direkt aus den Daten gelernt werden, sondern vor dem Trainingsprozess festgelegt werden müssen. Sie haben einen erheblichen Einfluss auf die Leistungsfähigkeit des CNNs und müssen sorgfältig ausgewählt werden.
    
    Das Tuning von Hyperparametern beinhaltet die Suche nach den besten Werten für diese Parameter, um eine optimale Leistung des CNNs zu erzielen. Dies kann durch manuelles Ausprobieren verschiedener Hyperparameterkombinationen oder durch den Einsatz von automatisierten Methoden wie Grid Search oder Bayesian Optimization erfolgen\footfullcite{amari1993backpropagation}.
    
    \begin{figure}[h]
        \centering
        \includegraphics[width=\textwidth]{img/hyperparameter_tuning.png}
        \caption{Hyperparameter tuning.}
        \label{fig:hyperparameter_tuning}
    \end{figure}
    
    Beispiele für Hyperparameter sind die Lernrate, die Anzahl der Schichten und Filter im Netzwerk, die Größe des Mini-Batches, die Dropout-Rate und die Regularisierungsparameter. Das richtige Tuning dieser Hyperparameter kann dazu beitragen, eine bessere Generalisierungsfähigkeit des Netzwerks zu erreichen und Überanpassung zu vermeiden.

\section{Regularisierungstechniken}
    Regularisierungstechniken\footfullcite{ghiasi2018dropblock} sind Methoden, die während des Trainingsprozesses angewendet werden, um die Überanpassung des Netzwerks an die Trainingsdaten zu reduzieren und die allgemeine Leistungsfähigkeit zu verbessern.
    
    Eine gängige Regularisierungstechnik ist die L1- und L2-Regularisierung\footfullcite{ibrahim2023anomaly}, bei der ein Regularisierungsterm zur Verlustfunktion hinzugefügt wird, der die Gewichte des Netzwerks beeinflusst. Dies hilft, die Gewichte zu reduzieren und die Modellkomplexität zu verringern.
    
    Ein weiteres Regularisierungsverfahren ist Dropout\footfullcite{labach2019survey}, bei dem während des Trainings zufällig einige Neuronen deaktiviert werden. Dadurch wird das Netzwerk gezwungen, redundante Merkmale zu lernen und erhöht die Robustheit gegenüber Überanpassung.
    
    Data Augmentation ist eine weitere Regularisierungstechnik, bei der die Trainingsdaten künstlich erweitert werden, indem sie transformiert oder mit Rauschen versehen werden. Dies ermöglicht es dem Netzwerk, mehr Variationen der Daten zu sehen und generalisierbarere Merkmale zu lernen.
    
    Die Anwendung von Regularisierungstechniken während des Trainingsprozesses trägt dazu bei, die Leistungsfähigkeit des CNNs zu verbessern und die Überanpassung an die Trainingsdaten zu reduzieren.

\chapter{Transfer Learning}

\section{Einführung in Transfer Learning}

    Transfer Learning ist eine leistungsstarke Technik im Bereich des Deep Learnings, die es uns ermöglicht, das Wissen von vorgefertigten Modellen zu nutzen und es auf neue Aufgaben anzuwenden.
    Dabei werden die erlernten Repräsentationen oder Parameter eines Modells wiederverwendet und auf eine andere verwandte Aufgabe übertragen. 
    Diese Herangehensweise hat in den letzten Jahren aufgrund ihrer Fähigkeit, Zeit und Rechenressourcen zu sparen und dennoch hohe Leistung bei neuen Aufgaben zu erzielen, erhebliche Aufmerksamkeit und Beliebtheit erlangt.
    Die grundlegende Idee hinter Transfer Learning ist, dass Modelle, die auf großen und vielfältigen Datensätzen wie ImageNet trainiert wurden, allgemeine Merkmale gelernt haben, die für eine Vielzahl von visuellen Erkennungsaufgaben nützlich sind. 
    Anstatt den Lernprozess für eine neue Aufgabe mit begrenzten Daten von Grund auf zu beginnen, können wir unser Modell mit den vorab trainierten Gewichten aus einer verwandten Aufgabe initialisieren. 
    Dadurch besitzt das Modell bereits Wissen über niedrig eingestufte Merkmale, Formen und Muster, was in der neuen Aufgabe von Vorteil sein kann.
    Einer der Hauptvorteile von Transfer Learning besteht darin, das Problem des Datenmangels zu umgehen. 
    Das Sammeln und Markieren großer Datenmengen für jede spezifische Aufgabe kann zeitaufwändig und kostspielig sein. 
    Durch die Nutzung von Transfer Learning können wir jedoch auf die große Menge an markierten Daten zurückgreifen, die für das Vortraining verfügbar sind, und somit den Bedarf an einem großen markierten Datensatz für die Ziel-Aufgabe verringern. 
    Dies ist besonders vorteilhaft in Bereichen, in denen das Erlangen von markierten Daten herausfordernd oder nicht praktikabel ist.
    Ein weiterer Vorteil von Transfer Learning liegt in seiner Fähigkeit zur Generalisierung auf neuen Aufgaben. 
    Indem wir Wissen von vortrainierten Modellen übertragen, nutzen wir effektiv die erlernten Repräsentationen, die aussagekräftige Merkmale erfassen. 
    Diese Fähigkeit zur Generalisierung ermöglicht es dem Modell, auch mit begrenzten Trainingsdaten für die Ziel-Aufgabe gute Leistungen zu erbringen. 
    Es trägt auch zur Reduzierung von Overfitting bei, da das Modell bereits aus einem vielfältigen Datensatz gelernt hat und robuste und diskriminierende Merkmale entwickelt hat.
    Darüber hinaus ermöglicht Transfer Learning, von der Expertise und den Forschungsanstrengungen zu profitieren, die in die Entwicklung vortrainierter Modelle investiert wurden. 
    Viele fortschrittliche Modelle und Architekturen wurden auf groß angelegten Datensätzen vortrainiert und erreichen hohe Genauigkeit bei verschiedenen Benchmark-Aufgaben. 
    Indem man diese vortrainierten Modelle als Ausgangspunkt nutzen, kann man ihre Architekturen und Merkmalsextraktoren verwenden und sie für eine spezifische Aufgabe abstimmen. Dadurch kann der Lernprozess beschleunigt werden.
    \footfullcite{pan2010transfer,yosinski2014transfer}

\section{Pre-Trained Models}

    Ein wichtiger Bestandteil des Transfer Learnings sind vortrainierte Modelle. 
    Diese vortrainierten Modelle sind neuronale Netzwerkmodelle, die auf großen Datensätzen trainiert wurden, in der Regel für eine andere Aufgabe als die aktuelle. 
    Durch das Training auf umfangreichen Datensätzen haben diese Modelle allgemeine Merkmale und Muster erlernt, die für eine Vielzahl von verwandten Aufgaben nützlich sein können.
    Vortrainierte Modelle sind das Ergebnis umfangreicher Trainingsverfahren auf großen Rechenressourcen, um komplexe Muster in den Daten zu erkennen. 
    Typischerweise werden vortrainierte Modelle auf großen Bilderkennungsdatensätzen wie ImageNet trainiert, die Millionen von Bildern mit verschiedenen Klassen umfassen. 
    Während des Trainings lernen diese Modelle, verschiedene visuelle Merkmale wie Kanten, Formen, Texturen und Objekte zu erkennen und zu extrahieren. 
    Durch diese umfangreiche Vorarbeit können vortrainierte Modelle als Ausgangspunkt für andere Aufgaben dienen, indem sie bereits gelernte Merkmale zur Verfügung stellen.
    Die Idee hinter der Verwendung vortrainierter Modelle im Transfer Learning besteht darin, das Wissen und die Merkmale, die in den vortrainierten Modellen enthalten sind, auf neue Aufgaben zu übertragen. 
    Anstatt ein Modell von Grund auf neu zu trainieren, können wir ein vortrainiertes Modell verwenden und es an die spezifischen Anforderungen unserer Aufgabe anpassen. 
    Da vortrainierte Modelle bereits eine gewisse Vorstellung von visuellen Merkmalen haben, können sie uns dabei unterstützen, auch mit begrenzten Daten gute Ergebnisse zu erzielen.
    Bei der Verwendung vortrainierter Modelle müssen wir jedoch beachten, dass die vortrainierte Aufgabe und die aktuelle Aufgabe zusammenpassen sollten.
    Wenn die vortrainierte Aufgabe ähnliche Merkmale oder Konzepte wie die aktuelle Aufgabe beinhaltet, ist die Wahrscheinlichkeit höher, dass das Transfer Learning erfolgreich ist. 
    Zum Beispiel könnte ein vortrainiertes Modell, das auf Bilderkennung trainiert wurde, als Ausgangspunkt für eine Aufgabe der Objekterkennung dienen, da beide Aufgaben visuelle Merkmale nutzen.

\section{Fine-tuning von vortrainierten Modellen}

    Die Feinabstimmung (Fine-tuning) ist ein wichtiger Schritt im Transfer Learning, der es ermöglicht, ein vortrainiertes Modell für eine spezifische Aufgabe anzupassen. 
    Bei der Feinabstimmung wird das vortrainierte Modell zunächst als Ausgangspunkt verwendet und anschließend werden die Gewichte des Modells mithilfe eines kleineren, auf die spezifische Aufgabe zugeschnittenen Datensatzes aktualisiert. 
    Dieser Prozess erlaubt es dem Modell, sich an die spezifischen Merkmale und Nuancen der neuen Aufgabe anzupassen.
    Der erste Schritt bei der Feinabstimmung besteht darin, das vortrainierte Modell zu laden, das auf einer großen, allgemeinen Aufgabe trainiert wurde. Dieses Modell verfügt bereits über ein gewisses Verständnis von visuellen Merkmalen, das durch das Training auf umfangreichen Datensätzen erworben wurde. 
    Anschließend werden die oberen Schichten des Modells, die für die spezifische Aufgabe weniger relevant sind, eingefroren, um zu verhindern, dass sie während des Trainings aktualisiert werden. 
    Dies ermöglicht es uns, die vortrainierten Merkmale beizubehalten, während wir die Gewichte der unteren Schichten des Modells anpassen.
    Der nächste Schritt besteht darin, ein kleineres, auf die spezifische Aufgabe zugeschnittenes Datenset zu verwenden, um das Modell zu trainieren. 
    Da die Anzahl der verfügbaren Daten möglicherweise begrenzt ist, ist es wichtig, Overfitting zu vermeiden und gleichzeitig das Modell an die spezifischen Merkmale der neuen Aufgabe anzupassen. 
    Durch die Verwendung eines kleineren Datensatzes können wir die Rechenressourcen effizienter nutzen und den Trainingsprozess beschleunigen.
    Während des Trainingsprozesses werden die Gewichte der unteren Schichten des Modells aktualisiert, um die Merkmale der neuen Aufgabe besser zu erfassen. 
    Da die oberen Schichten des Modells eingefroren sind, bleiben die bereits erlernten Merkmale erhalten, während die Gewichte der unteren Schichten an die neuen Daten angepasst werden. 
    Dadurch kann das Modell spezifische Muster und Merkmale der neuen Aufgabe erlernen, während es gleichzeitig von den allgemeinen Merkmalen des vortrainierten Modells profitiert.
    Die Feinabstimmung bietet mehrere Vorteile. 
    Erstens ermöglicht sie eine schnellere Konvergenz, da das Modell bereits über eine gute initiale Gewichtung verfügt und somit weniger Trainingsiterationen benötigt werden. 
    Zweitens kann die Leistung des Modells durch die Anpassung an die spezifischen Merkmale der neuen Aufgabe verbessert werden.
    Indem das Modell auf die Besonderheiten der neuen Aufgabe abgestimmt wird, kann es genauere Vorhersagen treffen und bessere Ergebnisse erzielen.

\section{Verwendung von vortrainierten Modellen} % als Merkmalsextraktoren}

    Eine alternative Methode des Transfer Learning besteht darin, das vortrainierte Modell als festen Merkmalsextraktor zu verwenden. 
    Dabei werden die Faltungsschichten des vortrainierten Modells genutzt, um Merkmale aus den Eingabedaten zu extrahieren, die anschließend einem neuen Klassifikator oder Regressor zugeführt werden. 
    Dieser Ansatz bietet verschiedene Vorteile, wie zum Beispiel die Möglichkeit, vortrainierte Modelle auf kleineren Datensätzen einzusetzen.
    Bei dieser Methode werden die Gewichte des vortrainierten Modells beibehalten und die oberen Schichten eingefroren, um sicherzustellen, dass die bereits erlernten Merkmale nicht verändert werden. 
    Die Faltungsschichten des Modells dienen dann als Merkmalsextraktoren, die relevante Informationen aus den Eingabedaten extrahieren. 
    Diese Merkmale werden anschließend als Eingabe für einen neuen Klassifikator oder Regressor verwendet, der auf die spezifische Aufgabe abgestimmt ist.
    Der Vorteil dieser Vorgehensweise liegt darin, dass das vortrainierte Modell bereits über ein umfangreiches Wissen verfügt, das auf großen Datensätzen trainiert wurde. 
    Indem wir diese vortrainierten Merkmalsextraktoren verwenden, können wir von dem bereits erlernten Wissen profitieren, ohne das gesamte Modell neu trainieren zu müssen. 
    Dies ist insbesondere dann vorteilhaft, wenn wir nur über einen begrenzten Datensatz verfügen, da das Training eines Modells von Grund auf mit wenigen Daten schwierig sein kann.
    Ein weiterer Vorteil besteht darin, dass die Merkmalsextraktionsschichten des vortrainierten Modells bereits gute Ergebnisse bei der Erkennung allgemeiner Merkmale erzielen. 
    Dies ermöglicht es uns, auch auf kleineren Datensätzen aussagekräftige Merkmale zu extrahieren und sie als Eingabe für den Klassifikator oder Regressor zu verwenden. 
    Somit können wir die Vorteile des vortrainierten Modells nutzen, um bessere Ergebnisse auf unseren spezifischen Aufgaben zu erzielen.
    Es ist wichtig zu beachten, dass die Verwendung vortrainierter Modelle als Merkmalsextraktoren bestimmte Einschränkungen hat. 
    Da die oberen Schichten des Modells eingefroren sind, können sie nicht auf die spezifischen Merkmale der neuen Aufgabe angepasst werden. 
    Daher ist diese Methode besonders geeignet, wenn die Merkmale der neuen Aufgabe bereits in den vortrainierten Schichten erfasst werden können. 
    In solchen Fällen bietet die Nutzung der vortrainierten Merkmalsextraktoren eine effiziente Möglichkeit, um genaue Vorhersagen zu treffen.

\section{Anwendung von DeepLabV3 ResNet50}

    In diesem Abschnitt wird die konkrete Umsetzung des Transfer Learning in unserem Projekt durch die Verwendung des DeepLabV3 ResNet50-Modells erläutert.
    DeepLabV3 ist eine hochmoderne Architektur, die für semantische Segmentierungsaufgaben entwickelt wurde und für ihre herausragende Leistung in der Bildverarbeitung und Szenenanalyse bekannt ist.
    Das ResNet50 dient als grundlegende Netzwerkstruktur innerhalb des DeepLabV3-Frameworks und nutzt Residual-Lernen, um das Training tiefer faltungs­basierter neuronaler Netzwerke zu erleichtern.
    Die Entscheidung, das DeepLabV3 ResNet50-Modell in unserem Projekt zu verwenden, basierte auf einer Vielzahl von Überlegungen.
    Die Architektur des Modells hat sich insbesondere im Bereich der semantischen Segmentierung als äußerst leistungsstark erwiesen, was eine wesentliche Aufgabe in unseren Projektzielen darstellt.
    Durch den Einsatz des DeepLabV3 ResNet50-Modells kann man das umfangreiche Training und die erlernten Repräsentationen nutzen, die das Modell auf großen Datensätzen erfasst hat.
    Dadurch kann das Netzwerk feine semantische Details in unserem spezifischen Anwendungsbereich erkennen.
    Die Verwendung eines vortrainierten Modells wie DeepLabV3 ResNet50 bietet eine Vielzahl von Vorteilen.
    Ein bemerkenswerter Vorteil besteht darin, dass man das Wissen, das durch vorherige Modelle erlangt wurde, in das Projekt übertragen kann, ohne ein neuronales Netzwerk von Grund auf trainieren zu müssen.
    Das vortrainierte Modell enthält eine Vielzahl von allgemeinen Merkmalen und Mustern, die während der umfangreichen Exposition des Modells gegenüber vielfältigen visuellen Daten während seiner Trainingsphase erlernt wurden.
    Durch die Nutzung dieser erlernten Merkmale kann man die Entwicklungsdauer des Projekts verkürzen und die Rechenbelastung, die mit dem Training eines Modells von Grund auf verbunden ist, verringern.
    Darüber hinaus bietet das DeepLabV3 ResNet50-Modell eine solide Grundlage für die Feinabstimmung, die es ermöglicht, das Netzwerk an spezifische Aufgaben anzupassen.
    Bei der Feinabstimmung werden die vortrainierten Gewichte des Netzwerks beibehalten und an die Feinheiten des Ziel-Datensatzes angepasst.
    Dieser Prozess ermöglicht eine schnellere Konvergenz des Lernprozesses des Modells, verkürzt die Gesamttrainingszeit und verbessert die endgültige Leistung bei der Segmentierungsaufgabe.
    Die Feinabstimmung ermöglicht es auch, das im vortrainierten Modell vorhandene Wissen an die spezifische Anwendungsdomäne anzupassen.
    Dabei können die allgemeinen Merkmale, die vom Modell extrahiert wurden, verfeinert und an die Feinheiten der Zielaufgabe angepasst werden.
    Zusammenfassend bietet die Einbindung des DeepLabV3 ResNet50-Modells in das Projekt mittels Transfer Learning eine solide Grundlage, um eine präzise semantische Segmentierung zu erreichen. 
    Indem man auf das vorhandene Wissen des Modells zurückgreift und die Vorteile der Feinabstimmung nutzt, kann man die Fachkenntnisse von DeepLabV3 ResNet50 nutzen und sie effektiv auf die einzigartige Anwendungsdomäne anpassen. 
    Diese Vorgehensweise beschleunigt nicht nur den Entwicklungsprozess des Projekts, sondern führt auch zu einer hochwertigen Lösung, indem das Modell von seinem umfangreichen Training mit vielfältigen visuellen Daten profitiert.
    Ein entscheidender Aspekt bei der Anwendung des DeepLabV3 ResNet50-Modells besteht darin, dass die vortrainierten Gewichte des Modells eingefroren bleiben, um die extrahierten Merkmale beizubehalten. 
    Die Faltungsschichten des Modells werden verwendet, um Merkmale aus den Eingabedaten zu extrahieren, die dann in einen neuen Klassifizierer oder Regressor eingespeist werden können. 
    Dieser Ansatz ermöglicht es, die Vorteile des vortrainierten Modells als leistungsstarken Feature-Extraktor zu nutzen, während man gleichzeitig einen neuen, auf die spezifische Aufgabe abgestimmten Klassifizierer oder Regressor trainiert.
    Die Verwendung von vortrainierten Modellen als Feature-Extraktoren bietet eine Reihe von Vorteilen. Insbesondere kann man auf kleineren Datensätzen arbeiten, da die vortrainierten Modelle bereits allgemeine Merkmale gelernt haben, die auf verschiedene Aufgaben übertragbar sind. 
    Dadurch wird der Bedarf an großen Trainingsdatensätzen reduziert, was sowohl die Datenbeschaffung als auch die Rechenressourcen erleichtert. 
    Darüber hinaus ermöglicht die Verwendung von vortrainierten Modellen als Feature-Extraktoren eine schnellere Entwicklung von Modellen, da der Schwerpunkt auf der Anpassung des Klassifizierers oder Regressors liegt, anstatt das gesamte Modell von Grund auf neu zu trainieren.
    Insgesamt bietet die Anwendung des DeepLabV3 ResNet50-Modells als Feature-Extraktor eine effektive Methode des Transfer Learning, um hochwertige Ergebnisse in der semantischen Segmentierungsaufgabe zu erzielen. 
    Durch die Kombination der Stärken des vortrainierten Modells und der Anpassung an die spezifische Anwendungsdomäne kann man die Effizienz verbessern und gleichzeitig genaue und zuverlässige Ergebnisse erzielen.

\chapter{Common Challenges and Solutions}

\section{Overfitting and underfitting}
\section{Vanishing and exploding gradients}
\section{Gradient descent optimization}
\section{Solutions to common challenges}


% Ist nur als platzhalter gedacht..

\chapter{Tools and Frameworks for CNN Training}

\section{PyTorch}
Introduce PyTorch as a popular deep learning framework known for its flexibility and dynamic computational graph. Discuss its key features, such as automatic differentiation, GPU acceleration, and extensive support for neural network architectures. Provide examples of PyTorch code snippets to demonstrate its ease of use and showcase its capabilities for CNN training.

\section{TensorFlow}
Discuss TensorFlow, one of the most widely used deep learning frameworks. Explain its static computational graph paradigm and its ability to efficiently utilize GPUs for accelerated training. Discuss TensorFlow's ecosystem, which includes high-level APIs like Keras and TensorFlow 2.0's eager execution mode. Illustrate the usage of TensorFlow through code snippets and highlight its strengths and popularity in the deep learning community.

\section{Keras}
Introduce Keras as a high-level deep learning library that runs on top of TensorFlow, CNTK, or Theano. Emphasize Keras' user-friendly interface, which simplifies the process of building, training, and evaluating neural networks. Discuss its versatility in supporting various neural network architectures, including CNNs, and its integration with popular backends. Provide examples of Keras code snippets to showcase its simplicity and accessibility.

\section{Caffe}
Discuss Caffe, a deep learning framework widely used for its efficiency and speed in CNN training. Explain Caffe's model definition language, which allows easy specification of network architectures. Discuss Caffe's pre-trained models and model zoo, which provide a vast collection of well-performing models for various tasks. Describe its popularity in computer vision applications and provide examples of Caffe code snippets.

\section{Other Popular Frameworks}
Discuss other notable deep learning frameworks that are commonly used for CNN training. Include frameworks such as MXNet, Theano, and Torch. Highlight their unique features, advantages, and any notable use cases. Briefly explain how they compare to the previously discussed frameworks.

\chapter{Conclusion and Future Work}

\section{Summary of the paper}
\section{Key takeaways}
\section{Future research directions}


\clearpage
%%%%%%%%%%%%%%%%%%%%%%%%%%%%%%%%%%%%%%%%%%%%%%%%%%%%%%%%%%%%%%%%%%%%%%%%%%%%%%
%% Descr:       Vorlage für Berichte der DHBW-Karlsruhe, Datei mit Abkürzungen
%% Author:      Prof. Dr. Jürgen Vollmer, vollmer@dhbw-karlsruhe.de
%% $Id: abk.tex,v 1.4 2017/10/06 14:02:03 vollmer Exp $
%% -*- coding: utf-8 -*-
%%%%%%%%%%%%%%%%%%%%%%%%%%%%%%%%%%%%%%%%%%%%%%%%%%%%%%%%%%%%%%%%%%%%%%%%%%%%%%%

\chapter*{Abkürzungsverzeichnis}

% Hier werden die Abkürzungen definiert
\begin{acronym}[DHBW]
    \acro{Abk}[Abk.]{Abkürzung}
    \acro{PPI}[PPIs]{Pixel per Inch}
    \acro{CNN}[CNNs]{Convolutional neural network}
    \acro{H2O}[\ensuremath{H_2O}]{Di-Hydrogen-Monoxid}
    \acro{SR}[SRs]{Super Resolution}
    \acro{GAN}[GANs] {Generative Adversarial Network}
    %TODO Abkürzungen für: 
    %PNG
    %JPG
    %SVG
    %XML
    %CSS
    %GIF
    %EPS
    %BMP
    %PSD
    %TIFF
    %FLOPS
    %PSNR-Y
    %MOS
    %RMSE
\end{acronym}              % Abkürzungsverzeichnis
\listoffigures             % Liste der Abbildungen
\listoftables              % Liste der Tabellen
\lstlistoflistings         % Liste der Listings

% Ab hier beginnt der Anhang
\appendix
\addcontentsline{toc}{chapter}{Anhang}

\addcontentsline{toc}{chapter}{Index}
\printindex

\addcontentsline{toc}{chapter}{Literaturverzeichnis}

% Haben Sie das "biblatex"-Paket nicht installiert, benutzen Sie folgendes:
% Ohne das "biblatex"-Paket (s. bericht.sty) produziert folgendes
% "deutsche" Zitate in Literaturverzeichnissen gemaß der Norm DIN 1505,
% Teil 2 vom Jan. 1984.
% Die Zitatmarken werden alphabetisch nach Verfassern
% sortiert und sind durch abgekürzte Verfasserbuchstaben plus
% Erscheinungsjahr in eckigen Klammern gekennzeichnet.

% \bibliographystyle{alphadin}
% \bibliography{bericht}

%%%%%%%%%%%%%%%%%%%%%%%%%%%%%%%%%%%%%%%5
% BIBLATEX
% Benutzt man das "biblatex"-Paket, muß man folgendes schreiben:
\def\refname{Literaturverzeichnis}
\printbibliography
%%%%%%%%%%%%%%%%%%%%%%%%%%%%%%%%%%%%%%%5

\end{document}
