


\chapter{Evaluation von Skalierungsmethoden}

\section{Qualitätsmetriken zur Bewertung von Skalierungsmethoden}

\subsection{Peak Signal-to-Noise Ratio (PSNR)}

- Definition und Berechnung von PSNR
- Anwendung von PSNR bei der Bewertung von Bildqualität
- Limitationen von PSNR

\subsection{Structural Similarity Index (SSIM)}

- Definition und Berechnung von SSIM
- Anwendung von SSIM bei der Bewertung von Bildqualität
- Vorteile von SSIM im Vergleich zu PSNR
- Limitationen von SSIM

\subsection{Computational Speed}

- Bedeutung von Computational Speed bei der Wahl einer Skalierungsmethode
- Methoden zur Messung von Computational Speed
- Trade-offs zwischen Bildqualität und Computational Speed

\subsection{Root-Mean-Square Error (RMSE)}

- Definition und Berechnung von RMSE
- Anwendung von RMSE bei der Bewertung von Bildqualität
- Limitationen von RMSE

\section{Kriterien zur Wahl der besten Methode}

\subsection{Mean Opinion Score (MOS)}

- Definition von MOS und dessen Bedeutung bei der Bewertung von Bildqualität
- Anwendung von MOS bei der Evaluierung von Skalierungsmethoden
- Limitationen von MOS
- Vergleich von MOS mit anderen Qualitätsmetriken 

\subsection{Bildvergleich und visuelle Bewertung}

- Visuelle Bewertung von Bildern durch menschliche Beobachter
- Methoden zur Bildvergleich (z.B. Side-by-Side-Vergleich, Triple-Stimulus-Test)
- Anwendung von visueller Bewertung bei der Evaluierung von Skalierungsmethoden
- Limitationen und Vorbehalte bei der visuellen Bewertung von Bildern

\subsection{Effektivität und Effizienz}

- Effektivität (Bildqualität) vs. Effizienz (Computational Speed)
- Evaluierung der Trade-offs zwischen Effektivität und Effizienz
- Anwendung von Kosten-Nutzen-Analysen bei der Wahl der besten Skalierungsmethode
- Relevanz von Effektivität und Effizienz in verschiedenen Anwendungsbereichen

\subsection{Zukunftsaussichten und Herausforderungen}

- Technologische Entwicklungen und deren Auswirkungen auf die Evaluierung von Skalierungsmethoden
- Herausforderungen bei der Evaluierung von Skalierungsmethoden in spezifischen Anwendungsbereichen (z.B. medizinische Bildgebung, Videokompression)
- Potenziale von KI-basierten Evaluierungsmethoden
- Ethische und soziale Implikationen der Bewertung von Skalierungsmethoden

\newpage