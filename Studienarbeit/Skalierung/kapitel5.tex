

\chapter{Evaluation von Skalierungsmethoden}

\section{Qualitätsmetriken von Skalierungsmethoden}

    Die Evaluation von Skalierungsmethoden in der Bildverarbeitung bedarf einer facettenreichen Palette an Qualitätsmetriken, welche eine präzise Analyse und Vergleichbarkeit unterschiedlicher Methoden ermöglichen. 
    Im Rahmen dieser Arbeit werden ausgewählte sowie zentrale Qualitätsmetriken für die Bewertung von Skalierungsmethoden erörtert und diskutiert.

    \subsection{\ac{PSNR}}
        Das Peak Signal-to-Noise Ratio \ac{PSNR} ist eine der am häufigsten verwendeten Metriken zur Bewertung der Qualität von Bildern. 
        Es misst die Qualität einer Bildrekonstruktion, indem es den Unterschied zwischen einem Originalbild und einem rekonstruierten Bild berechnet und diesen Unterschied durch das maximale Signal (Peak) und das Rauschen (Noise) im Originalbild teilt. 
        Je höher der PSNR-Wert, desto geringer ist der Unterschied zwischen Original- und Rekonstruktionsbildern und desto besser ist die Qualität der Rekonstruktion.
        \footfullcite{wang2004image}
        
        \subsubsection{Definition und Berechnung von \ac{PSNR}}
            Die Definition von PSNR lautet wie folgt:
            \begin{equation}
            PSNR = 20 \log_{10} \left( \frac{MAX_I}{\sqrt{MSE}} \right) 
            %TODO prüf mal ob die Formel stimmt. Bin verzweifelt und hab die irwo aus stackoverflow geholt
            \end{equation}

            Dabei ist $MAX_I$ der maximal mögliche Pixelwert des Bildes.
            Bei der Darstellung von Pixeln mit 8 Bits pro Abtastwert ist dieser Wert 255.
            Allgemeiner ausgedrückt: Wenn die Samples mit linearer PCM mit B Bits pro Sample dargestellt werden, ist MAXI $2^B - 1$\footfullcite{wiki:Peak_signal-to-noise_ratio}.
            
            % wobei Peak der höchste mögliche Wert des Signals ist, der meist auf 255 bei 8-Bit-Graustufenbildern oder 65535 bei 16-Bit-Graustufenbildern gesetzt wird. %nur so semi verstanden aber die Quellen haben da ünbereingestimmt (stackoverflow help me)

            \begin{equation}
                MSE = \frac{1}{n*m} \sum_{i=0}^{n-1} \sum_{i=0}^{m-1} (I(i,j) - K(i,j))^2
            \end{equation}
            
            Der Mean Squared Error \ac{MSE} ist definiert als der Durchschnitt der quadrierten Unterschiede zwischen jedem Pixel des Originalbildes und dem rekonstruierten Bild.
            Je höher der PSNR-Wert, desto geringer ist der \ac{MSE} und desto besser ist die Qualität der Rekonstruktion.
            
        %TODO Quelle für Formel (wWill kein Stackoverflow link angeben) 
        \subsubsection{Anwendung von \ac{PSNR} bei der Bewertung von Bildqualität}
            Obwohl \ac{PSNR} eine weit verbreitete Methode zur Bewertung der Qualität von Bildern ist, hat sie auch ihre Limitationen\footfullcite{sheikh2006image}.
            Zum einen berücksichtigt sie nur die Fehler zwischen Original- und Rekonstruktionsbildern und vernachlässigt andere Faktoren wie Bildverzerrungen, die durch Komprimierung oder Filterung entstehen können.
            Zum andern ist der \ac{PSNR}-Wert nicht sensitiv für menschlich wahrgenommene Verzerrungen\footfullcite{mittal2012no}, wie z.B. Farbverschiebungen oder Artefakte.
            Trotz dieser Limitationen bleibt \ac{PSNR} eine wichtige Metrik in der Bildverarbeitung und wird oft in der Praxis verwendet, um die Qualität von Bildrekonstruktionen zu bewerten und zu vergleichen.
            \footfullcite{korhonen2012peak}
    \subsection{Structural Similarity Index Measure (SSIM)}

        Der Structural Similarity Index \ac{SSIM} ist eine Metrik zur Bewertung der strukturellen Ähnlichkeit zwischen einem Originalbild und einem rekonstruierten Bild. 
        Im Gegensatz zum PSNR berücksichtigt der \ac{SSIM} nicht nur die Pixelwerte, sondern auch die Struktur und Textur des Bildes. 
        Der \ac{SSIM} berechnet die Ähnlichkeit zwischen den beiden Bildern anhand von drei Faktoren: Helligkeit, Kontrast und Struktur. Die Formel zur Berechnung von \ac{SSIM} ist wie folgt:
        
        \begin{equation}
            \text{SSIM}(x, y) = \frac{{(2\mu_x\mu_y + c_1)(2\sigma_{xy} + c_2)}}{{(\mu_x^2 + \mu_y^2 + c_1)(\sigma_x^2 + \sigma_y^2 + c_2)}}
        \end{equation}

        In dieser Formel repräsentieren \(x\) und \(y\) zwei verglichene Bilder.
        Die \ac{SSIM} misst die strukturelle Ähnlichkeit zwischen diesen Bildern.
        \(\mu_x\) und \(\mu_y\) sind die Mittelwerte von \(x\) und \(y\), während \(\sigma_x^2\) und \(\sigma_y^2\) die Varianzen von \(x\) und \(y\) darstellen.
        \(\sigma_{xy}\) repräsentiert die Kovarianz zwischen \(x\) und \(y\).
        Die Konstanten \(c_1\) und \(c_2\) sind kleine Werte, die zur Stabilisierung der Division hinzugefügt werden
        \footfullcite{chen2011fast}\footfullcite{wiki:Structural_similarity}.

        
        \subsubsection{Anwendung von SSIM bei der Bewertung von Bildqualität}
        
            \ac{SSIM} wird häufig verwendet, um die Qualität von Bildrekonstruktionen zu bewerten. Es hat sich gezeigt, dass \ac{SSIM} besser als PSNR die wahrgenommene Bildqualität wiederspiegelt. 
            Dies liegt daran, dass \ac{SSIM} die strukturelle Ähnlichkeit zwischen den beiden Bildern berücksichtigt, während \ac{PSNR} nur die Differenz der Pixelweite betrachtet.

        \subsubsection{Vorteile von SSIM im Vergleich zu PSNR}
        
            Im Vergleich zum \ac{PSNR} hat \ac{SSIM} mehrere Vorteile. Zum einen berücksichtigt es die strukturelle Ähnlichkeit zwischen den beiden Bildern, was zu einer besseren Bewertung der wahrgenommenen Bildqualität führt. 
            Zum anderen ist \ac{SSIM} in der Lage, Verzerrungen zu erkennen, die durch Kompression oder andere Arten von Bildverarbeitung verursacht werden, während \ac{PSNR} dies nicht tut.
        \footfullcite{5596999}
        
        \subsubsection{Limitationen von SSIM}
        
            Obwohl \ac{SSIM} eine bessere Metrik zur Bewertung der Bildqualität als \ac{PSNR} darstellt, hat es auch seine Limitationen. 
            \ac{SSIM} ist anfällig für Helligkeits- und Kontrastunterschiede zwischen den beiden Bildern und kann bei der Bewertung von stark komprimierten Bildern ungenau sein. 
            Auch bei der Anwendung auf Bilder mit unterschiedlichen Strukturen kann die \ac{SSIM} eine ungenaue Bewertung liefern.
        \footfullcite{1284395}
    \subsection{Mean Opinion Score (MOS)}
        Der Mean Opinion Score \ac{MOS} ist eine subjektive Metrik, die die wahrgenommene Qualität einer Bildrekonstruktion misst.
        MOS basiert auf der Bewertung durch menschliche Beobachter, die gebeten werden, die Qualität von Original- und Rekonstruktionsbildern auf einer Skala von 1 bis 5 oder 1 bis 10 zu bewerten.
        MOS ist eine wichtige Metrik, da sie die subjektive Wahrnehmung der Qualität eines Bildes durch den Betrachter berücksichtigt.
        \footfullcite{sheikh2006image}
        
    \subsection{Peak Signal-to-Noise Ratio der Y-Komponente (PSNR-Y)}
        Die Y-Komponente im YCbCr-Farbraum enthält die Helligkeitsinformationen des Bildes. 
        Das Peak Signal-to-Noise Ratio der Y-Komponente \ac{PSNR-Y} ist eine spezielle Version des \ac{PSNR}, die nur die Helligkeitsinformationen des Originalbildes und des rekonstruierten Bildes berücksichtigt. 
        \ac{PSNR}-Y ist eine wichtige Metrik zur Bewertung der Qualität von Skalierungsmethoden für Graustufen- oder Schwarz-Weiß-Bilder.        
        Diese Qualitätsmetriken sind wichtigeWerkzeuge für diew Bewertung und den Vergleich von Skalierungsmelthoden in der Bildverarbeitung.
        Es ist jedoch wichtig zu beachten, dass keine einzelne Metrik alle Aspekte der Bildqualität abdeckt. Eine umfassende Bewertung sollte mehrere Metriken kombinieren und auch die subjektive Wahrnehmung 
        \footfullcite{huang2010new}
        
    \subsection{Computational Speed}
    
        Die Wahl einer Skalierungsmethode hängt nicht nur von der Bildqualität, sondern auch von der benötigten Rechenleistung ab. 
        Die Computational Speed ist daher ein wichtiger Faktor bei der Auswahl einer Skalierungsmethode.
        Es gibt verschiedene Methoden zur Messung von Computational Speed, wie z.B. die Messung der benötigten Zeit, um ein Bild zu skalieren oder die Berechnung von \ac{FLOPS}. 
        Eine genauere Messung kann durch die Verwendung von Benchmarks erreicht werden, die es ermöglichen, verschiedene Skalierungsmethoden auf derselben Hardware zu vergleichen.
        Jedoch muss bei der Messung der Computational Speed berücksichtigt werden, dass die Leistung des verwendeten Computers die Ergebnisse beeinflussen kann. 
        Ein schnellerer Computer kann eine Methode schneller ausführen als ein langsamerer Computer. 
        Daher ist es wichtig, die Messungen auf einem vergleichbaren Computer durchzuführen und die Ergebnisse zu normalisieren.
        Ein wichtiger Trade-off besteht zwischen der Bildqualität und der Computational Speed. 
        Eine Methode, die eine höhere Bildqualität liefert, benötigt in der Regel mehr Rechenleistung und ist daher langsamer als eine Methode mit niedrigerer Bildqualität. 
        Es ist daher wichtig, die gewünschte Bildqualität und die verfügbare Rechenleistung abzuwägen und eine Methode zu wählen, die den Anforderungen am besten entspricht. 
        Es können auch Techniken wie progressiver Skalierung verwendet werden, um eine gute Bildqualität mit einer angemessenen Geschwindigkeit zu erreichen.
        \footfullcite{xu2017efficient,choi2019lightweight}
        
    \subsection{Root-Mean-Square Error (RMSE)}
    
        Der Root-Mean-Square Error \ac{RMSE} ist eine Metrik zur Bewertung der Qualität von Bildrekonstruktionen. 
        Ein Nachteil von RMSE ist, dass er nur den Fehler zwischen Pixelwerten betrachtet und keine Berücksichtigung von strukturellen Unterschieden im Bild nimmt. 
        Daher wird RMSE oft in Kombination mit anderen Metriken wie \ac{PSNR} und \ac{SSIM} verwendet, um ein umfassenderes Bild der Bildqualität zu erhalten.
        Abschließend ist zu sagen, dass beider Bewertung von Bildqualität die Wahl der Metrik von der spezifischen Anwendung abhängt. 
        Während \ac{PSNR} und \ac{SSIM} für die Bewertung von Bildern in vielen Anwendungen ausreichend sein können, kann in anderen Fällen die subjektive Wahrnehmung des menschlichen Betrachters durch MOS eine bessere Metrik sein. 
        Darüber hinaus ist bei der Wahl einer Skalierungsmethode auch die Messung von Computational Speed und das Abwägen von Trade-offs zwischen Bildqualität und Geschwindigkeit von entscheidender Bedeutung.
        \footfullcite{morad1996role}
        
\section{Kriterien zur Auswahl der Skalierungsmethode} %TODO Hier können wir viele Bilder einbauen um die Vergleichsgrafiken besser zu erklären (@MARC)

    \subsection{Bildvergleich und visuelle Bewertung}
        
        Die Evaluation von Skalierungsmethoden in der Bildverarbeitung erfordert eine umfassende und präzise Analyse verschiedener Kriterien, um die optimale Methode auszuwählen. 
        Ein wesentlicher Aspekt bei dieser Bewertung ist der Bildvergleich und die visuelle Bewertung, die auf der subjektiven Wahrnehmung von menschlichen Beobachtern beruhen.
        Die visuelle Bewertung von Bildern durch menschliche Beobachter ermöglicht eine direkte und subjektive Beurteilung der Bildqualität basierend auf wahrgenommenen visuellen Eigenschaften. 
        Verschiedene Methoden des Bildvergleichs werden eingesetzt, wie beispielsweise der Side-by-Side-Vergleich, bei dem zwei Bilder direkt miteinander verglichen werden, oder der Triple-Stimulus-Test, bei dem ein Originalbild mit zwei rekonstruierten Bildern verglichen wird.
        Diese Methoden des Bildvergleichs und der visuellen Bewertung werden auch bei der Evaluierung von Skalierungsmethoden angewendet. 
        Durch den Vergleich der rekonstruierten Bilder mit dem Originalbild können verschiedene Skalierungsmethoden anhand ihrer visuellen Qualität beurteilt und verglichen werden. 
        dies ermöglicht eine umfangreiche Analyse und eine direkte Gegenüberstellung der unterschiedlichen Methoden.
        Es ist jedoch von entscheidender Bedeutung, die Limitationen und Vorbehalte im Zusammenhang mit der visuellen Bewertung von Bildern zu berücksichtigen. 
        Die subjektive Wahrnehmung kann von Person zu Person variieren und wird auch von anderen Faktoren wie Beleuchtung, Betrachtungsabstand und individuellen Präferenzen beeinflusst. 
        Darüber hinaus können komplexe visuelle Verzerrungen nicht immer präzise durch die visuelle Bewertung erfasst werden. 
        Daher sollten bei der Bewertung von Skalierungsmethoden auch objektive Metriken und weitere Evaluierungskriterien einbezogen werden.
        Insgesamt spielt der Bildvergleich und die visuelle Bewertung eine bedeutende Rolle bei der Evaluierung von Skalierungsmethoden in der Bildverarbeitung.
        Durch die Kombination der subjektiven visuellen Bewertung mit objektiven Metriken können fundierte Entscheidungen getroffen werden, um die optimale Methode gemäß den spezifischen Anforderungen der Anwendung auszuwählen. 
        Die sorgfältige Berücksichtigung der genannten Limitationen trägt dazu bei, zuverlässige und aussagekräftige Ergebnisse zu erzielen und die Qualität von Bildrekonstruktionen bestmöglich zu bewerten.
        \footfullcite{ye2012no,wang2004image,eckert1998perceptual}
    
    \subsection{Effektivität und Effizienz}

        Die Bewertung von Skalierungsmethoden in der Bildverarbeitung erfordert eine umfassende Betrachtung sowohl der Effektivität, also der Bildqualität, als auch der Effizienz, insbesondere der Computational Speed. 
        Es besteht ein inhärenter Trade-off zwischen der erzielten Bildqualität und der benötigten Rechenleistung, der bei der Auswahl der optimalen Skalierungsmethode berücksichtigt werden muss.
        Die Effektivität einer Skalierungsmethode bezieht sich auf die erzielte Bildqualität der rekonstruierten Bilder. 
        Eine Methode, die eine höhere Bildqualität liefert, wird in der Regel auch mehr Rechenleistung benötigen und somit langsamer sein als eine Methode mit geringerer Bildqualität. 
        Daher ist es von großer Bedeutung, die gewünschte Bildqualität und die verfügbare Rechenleistung sorgfältig abzuwägen, um die beste Skalierungsmethode zu wählen.


        \begin{table}[!h]
        \centering
        \begin{tabular}{ccccccc}
            \textbf{Model} & \textbf{PLCC} & \textbf{MAE} & \textbf{RMS} & \textbf{SRCC} & \textbf{KRCC} & \textbf{Computation time} \\ \hline
            PSNR           & 0.9062        & 7.4351       & 9.8191       & 0.8804        & 0.6886        & 1                                     \\
            SSIM           & 0.9253        & 6.9203       & 8.8069       & 0.9014        & 0.7246        & 22.65                                 \\
            MS-SSIM        & 0.8945        & 8.1969       & 10.384       & 0.8619        & 0.6605        & 48.49                                 \\
            SSIMplus       & 0.9732        & 4.3192       & 5.3451       & 0.9349        & 0.7888        & 7.83                                  
        \end{tabular}
        \caption{Leistungsvergleich zwischen PSNR, SSIM, MS-SSIM, VQM, PQR-Tek, DMOS-Tek, JND-VC,
DMOS-VC und SSIMplus einschließlich aller Geräte\footfullcite{Rehman_Zeng_Wang_2015}.}
        \label{tab:Leistungsvergleich}
        \end{table}


        
        Die Evaluierung der Trade-offs zwischen Effektivität und Effizienz erfordert eine gründliche Analyse der Leistungsmerkmale verschiedener Skalierungsmethoden. 
        Hierbei können Kosten-Nutzen-Analysen hilfreich sein, um die Auswirkungen der gewählten Methode auf die Bildqualität und die erforderliche Rechenleistung abzuschätzen. 
        Indem die Kosten in Bezug auf die erzielte Bildqualität bewertet werden, kann die optimale Methode entsprechend den spezifischen Anforderungen des Anwendungsbereichs ausgewählt werden.
        Die Relevanz von Effektivität und Effizienz variiert je nach Anwendungsbereich der Bildverarbeitung. 
        In einigen Fällen, wie beispielsweise medizinischen Bildgebungsverfahren oder hochauflösenden Videoanwendungen, ist eine hohe Bildqualität von größter Bedeutung, selbst wenn dies mit einem höheren Rechenaufwand einhergeht. 
        In anderen Szenarien, wie in Echtzeit-Anwendungen oder mobilen Geräten, kann die Effizienz und die schnelle Verarbeitung der Bilder priorisiert werden, auch wenn dies zu einer gewissen Einbuße der Bildqualität führt.
        Insgesamt ist es von entscheidender Bedeutung, sowohl die Effektivität als auch die Effizienz bei der Wahl der besten Skalierungsmethode zu berücksichtigen. 
        Die richtige Balance zwischen Bildqualität und erforderlicher Rechenleistung zu finden, ermöglicht die optimale Nutzung von Ressourcen und die Erfüllung der spezifischen Anforderungen des jeweiligen Anwendungsbereichs der Bildverarbeitung.
        \footfullcite{zhang2019efficiency}\footfullcite{gu2019comprehensive}\footfullcite{hu2020efficient}

    \subsection{Zukunftsaussichten und Herausforderungen}
    
        Die kontinuierliche Entwicklung von Technologien stellt neue Herausforderungen bei der Evaluierung von Skalierungsmethoden in der Bildverarbeitung dar.
        Fortschritte in der Hardware, wie leistungsstärkere Prozessoren und spezialisierte Beschleuniger, können die Rechenleistung erhöhen und somit neue Möglichkeiten für komplexe Skalierungsalgorithmen eröffnen. Gleichzeitig führen neue Technologien wie Virtual Reality, Augmented Reality und 8K-Bildschirme zu höheren Anforderungen an die Bildqualität und erfordern präzisere Bewertungsmethoden.
        Spezifische Anwendungsbereiche, wie die medizinische Bildgebung oder die Videokompression, stellen weitere Herausforderungen bei der Evaluierung von Skalierungsmethoden dar. 
        In der medizinischen Bildgebung ist die genaue Beurteilung der Bildqualität von entscheidender Bedeutung für die richtige Diagnosestellung und Behandlungsplanung. 
        Hier müssen spezielle Metriken und Evaluierungsverfahren entwickelt werden, um den Anforderungen dieses Bereichs gerecht zu werden. 
        Ebenso erfordert die Videokompression eine sorgfältige Evaluierung der Skalierungsmethoden, um eine ausreichende Qualität der komprimierten Videos zu gewährleisten und gleichzeitig die Effizienz der Kompression zu maximieren.
        Ein vielversprechendes Potenzial für die zukünftige Evaluierung von Skalierungsmethoden liegt in der Anwendung von KI-basierten Evaluierungsmethoden. 
        Durch den Einsatz von maschinellem Lernen und neuronalen Netzwerken können automatisierte Bewertungsalgorithmen entwickelt werden, die die menschliche Wahrnehmung von Bildqualität besser simulieren. 
        Diese KI-basierten Ansätze können eine schnellere und objektivere Evaluierung ermöglichen und den Entwicklungsprozess von Skalierungsmethoden beschleunigen.
        Bei der Bewertung von Skalierungsmethoden sollten jedoch auch ethische und soziale Implikationen berücksichtigt werden. 
        Die Entwicklung von immer leistungsfähigeren Skalierungsmethoden kann auch potenzielle Risiken mit sich bringen, wie zum Beispiel die Manipulation von Bildern oder die Schaffung von gefälschten Inhalten. 
        Es ist daher wichtig, entsprechende Schutzmechanismen zu entwickeln, um den Missbrauch solcher Technologien zu verhindern und die Privatsphäre und Integrität von Personen zu wahren.
        Zusammenfassend lassen sich die Zukunftsaussichten für die Evaluierung von Skalierungsmethoden als vielversprechend betrachten. 
        Durch die Berücksichtigung technologischer Entwicklungen, die Bewältigung spezifischer Herausforderungen in verschiedenen Anwendungsbereichen, die Nutzung von KI-basierten Evaluierungsmethoden und die Beachtung ethischer und sozialer Implikationen können wir die Qualität und Effizienz von Skalierungsmethoden weiter verbessern und gleichzeitig sicherstellen, dass sie verantwortungsvoll und zum Wohl der Gesellschaft eingesetzt werden.
        \footfullcite{zhang2015learning}\footfullcite{blau2018perception}\footfullcite{liu2016ssd,zhang2018split}
\newpage