\chapter{Conclusion and Future Work}

\section{Summary of the paper}

Yes, yes. Very good paper. Will show my kids and wife. They be proud. Yes, yes.

\section{Key takeaways}

\section{Future research directions}
